\documentclass{ctexart}
%\documentclass[a4paper,10pt]{scrartcl}

\usepackage[colorlinks,
            linkcolor=black,
            anchorcolor=blue,
            citecolor=green]{hyperref}
  


\usepackage{multirow}

\usepackage{geometry}
\geometry{left=2.5cm,right=2.5cm,top=2.5cm,bottom=2.5cm}

\hfuzz=\maxdimen
\tolerance=10000
\hbadness=10000
\usepackage{xltxtra}
\usepackage{titlesec}
\titleformat{\section}[hang]{\Huge\bfseries}{第\,\thesection\,章}{1em}{}


%设置section的中文编号
% \usepackage[fntef]{ctexcap}
% \setromanfont[Mapping=tex-text]{Linux Libertine O}


% \setsansfont[Mapping=tex-text]{DejaVu Sans}
% \setmonofont[Mapping=tex-text]{DejaVu Sans Mono}

\title{Computer Organization}
\author{}
\date{}

\begin{document}
\maketitle

\newpage

\tableofcontents

\section{计算机系统概述}





\section{数据的表示和运算}




\section{存储系统}

\subsection{存储器的层次结构}
\subsubsection{存储器的分类}
1、按在计算机中的作用分类

1)主存储器

2)辅助存储器

3)高速缓冲存储器

2、按存储介质分类

3、按存取方式分类

\subsubsection{存储器的性能指标}

存储容量

单位成本

存储速度

读取时间

存取周期

主存带宽

\subsection{存储器的层次化结构}

\subsubsection{多级存储结构}



\subsection{半导体随机存储器}



\subsection{主存储器与CPU的连接}



\subsection{双口RAM和多模块存储器}

\subsection{高速缓冲存储器}

\subsection{虚拟存储器}





\section{指令系统}

\subsection{指令格式}

\subsection{指令寻址方式}

\subsection{CISC和RISC的基本概念}







\section{中央处理器}

\subsection{CPU的功能和基本结构}

\subsubsection{CPU的功能}

指令控制
取指令、分析指令、执行指令

操作控制
指令功能由若干操作信号组合实现,CPU管理并产生由内存取出的每条指令的操作信号,把各种操作信号送往相应的部件,控制这些部件按指令进行动作。

时间控制
对操作进行时间上的控制

数据加工
算术运算和逻辑运算

中断处理
处理异常情况和特殊请求。

\subsubsection{CPU的基本结构}

1、运算器

算术逻辑单元ALU
算术逻辑运算

暂存寄存器
暂存主存读取的数据

累加寄存器ACC
通用寄存器,存储ALU的运算结果。

通用寄存器
AX、BX、CX、DX、SP,存放操作数。SP堆栈指针,指向栈顶。

程序状态字寄存器PSW
溢出标识OP、符号标识SF、零标识ZF、进位标识CF

移位器
移位运算

计数器CT
控制乘除运算的操作步数

2、控制器

程序计数器PC
下一条指令在主存中的存放地址。

指令寄存器IR
保存当前执行的指令

指令译码器
译码操作码字段。

存储器地址寄存器MAR
存放主存单元的地址

存储器数据寄存器MDR
存放主存的写入信息或者读出的信息

时序系统
产生时序信号,由统一时钟CLOCK分频得到。

微操作信号发生器
根据IR和PSW内容及时序信号,产生控制整个计算机系统的各种控制信号。


\subsection{指令执行过程}

\subsubsection{指令周期}

CPU从主存每取出并执行一条指令所需的全部时间。

一条指令周期由若干时钟周期组成。

定长指令周期、不定长指令周期

\subsubsection{指令周期的数据流}

1、取指周期

根据PC中的内容从主存中取出指令代码并存放在IR中,取指时PC+1

2、间址周期
取操作数的有效地址

3、执行周期
根据IR中的指令字的操作码和操作数通过ALU操作产生结果。不同指令执行周期不一样。

4、中断周期
处理中断请求。


\subsubsection{指令执行方案}

1、单指令周期
所有指令都用相同的执行时间完成。

2、多指令周期
不同类型的指令选用不同的执行步骤完成。

3、流水线方案
指令之间可以并行执行的方案。


\subsection{数据通路的功能和基本结构}

\subsubsection{数据通路的功能}
数据通路:数据在功能部件之间传送的路径。

实现CPU内部的运算器与寄存器以及寄存器之间的数据交换。

\subsubsection{数据通路的基本结构}
CPU内部总线方式
所有寄存器的输入端和输出端都在一条或多条公共通路上。

专用数据通路方式
根据指令执行过程中数据和地址的流动方向安排连接线路。

!!!!CPU通路的操作。



\subsection{控制器的功能和工作原理}

\subsubsection{控制器的结构和功能}
控制器从数据总线接受指令信息,从运算器接受指令转移地址,送出指令地址到地址总线。

从主存取一条指令(IR),指出下一条指令在从主存的位置(PC)。
译码指令,产生相应的操作控制信号。
控制CPU、主存、输入和输出设备之间的数据流动方向。


\subsubsection{硬布线控制器}

1、硬布线控制单元图

CU输入信号来源:

(1)IR中指令的操作码输入到控制单元(CU)中,与时钟配合产生不同的控制信号。

(2)时序系统产生机器周期和节拍信号。

(3)执行单元反馈信息(标志)

根据节拍信号,微操作命令发出(控制信号)到CPU内部或外部控制总线上。
\\\\

2、硬布线控制器的时序系统及微操作

(1)时钟周期
时钟信号控制节拍发生器,产生节拍,每个节拍宽度对应一个时钟周期。

(2)机器周期
存取指令的周期为基准。存储字长等于指令字长的前提下,取指周期看做机器周期。

(3)指令周期
 
(4)微操作命令分析

取指周期

PC→AR

1→R

M(MAR)→MDR

OP(IR)→CU

(PC)+1→PC
\\

间址周期
Ad(IR)→MAR

1→R

M(MAR)→MDR
\\

执行周期

(1)非访存指令

(2)访存指令

(3)转移指令

3、CPU的控制方式

(1)同步控制方式

以最长微操作寻列为标准,采取相同的时间间隔和相同数目的节拍,作为机器周期运行不同的指令。

(2)异步控制方式

各部件按自身固有速度工作,通过应答方式进行联络。

(3)联合控制方式

对大部分采用同步控制,小部分采用异步控制的办法。

4、硬布线控制单元设计步骤

\subsubsection{微程序控制器}

1、微程序控制的基本概念

2、微程序控制器组成和工作过程

3、微指令的编码方式

4、微指令的地址形成方式

5、微指令的格式

6、微程序控制单元的设计步骤

7、动态微程序设计和毫微程序设计

8、硬布线和微程序控制器的特点


\subsection{指令流水线}










\section{总线}

\subsection{总线基本概念}

\subsection{总线仲裁}

\subsection{总线操作和定时}

\subsection{总线标准}

















\section{输入/输出系统}
\subsection{ I/O 系统基本概念}
1)外部设备:包括输入输出设备及通过输入输出接口才能访问的外存储设备。

2)接口:在各个外设与主机之间的数据传输时进行各种协调工作的逻辑部件。协调包括传输过程中速度的匹配、电平和格式转换等。

3)输入设备:用于向计算机系统输入命令和文本、数据等信息的部件。
键盘和鼠标是最近部的输入设备。

4)输出设备:用于将计算机系统中的信息输出到计算机外部进行显示、交换等的部件。
显示器和打印机是最基本的输出设备。

5)外存设备:是指除计算机内存及CPU缓存等以外的存储器。硬磁盘、光盘等是最基本的外存设备。

I/O软件:驱动程序、用户程序、管理程序、升级补丁,采用I/O指令和通道指令实现CPU和I/O设备的信息交换。

I/O硬件:外部设备、设备控制器和接口、I/O总线等。通过社会被控制器来控制I/O设备的具体动作,通过I/O接口与主机(总线)相连。

\subsection{外部设备}

\subsubsection{输入设备}
1、键盘

最常用的输入设备
\\\\

2、鼠标

机械式、光电式


\subsubsection{输出设备}

1、显示器

阴极射线管CRT显示器、液晶显示器LCD、LED显示器


2、打印机

针式打印机、喷墨式打印机、激光打印机

\subsubsection{外存储器}

辅助存储器,磁表面存储器

\textcircled{1} 存储容量大,位价位低
\textcircled{2} 介质重复使用
\textcircled{3} 记录信息可以长期保存而不丢失
\textcircled{4} 非破坏性读出

\textcircled{1} 读取速度慢
\textcircled{2} 机械结构复杂
\textcircled{3} 对工作环境要求高

1、磁盘存储器
(1)组成

\textcircled{1} 存储区域
存储区域

磁头数

柱面数

{2} 磁记录原理

{3 }磁盘的性能指标

\textcircled{1}容量
\textcircled{2}记录密度
\textcircled{3}平均存取时间
\textcircled{4}数据传输率

(4)磁盘地址

2、磁盘阵列

3、光盘存储器

4、固态硬盘

\subsection{I/O接口}

\subsection{I/O方式}



\end{document}
