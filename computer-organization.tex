\documentclass{ctexart}
%\documentclass[a4paper,10pt]{scrartcl}

\usepackage[colorlinks,
            linkcolor=black,
            anchorcolor=blue,
            citecolor=green]{hyperref}
  


\usepackage{multirow}

\usepackage{geometry}
\geometry{left=2.5cm,right=2.5cm,top=2.5cm,bottom=2.5cm}

\hfuzz=\maxdimen
\tolerance=10000
\hbadness=10000
\usepackage{xltxtra}
\usepackage{titlesec}
\titleformat{\section}[hang]{\Huge\bfseries}{第\,\thesection\,章}{1em}{}


%设置section的中文编号
% \usepackage[fntef]{ctexcap}
% \setromanfont[Mapping=tex-text]{Linux Libertine O}


% \setsansfont[Mapping=tex-text]{DejaVu Sans}
% \setmonofont[Mapping=tex-text]{DejaVu Sans Mono}

\title{Computer Organization}
\author{}
\date{}

\begin{document}
\maketitle

\newpage

\tableofcontents

\section{计算机系统概述}

\section{数据的表示和运算}

\section{存储系统}

\section{指令系统}

\section{中央处理器}

\section{总线}

\section{输入/输出系统}
\subsection{ I/O 系统基本概念}
1)外部设备:包括输入输出设备及通过输入输出接口才能访问的外存储设备。

2)接口:在各个外设与主机之间的数据传输时进行各种协调工作的逻辑部件。协调包括传输过程中速度的匹配、电平和格式转换等。

3)输入设备:用于向计算机系统输入命令和文本、数据等信息的部件。
键盘和鼠标是最近部的输入设备。

4)输出设备:用于将计算机系统中的信息输出到计算机外部进行显示、交换等的部件。
显示器和打印机是最基本的输出设备。

5)外存设备:是指除计算机内存及CPU缓存等以外的存储器。硬磁盘、光盘等是最基本的外存设备。

I/O软件:驱动程序、用户程序、管理程序、升级补丁,采用I/O指令和通道指令实现CPU和I/O设备的信息交换。

I/O硬件:外部设备、设备控制器和接口、I/O总线等。通过社会被控制器来控制I/O设备的具体动作,通过I/O接口与主机(总线)相连。

\subsection{外部设备}

\subsubsection{输入设备}
1、键盘

最常用的输入设备
\\\\

2、鼠标

机械式、光电式


\subsubsection{输出设备}

1、显示器

阴极射线管CRT显示器、液晶显示器LCD、LED显示器


2、打印机

针式打印机、喷墨式打印机、激光打印机

\subsubsection{外存储器}

辅助存储器,磁表面存储器

\textcircled{1} 存储容量大,位价位低
\textcircled{2} 介质重复使用
\textcircled{3} 记录信息可以长期保存而不丢失
\textcircled{4} 非破坏性读出

\textcircled{1} 读取速度慢
\textcircled{2} 机械结构复杂
\textcircled{3} 对工作环境要求高

1、磁盘存储器
(1)组成

\textcircled{1} 存储区域
存储区域

磁头数

柱面数

{2} 磁记录原理

{3 }磁盘的性能指标

\textcircled{1}容量
\textcircled{2}记录密度
\textcircled{3}平均存取时间
\textcircled{4}数据传输率

(4)磁盘地址

2、磁盘阵列

3、光盘存储器

4、固态硬盘

\subsection{I/O接口}

\subsection{I/O方式}



\end{document}
