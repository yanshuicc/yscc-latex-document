\documentclass{ctexart}
%\documentclass[a4paper,10pt]{scrartcl}

\usepackage[colorlinks,
            linkcolor=black,
            anchorcolor=blue,
            citecolor=green]{hyperref}
  


\usepackage{multirow}

\usepackage{geometry}
\geometry{left=2.5cm,right=2.5cm,top=2.5cm,bottom=2.5cm}

\hfuzz=\maxdimen
\tolerance=10000
\hbadness=10000
\usepackage{xltxtra}
\usepackage{titlesec}
\titleformat{\section}[hang]{\Huge\bfseries}{第\,\thesection\,章}{1em}{}


%设置section的中文编号
% \usepackage[fntef]{ctexcap}
% \setromanfont[Mapping=tex-text]{Linux Libertine O}


% \setsansfont[Mapping=tex-text]{DejaVu Sans}
% \setmonofont[Mapping=tex-text]{DejaVu Sans Mono}

\title{Computer Organization}
\author{}
\date{}

\begin{document}
\maketitle

\newpage

\tableofcontents




\section{绪论}

\section{线性表}

\section{栈和队列}

\section{树与二叉树}



\subsection{树的基本概念}


\subsubsection{树的定义}

树是N(N$\geq$0)个借点的有限集合,N=0时,称为空树。

非空树:

1)有且仅有一个称为根的结点。

2)N大于1时,可以分为m个互不相交的有限集合,每一个集合为子树。


\subsubsection{基本术语}

祖先结点、子孙结点、双亲结点、孩子结点、兄弟结点

结点的度:结点的子结点数。

树的度:树中最大的度。

分支结点(非终端结点):度大于0。

叶子结点(终端结点):度等于0。

层次:从树根开始,根节点为第一层,子结点为第二层。

深度:从根结点开始自顶向下逐层累加。

高度:从叶结点开始从低向上逐层累加。

树的高度(深度):树的最大层数。

有序树:

\subsection{二叉树的概念}

\subsection{二叉树遍历和线索二叉树}

\subsection{树、森林}

\subsection{树与二叉树的应用}

\section{图}

\section{查找}

\section{排序}

\subsection{}









\end{document}
