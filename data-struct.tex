\documentclass{ctexart}
%\documentclass[a4paper,10pt]{scrartcl}

\usepackage[colorlinks,
            linkcolor=black,
            anchorcolor=blue,
            citecolor=green]{hyperref}
  


\usepackage{multirow}

\usepackage{geometry}
\geometry{left=2.5cm,right=2.5cm,top=2.5cm,bottom=2.5cm}

\hfuzz=\maxdimen
\tolerance=10000
\hbadness=10000
\usepackage{xltxtra}
\usepackage{titlesec}
\titleformat{\section}[hang]{\Huge\bfseries}{第\,\thesection\,章}{1em}{}


%设置section的中文编号
% \usepackage[fntef]{ctexcap}
% \setromanfont[Mapping=tex-text]{Linux Libertine O}


% \setsansfont[Mapping=tex-text]{DejaVu Sans}
% \setmonofont[Mapping=tex-text]{DejaVu Sans Mono}

\title{Computer Organization}
\author{}
\date{}

\begin{document}
\maketitle

\newpage

\tableofcontents




\section{绪论}

\section{线性表}

顺序表

单链表

双链表

\section{栈和队列}


\section{树与二叉树}



\subsection{树的基本概念}


\subsubsection{树的定义}

树是N(N$\geq$0)个结点的有限集合,N=0时,称为空树。

非空树:

1)有且仅有一个称为根的结点。

2)N大于1时,可以分为m个互不相交的有限集合,每一个集合为子树。


\subsubsection{基本术语}

祖先结点、子孙结点、双亲结点、孩子结点、兄弟结点

结点的度:结点的子结点数。

树的度:树中最大的度。

分支结点(非终端结点):度大于0。

叶子结点(终端结点):度等于0。

层次:从树根开始,根节点为第一层,子结点为第二层。

深度:从根结点开始自顶向下逐层累加。

高度:从叶结点开始从低向上逐层累加。

树的高度(深度):树的最大层数。

有序树:树从左到右有次序的。

无序树:反之。

路径:树中两个结点之间所经过的结点序列构成。

路径长度:所经过路径上所经过边的个数。

森林:是m(m$\leq$0)棵互不相交的树的集合。

树的性质:

1)树中的结点数等于所有结点的度数加1

2)度为m的树中第i层上至多有$m^{i-1}$个结点($i\geq 1$)

(最多:第1层1个,第2层最多m个,第3层最多m^2个,第i层$m^{i-1}$个)

3)高为h的m叉树至多有($m^h$-1)/(m-1)个结点

(最多:第1层1个,第2层m+1个,第3层$m^2+m+1$个,第h层$m^{h-1}+...+m^2+m^1+m^0$个,即$(m^h-1)/(m-1)$)

4)具有n个结点的m叉树的最小高度为[$log_m$(n(m-1)+1]


\subsection{二叉树的概念}

\subsubsection{二叉树的定义及其主要特性}


1、二叉树定义

每个结点最多只有两棵子树。

有序树

二叉树和度为2有序树的树:

二叉树可以为空,度为2的树至少3个结点。

度为2的有序树,次序是相对的,度为1无须分左右。二叉树次序是固定的左右结点。

2、特殊的二叉树

满二叉树:高度为h,并且含有$2^h-1$个结点的二叉树。

完全二叉树:
h高度、n个结点的二叉树,结点和h高度的满二叉树对应,成为完全二叉树。

二叉排序树:
左子树小于根结点,右子树大于根结点

平衡二叉树:
树上任一结点的左子树和右子树深度之差不超过1。

3、二叉树性质

\subsubsection{二叉树的存储结构}

1、顺序存储结构

完全二叉树和满二叉树采用顺序存储比较合适。

2、链式存储结构

data、lchild、rchild

\subsection{二叉树遍历和线索二叉树}

先序

中序

后序

\subsection{树、森林}




\subsection{树与二叉树的应用}




\section{图}

\section{查找}

\section{排序}

\subsection{}









\end{document}
