\documentclass{ctexart}
\usepackage{verbatim}
\usepackage{amsmath}
\usepackage{bm}

%箭头等号等
\usepackage{extarrows}
%链接颜色,主要是目录链接
\usepackage[colorlinks,
            linkcolor=black,
            anchorcolor=blue,
            citecolor=green]{hyperref}


\hfuzz=\maxdimen
\tolerance=10000
\hbadness=10000

%多栏目录
\usepackage{multicol}
\AtBeginDocument{\addtocontents{toc}{\protect\begin{multicols}{2}}}
\AtEndDocument{\addtocontents{toc}{\protect\end{multicols}}}

\begin{document}
%\setlength{\parindent}{0pt}

%\begin{comment}
\tableofcontents
\newpage

\begin{gather*} 
%矩阵
\begin{matrix} 0 & 1  \\ 1 & 0 \end{matrix} \quad 
%括号矩阵
\begin{pmatrix}0 & -i \\ i & 0 \end{pmatrix}\\ 
%中括号矩阵
\begin{bmatrix}0 & -1 \\ 1 & 0 \end{bmatrix}\quad 
%大括号矩阵
\begin{Bmatrix}1 & 0  \\ 0 & -1\end{Bmatrix}\\
%行列式
\begin{vmatrix}a & b  \\ c & d \end{vmatrix}\quad 
%双竖线行列式
\begin{Vmatrix}i & 0  \\ 0 & -i\end{Vmatrix}

\begin{array}{cc} 0 & 1 \\ 1 & 0 \end{array} \quad 
\left(\begin{array}{cc} 0 & -i \\ i & 0 \end{array}\right) \\
\left[\begin{array}{cc} 0 & -1 \\ 1 & 0 \end{array}\right] \quad 
\left\{\begin{array}{cc} 1 & 0 \\ 0 & -1 \end{array}\right\} \\ 
\left|\begin{array}{cc} a & b \\ c & d \end{array}\right| \quad
\left\|\begin{array}{cc} i & 0 \\ 0 & -i \end{array}\right\|

------------------------------------------------------------------------

\[\left(\begin{array}{cccc} a_{11} & a_{12} & \cdots & a_{1n} \\ &a_{22} & \cdots &a_{2n} \\ & & \ddots & \vdots \\ \multicolumn{2}{c}{\raisebox{1.3ex}[0pt]{\Huge0}} & &a_{nn} \end{array}\right) \] \[\begin{pmatrix} a_{11} & a_{12} & \cdots & a_{1n} \\ &a_{22} & \cdots &a_{2n} \\ & & \ddots & \vdots \\ \multicolumn{2}{c}{\raisebox{1.3ex}[0pt]{\Huge0}} & &a_{nn} \end{pmatrix} \]


\[ \begin{array}{c@{\hspace{-5pt}}l} \left(\begin{array}{ccc|ccc} a&\cdots &a &b &\cdots&b\\ &\ddots &\vdots &\vdots &\adots\\ & &a& b \\\hline & & & c &\cdots &c\\ & & & \vdots& &\vdots\\ \multicolumn{3}{c|}{\raisebox{2ex}[0pt]{\Huge0}} & c & \cdots & c \end{array}\right) &\begin{array}{l}\left.\rule{0mm}{7mm}\right\}p\\ \\\left.\rule{0mm}{7mm}\right\}q \end{array}\\[-5pt] \begin{array}{cc}\underbrace{\rule{17mm}{0mm}}_m& \underbrace{\rule{17mm}{0mm}}_m\end{array}& \end{array} \]

\[ \begin{pmatrix} a_{11}&a_{12}&\ldots&a_{1n}\\ a_{21}&a_{22}&\ldots&a_{2n}\\ \hdotsfor{4}\\ a_{n1}&a_{n2}&\ldots&a_{nn}\\ \end{pmatrix} \] 和 \[\begin{pmatrix} a_{11}&a_{12}&\ldots&a_{1n}\\ a_{21}&a_{22}&\ldots&a_{2n}\\ \vdots &\vdots & &\vdots\\ a_{n1}&a_{n2}&\ldots&a_{nn}\\ \end{pmatrix}\]


\end{gather*}

\section{行列式}

\subsubection{定义}

行列式

逆序数

余子式

代数余子式

\section{矩阵}

\section{向量}
 
\subsection{向量组的线性相关}

\subsubsection{定义}



n维向量:数域P上n个数排成一个有序数组[$a_1,a_2,\dots ,a_3$]称为数域P上的n维向量。

线性表出:若$\beta$能表示成$k_1\alpha_1+k_2\alpha_2+\dots +k_m\alpha_m$,则称$\beta$能由$\alpha_1,\alpha_2,\dots,\alpha_m$线性表出。

线性相关:对m个n维向量$\alpha_1,\alpha_2,\dots,\alpha_m$,若存在不全为零的数$k_1,k_2,\dots,k_m$,使得$k_1\alpha_1+k_2\alpha_2+\dots +k_m\alpha_m=\bm{0}$成立,则称向量组$\alpha_1,\alpha_2,\dots,\alpha_m$线性相关。

线性无关:不存在不全为零的数$k_1,k_2,\dots,k_m$,使得$k_1\alpha_1+k_2\alpha_2+\dots +k_m\alpha_m=\bm{0}$。\\
或对于任意不全为零的数$k_1,k_2,\dots,k_m$,使得$k_1\alpha_1+k_2\alpha_2+\dots +k_m\alpha_m=\bm{0}$,均有$k_1\alpha_1+k_2\alpha_2+\dots +k_m\alpha_m \neq \bm{0}$

\subsection{极大线性无关组、秩}

\subsubsection{定义}
极大线性无关组:向量组$\alpha_{i1},\alpha_{i2},\dots,\alpha_{ir}(1\leq r \leq s)$是向量组

秩:极大无关组中向量的个数。(A中有一个r阶子式$D_r\neq 0$, 并且所有含$D_r$的r+1阶子式(如果存在的话)$D_{r+1}=0$。)

r(A+B)<=r(A)+r(B)

r(AB)<=min(r(A),r(B))


\section{线性方程组}



\section{特征值、特征向量、相似矩阵}



\section{二次型}




\end{document}