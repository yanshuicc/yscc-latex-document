\documentclass{ctexart}
\usepackage[colorlinks,
            linkcolor=black,
            anchorcolor=blue,
            citecolor=green]{hyperref}

\usepackage{geometry}
\geometry{left=2.5cm,right=2.5cm,top=2.5cm,bottom=2.5cm}

\hfuzz=\maxdimen
\tolerance=10000
\hbadness=10000

\begin{document}
\tableofcontents

\section{第1章\ 计算机网络体系结构}

\subsection{计算机网络概述}
\subsubsection{计算机网络的概念}
\subsubsection{计算机网络的组成}
\subsubsection{计算机网络的功能}
\subsubsection{计算机网络的分类}
\subsubsection{计算机网络的标准化工作及相关组织}
\subsubsection{计算机网络的性能指标}

\subsection{计算机网络体系结构与参考模型}
\subsubsection{计算机网络分层结构}
\subsubsection{计算机网络协议、借口、服务的概念}
\subsubsection{ISO/OSI参考模型和TCP/IP模型}

\section{第2章\ 物理层}
\subsection{通信基础}
\subsubsection{基本概念}
\subsubsection{奈奎斯特定理与香农定理}
\subsubsection{编码与调制}
\subsubsection{电路交换、报文交换与分组交换}
\subsubsection{数据包与虚电路}
\subsection{传输介质}
\subsubsection{双绞线、同轴电缆、光纤与无线传输介质}
\subsubsection{物理层接口的特性}
\subsection{物理层设备}
\subsubsection{中继器}
\subsubsection{集线器}

\section{第3章\ 数据链路层}
\subsection{数据链路层的功能}
\subsection{组帧}
\subsection{差错控制}
\subsection{流量控制与可靠传输机制}
\subsection{介质访问控制}
\subsection{局域网}
\subsection{广域网}
\subsection{数据链路层设备}

\section{第4章\ 网络层}
\subsection{网络层的功能}
\subsubsection{异构网络互联}
网络互联:指将两个以上的计算机网络,通过一定的方法,用一种或多种通信处理设备(即中间设备)相互连接起来,以构成更大的网络系统。
\\\\
中间设备:又称中间系统或中继系统。

根据中继系统所在层次,分为下列四中不同的中继系统。

物理层:中继器、集线器(Hub)

数据链路层:网桥、交换机

网络层:路由器

网络层以上:网关
\\\\

\subsubsection{路由与转发}
路由器功能:路由选择(确定那一条路径)、分组转发(当一个分组到达时所采取的动作)

路由选择:指按照复杂的分布式算法,根据从各相邻路由器所得到的关于整个网络拓扑的变化情况,动态的改变所选择的路由。

分组转发:指路由器根据转发表将用户的IP数据包从合适的端口转发出去。

\subsubsection{4.1.3\ 拥塞控制}


\subsection{路由算法}
\subsubsection{静态路由与动态路由}

\subsubsection{距离-向量路由算法}

\subsubsection{链路状态路由算法}

\subsubsection{层次路由}

\subsection{IPV4}
\subsection{IPV6}
\subsection{路由协议}
\subsection{IP组播}
\subsection{移动IP}
\subsection{网络层设备}

\section{第5章\ 传输层}
\subsection{传输层提供的服务}
\subsection{UDP协议}
\subsection{TCP协议}

\section{第6章\ 应用层}
\subsection{网络应用模型}
\subsection{DNS系统}
\subsection{文件传输协议FTP}
\subsection{电子邮件}
\subsection{万维网WWW}


\end{document}
