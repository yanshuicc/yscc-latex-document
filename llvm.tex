\documentclass{ctexart}
\usepackage{listings}
\usepackage{chngpage}
\usepackage{xcolor}

\begin{document}
\tableofcontents
\newpage
\lstset{
	language=C++,		
	numbers=left, 
	basicstyle=\fontsize{10pt}{5pt}\selectfont ,
	%stepnumber=2, 
	tabsize=2,                    	%tab两空格
	%numberstyle=\tiny,
	keywordstyle=\color{red!70},
	commentstyle=\color{blue!70}\fontsize{9pt}{5pt}\selectfont ,	
	frame=shadowbox, 
	%frame=leftline,	
	rulesepcolor=\color{red!20!green!20!blue!20},
	escapeinside=``,					%逃逸文字
	%xleftmargin=2em,			
	%xrightmargin=2em, 				
	aboveskip=1em,
	breaklines=true,               % 自动断行
	extendedchars=false,				%代码跨页
}

\section{Dwraf}

Dwraf(Debugging With Attributed Record Formats)字面直译过来是属性化记录格式调试,是调试信息的一种标准格式。下面介绍几个相关的文件及其中类的结构关系。\\
(默认“.”为llvm项目的源码根目录)\\

\fbox{
	\parbox{\textwidth}{
		./include/llvm/Support

		\begin{adjustwidth}{5em}{0em}
			Dwarf.h\\
		\end{adjustwidth}

		./include/llvm/CodeGen
		\begin{adjustwidth}{5em}{0em}
			DIE.h\\
		\end{adjustwidth}

		./lib/CodeGen/AsmPrinter
		\begin{adjustwidth}{5em}{0em}
		    AddressPoll.h\\
			DwarfAccelTable.h\\
			DwarfCompileUnit.h\\
			DwarfDebug.h\\
			DwarfException.h\\
			DwarfExpression.h\\
			DwarfFile.h\\
			DwarfStringPool.h\\
			DwarfUnit.h\\
			EHStreamer.h\\
		\end{adjustwidth}
	}
}

%\rule{80ex}{0.1ex}


\subsection{Dwarf.h}
Dwarf.h(./include/include/Support/Dwarf.h)中定义了相关的标识常量。具体的标准文档可以从http://www.dwarfstd.org.下载。\par

Dwarf.h文件中定义了大量的枚举常量,将调试程序信息的相关操作对应到一个常数(码),包括(括号内简单说明枚举声明的意思和部分枚举成员)

\fbox{
	\parbox{\textwidth}{
		enum:
		\begin{adjustwidth}{5em}{0em}
			Tag(uint16\_t标签:array\_type、class\_type...)\\
			Attribute(uint16\_t属性:sibling、location、name、ordering)\\
			Form(uint16\_t属性格式码:addr、block2、block4...)\\
			LocationAtom(操作码:addr、deref、dup、drop、div...)\\
			TypeKind(属值码:address、boolean、complex\_ float、float...)\\
			DecimalSignEncoding(十进制符号码) \\
			EndianityEncoding(结束符码)\\
			AccessAttribute(访问属性:public、protected、private)\\
			VisibilityAttribute(可访问属性)\\
			VirtualityAttribute(虚属性)\\
			SourceLanguage(代码语言:c89、c、ada83...)\\
			CaseSensitivity(大小写敏感)\\
			CallingConvention(调用约定:normal、program、nocall、lo\_user、hi\_user)\\
			InlineAttribute(内联属性)\\
			ArrayDimensionOrdering(数组维:row、col)\\
			DiscriminantList(判别表:lable、range)\\
			LineNumberOps(行号操作码:extended\_op、copy、advance\_pc、advance\_line )\\
			LineNumberExtendedOps(行号扩展操作:end\_sequence、set\_address、define\_file...)\\
			MacinfoRecordType(宏类型码:define、undef、start\_file...)\\
			CallFrameInfo(调用框架指令码:extended、nop、advance...)\\
			Constants(常量:no、yes、absptr、omit...)\\
			LocationListEntry()\\
			ApplePropertyAttributes(object-c属性:readonly、getter、assign...)\\
			AcceleratorTable()\\
			GDBIndexEntryKind()\\
			GDBIndexEntryLinkage()\\
		\end{adjustwidth}
	}
}




\subsection{DIE.h}
DIE(debug information entry)是调试信息入口的意思。是Dwarf存储程序中变量函数信息的类。
DIE.h(./include/llvm/CodeGen)的头文件中14个相关的类。后面几个类是基本相同的结构,都是从DIEValue中继承出来的。

\fbox{
	\parbox{\textwidth}{
		class
		\begin{adjustwidth}{5em}{0em}
			DIEAbbrevData [llvm]\\
			DIEAbbrev [llvm]\\
			DIE [llvm]\\
			DIEValue [llvm]\\
			DIEInteger [llvm]\\
			DIEExpr [llvm]\\
			DIELabel [llvm]\\
			DIEDelta [llvm]\\
			DIEString [llvm]\\
			DIEEntry [llvm]\\
			DIETypeSignature [llvm]\\
			DIELoc [llvm]\\
			DIEBlock [llvm]\\
			DIELocList [llvm]\\
		\end{adjustwidth}	
	}
}

这14个类的成员函数多是对成员变量的操作,而成员变量则是前面介绍的Dwarf.h中声明的枚举类型,下面列出这14个类中的成员变量,简单说明这些类之间的嵌套关系。

\subsubsection{DIEAbbrevData.class}
DIEAbbrevData是被DIEAbbrev中Data的SmallVector的类型。
\begin{lstlisting}[language=c++]
class DIEAbbrevData {
  /// Attribute - Dwarf attribute code.
  /// 属性码 - Dwarf.h文件中声明的枚举16位uint16\_t
  dwarf::Attribute Attribute;

  /// Form - Dwarf form code.
  /// 格式码  Dwarf.h文件中声明的枚举16位uint16\_t
  dwarf::Form Form;
  ...
}
\end{lstlisting}

\subsubsection{DIEAbbrev.class}
\begin{lstlisting}[language=c++]
class DIEAbbrev : public FoldingSetNode {
  /// Unique number for node.
  /// 唯一节点数字
  unsigned Number;

  /// Tag - Dwarf tag code.
  ///dwarf标签码
  dwarf::Tag Tag;

  /// Children - Whether or not this node has children.
  /// 是否有子节点
  // This cheats a bit in all of the uses since the values in the standard
  // are 0 and 1 for no children and children respectively.
  bool Children;

  /// Data - Raw data bytes for abbreviation.
  /// 申请了12个DIEAbbrevData空间的Data
  SmallVector<DIEAbbrevData, 12> Data;
  ...
 }
\end{lstlisting}

\subsubsection{DIE.class}
DIE是调试信息入口的主类,包括各种调试信息和子DIE,及父DIE指针。函数成员包括获取设置这些信息值,和添加Value和child。
\begin{lstlisting}[language=c++]
class DIE {
protected:
  /// Offset - Offset in debug info section.
  /// 偏移量 调试信息段偏移量
  unsigned Offset;

  /// Size - Size of instance + children.
  /// 自己+子DIE的总共大小
  unsigned Size;

  /// Abbrev - Buffer for constructing abbreviation.
  /// 缩写
  DIEAbbrev Abbrev;

  /// Children DIEs.
  /// 子DIE
  // This can't be a vector<DIE> because pointer validity is requirent for the
  // Parent pointer and DIEEntry.
  // It can't be a list<DIE> because some clients need pointer validity before
  // the object has been added to any child list
  // (eg: DwarfUnit::constructVariableDIE). These aren't insurmountable, but may
  // be more convoluted than beneficial.
  std::vector<std::unique_ptr<DIE>> Children;

  DIE *Parent;

  /// Attribute values.
  /// 属性值
  SmallVector<DIEValue *, 12> Values;
  ...
}
\end{lstlisting}

\subsubsection{DIEValue.class}
DIEValue是基本的值类,成员变量仅声明了类型,成员函数限于获取类型,虚函数声明了查值和类型的大小。
\begin{lstlisting}[language=c++]
class DIEValue {
...
protected:
  /// Ty - Type of data stored in the value.
  /// 类型
  Type Ty;
  ...
}
\end{lstlisting}


DIEValue中的Type是类中声明枚举。
\begin{lstlisting}[language=c++]
class DIEValue {
  enum Type {
    isInteger,
    isString,
    isExpr,
    isLabel,
    isDelta,
    isEntry,
    isTypeSignature,
    isBlock,
    isLoc,
    isLocList,
  };
  ...
}
\end{lstlisting}

继承自DIEValue的DIEInteger,多了个64位的Integer。函数成员多了BestForm(选择最好的int格式)、GetValue(返回Integer的数值)、classof(返回DIEValue中枚举的int)
\subsubsection{DIEInteger.class}
\begin{lstlisting}[language=c++]
class DIEInteger : public DIEValue {
  uint64_t Integer;
  ...
}
\end{lstlisting}

DIEExpr基本同DIEInteger,只是是针对的表达式的类。
\subsubsection{DIEExpr.class}
\begin{lstlisting}[language=c++]
class DIEExpr : public DIEValue {
  const MCExpr *Expr;
  ...
}
\end{lstlisting}

DIELabel同上,针对于label(MCSymbol ./include/llvm/MC)
\subsubsection{DIELabel.class}
\begin{lstlisting}[language=c++]
class DIELabel : public DIEValue {
  const MCSymbol *Label;
  ...
}
\end{lstlisting}

DIEDelta同上,针对Delta(增量,所以有两个label)
\subsubsection{DIEDelta.class}
\begin{lstlisting}[language=c++]
class DIEDelta : public DIEValue {
  const MCSymbol *LabelHi;
  const MCSymbol *LabelLo;
  ...
}
\end{lstlisting}

\subsubsection{DIEString.class}
\begin{lstlisting}[language=c++]
class DIEString : public DIEValue {
  const DIEValue *Access;
  StringRef Str;
  ...
}
\end{lstlisting}

\subsubsection{DIEEntry.class}
\begin{lstlisting}[language=c++]
class DIEEntry : public DIEValue {
  DIE &Entry;
  ...
}
\end{lstlisting}


\subsubsection{DIETypeSignature.class}
\begin{lstlisting}[language=c++]
class DIETypeSignature : public DIEValue {
  const DwarfTypeUnit &Unit;
  ...
}
\end{lstlisting}



\subsubsection{DIELoc.class}
\begin{lstlisting}[language=c++]
class DIELoc : public DIEValue, public DIE {
  ...
}
\end{lstlisting}


\subsubsection{DIEBlock.class}
\begin{lstlisting}[language=c++]
class DIEBlock : public DIEValue, public DIE {
  mutable unsigned Size; // Size in bytes excluding size header.
  ...
}
\end{lstlisting}


\subsubsection{DIELocList.class}
\begin{lstlisting}[language=c++]
class DIELocList : public DIEValue {
  // Index into the .debug_loc vector.
  size_t Index;
  ...
}
\end{lstlisting}

\newpage





\subsection{AddressPool.h}
AddressPool(地址池)成员变量为DenseMap<MCsymbol,地址入口>和一个是否使用的bool。
MCsymbol.h(./include/llvm/MC/)中定义了MCsymbol类。
地址入口则是类里面声明的结构体AddressPoolEntry。

\subsubsection{AddressPool.class}
\begin{lstlisting}[language=c++]
class AddressPool {
  struct AddressPoolEntry {
    unsigned Number;
    bool TLS;
    ...
  };
  DenseMap<const MCSymbol *, AddressPoolEntry> Pool;

  /// Record whether the AddressPool has been queried for an address index since
  /// the last "resetUsedFlag" call. Used to implement type unit fallback - a
  /// type that references addresses cannot be placed in a type unit when using
  /// fission.
  bool HasBeenUsed;
  ...
}
\end{lstlisting}

\subsection{DwarfAccelTable.h}

Dwarf加速表,是一种间接的哈希表优化,格式包括:
\\
HEADER      BUCKETS   HASHES      OFFSETS     DATA
\\
都是DwarfAccelTable的class成员。

\fbox{
	\parbox{\textwidth}{
		class
		\begin{adjustwidth}{5em}{0em}
		DwarfAccelTable [llvm]\\
		\end{adjustwidth}
	}
}

header有编号,版本号、hash函数的类型,buckets数目,hash的总数,data的结构,结构体长度。
\subsubsection{TableHeader.struct}
\begin{lstlisting}[language=c++]
struct TableHeader {
	/* 'HASH' magic value to allow endian detection*/
	uint32_ t magic; 
	/* Version number.*/
	uint16_ t version; 
	/* The hash function enumeration that was used.*/
	uint16_ t hash_function; 
	/* The number of buckets in this hash table.*/
	uint32_ t bucket_count; 
	/* The total number of unique hash values
	and hash data offsets in this table.  */
	uint32_ t hashes_count; 
	/* The bytes to skip to get to the hash
	indexes (buckets) for correct alignment.*/
	uint32_ t header_data_len; 
	...
\}
\end{lstlisting}

\subsubsection{Atom.struct}
\begin{lstlisting}[language=C++]
struct Atom {
	uint16_ t type; // enum AtomType
	uint16_ t form; // DWARF DW_FORM_defines
	...
\};
\end{lstlisting}

\subsubsection{TableHeaderData.struct}
\begin{lstlisting}[language=C++]
struct TableHeaderData {
	uint32_ t die_ offset_ base;
	SmallVector<Atom, 3> Atoms;
	...
}
\end{lstlisting}

\subsubsection{HashDataContents.struct}
\begin{lstlisting}[language=C++]
struct HashDataContents {
	const DIE *Die;   // Offsets
	char Flags; // Specific flags to output
	...
}
\end{lstlisting}

\subsubsection{DataArray.struct}
\begin{lstlisting}[language=C++]
struct DataArray {
	MCSymbol *StrSym;
	std::vector<HashDataContents *> Values;
	...
}
\end{lstlisting}

\subsubsection{HashData.struct}
\begin{lstlisting}[language=C++]
struct HashData {
	StringRef Str;
	uint32_ t HashValue;
	MCSymbol *Sym;
	DwarfAccelTable::DataArray &Data; // offsets
	...
}
\end{lstlisting}




\subsection{DwarfCompileUnit.h}

\fbox{
	\parbox{\textwidth}{
		class
		\begin{adjustwidth}{5em}{0em}
			DwarfCompileUnit [llvm]\\
		\end{adjustwidth}
	}
}


\subsubsection{DwarfCompileUnit.class}
\begin{lstlisting}[language=C++]
class DwarfCompileUnit : public DwarfUnit {
  /// The attribute index of DW_AT_stmt_list in the compile unit DIE, avoiding
  /// the need to search for it in applyStmtList.
  unsigned stmtListIndex;

  /// Skeleton unit associated with this unit.
  DwarfCompileUnit *Skeleton;

  /// A label at the start of the non-dwo section related to this unit.
  MCSymbol *SectionSym;

  /// The start of the unit within its section.
  MCSymbol *LabelBegin;

  /// GlobalNames - A map of globally visible named entities for this unit.
  StringMap<const DIE *> GlobalNames;

  /// GlobalTypes - A map of globally visible types for this unit.
  StringMap<const DIE *> GlobalTypes;

  // List of range lists for a given compile unit, separate from the ranges for
  // the CU itself.
  SmallVector<RangeSpanList, 1> CURangeLists;

  // List of ranges for a given compile unit.
  SmallVector<RangeSpan, 2> CURanges;

  // The base address of this unit, if any. Used for relative references in
  // ranges/locs.
  const MCSymbol *BaseAddress;
  ...
}
\end{lstlisting}

\subsection{DwarfDebug.h}

\fbox{
	\parbox{\textwidth}{
		class
		\begin{adjustwidth}{5em}{0em}
		SrcLineInfo [llvm]\\
		DbgVariable [llvm]\\
		DwarfDebug [llvm]\\
		\end{adjustwidth}
		}
}


\subsubsection{SrcLineInfo.class}
\begin{lstlisting}[language=C++]
/// \brief This class is used to record source line correspondence.
class SrcLineInfo {
  unsigned Line;     // Source line number.
  unsigned Column;   // Source column.
  unsigned SourceID; // Source ID number.
  MCSymbol *Label;   // Label in code ID number.
  ...
}
\end{lstlisting}


\subsubsection{DbgVariable.class}
\begin{lstlisting}[language=C++]
/// \brief This class is used to track local variable information.
class DbgVariable {
  DIVariable Var;             // Variable Descriptor.
  DIExpression Expr;          // Complex address location expression.
  DIE *TheDIE;                // Variable DIE.
  unsigned DotDebugLocOffset; // Offset in DotDebugLocEntries.
  const MachineInstr *MInsn;  // DBG_VALUE instruction of the variable.
  int FrameIndex;
  DwarfDebug *DD;
  ...
}
\end{lstlisting}


\subsubsection{DwarfDebug.class}
\begin{lstlisting}[language=C++]
/// \brief Collects and handles dwarf debug information.
class DwarfDebug : public AsmPrinterHandler {
  // Target of Dwarf emission.
  AsmPrinter *Asm;

  // Collected machine module information.
  MachineModuleInfo *MMI;

  // All DIEValues are allocated through this allocator.
  BumpPtrAllocator DIEValueAllocator;

  // Maps MDNode with its corresponding DwarfCompileUnit.
  MapVector<const MDNode *, DwarfCompileUnit *> CUMap;

  // Maps subprogram MDNode with its corresponding DwarfCompileUnit.
  MapVector<const MDNode *, DwarfCompileUnit *> SPMap;

  // Maps a CU DIE with its corresponding DwarfCompileUnit.
  DenseMap<const DIE *, DwarfCompileUnit *> CUDieMap;

  // List of all labels used in aranges generation.
  std::vector<SymbolCU> ArangeLabels;

  // Size of each symbol emitted (for those symbols that have a specific size).
  DenseMap<const MCSymbol *, uint64_t> SymSize;

  // Provides a unique id per text section.
  typedef DenseMap<const MCSection *, SmallVector<SymbolCU, 8> > SectionMapType;
  SectionMapType SectionMap;

  LexicalScopes LScopes;

  // Collection of abstract variables.
  DenseMap<const MDNode *, std::unique_ptr<DbgVariable>> AbstractVariables;
  SmallVector<std::unique_ptr<DbgVariable>, 64> ConcreteVariables;

  // Collection of DebugLocEntry. Stored in a linked list so that DIELocLists
  // can refer to them in spite of insertions into this list.
  SmallVector<DebugLocList, 4> DotDebugLocEntries;

  // This is a collection of subprogram MDNodes that are processed to
  // create DIEs.
  SmallPtrSet<const MDNode *, 16> ProcessedSPNodes;

  // Maps instruction with label emitted before instruction.
  DenseMap<const MachineInstr *, MCSymbol *> LabelsBeforeInsn;

  // Maps instruction with label emitted after instruction.
  DenseMap<const MachineInstr *, MCSymbol *> LabelsAfterInsn;

  // History of DBG_VALUE and clobber instructions for each user variable.
  // Variables are listed in order of appearance.
  DbgValueHistoryMap DbgValues;

  // Previous instruction's location information. This is used to determine
  // label location to indicate scope boundries in dwarf debug info.
  DebugLoc PrevInstLoc;
  MCSymbol *PrevLabel;

  // This location indicates end of function prologue and beginning of function
  // body.
  DebugLoc PrologEndLoc;

  // If nonnull, stores the current machine function we're processing.
  const MachineFunction *CurFn;

  // If nonnull, stores the current machine instruction we're processing.
  const MachineInstr *CurMI;

  // If nonnull, stores the CU in which the previous subprogram was contained.
  const DwarfCompileUnit *PrevCU;

  // Section Symbols: these are assembler temporary labels that are emitted at
  // the beginning of each supported dwarf section.  These are used to form
  // section offsets and are created by EmitSectionLabels.
  MCSymbol *DwarfInfoSectionSym, *DwarfAbbrevSectionSym;
  MCSymbol *DwarfStrSectionSym, *TextSectionSym, *DwarfDebugRangeSectionSym;
  MCSymbol *DwarfDebugLocSectionSym, *DwarfLineSectionSym, *DwarfAddrSectionSym;
  MCSymbol *FunctionBeginSym, *FunctionEndSym;
  MCSymbol *DwarfInfoDWOSectionSym, *DwarfAbbrevDWOSectionSym;
  MCSymbol *DwarfTypesDWOSectionSym;
  MCSymbol *DwarfStrDWOSectionSym;
  MCSymbol *DwarfGnuPubNamesSectionSym, *DwarfGnuPubTypesSectionSym;

  // As an optimization, there is no need to emit an entry in the directory
  // table for the same directory as DW_AT_comp_dir.
  StringRef CompilationDir;

  // Counter for assigning globally unique IDs for ranges.
  unsigned GlobalRangeCount;

  // Holder for the file specific debug information.
  DwarfFile InfoHolder;

  // Holders for the various debug information flags that we might need to
  // have exposed. See accessor functions below for description.

  // Holder for imported entities.
  typedef SmallVector<std::pair<const MDNode *, const MDNode *>, 32>
  ImportedEntityMap;
  ImportedEntityMap ScopesWithImportedEntities;

  // Map from MDNodes for user-defined types to the type units that describe
  // them.
  DenseMap<const MDNode *, const DwarfTypeUnit *> DwarfTypeUnits;

  SmallVector<std::pair<std::unique_ptr<DwarfTypeUnit>, DICompositeType>, 1> TypeUnitsUnderConstruction;

  // Whether to emit the pubnames/pubtypes sections.
  bool HasDwarfPubSections;

  // Whether or not to use AT_ranges for compilation units.
  bool HasCURanges;

  // Whether we emitted a function into a section other than the default
  // text.
  bool UsedNonDefaultText;

  // Version of dwarf we're emitting.
  unsigned DwarfVersion;

  // Maps from a type identifier to the actual MDNode.
  DITypeIdentifierMap TypeIdentifierMap;

  // DWARF5 Experimental Options
  bool HasDwarfAccelTables;
  bool HasSplitDwarf;

  // Separated Dwarf Variables
  // In general these will all be for bits that are left in the
  // original object file, rather than things that are meant
  // to be in the .dwo sections.

  // Holder for the skeleton information.
  DwarfFile SkeletonHolder;

  /// Store file names for type units under fission in a line table header that
  /// will be emitted into debug_line.dwo.
  // FIXME: replace this with a map from comp_dir to table so that we can emit
  // multiple tables during LTO each of which uses directory 0, referencing the
  // comp_dir of all the type units that use it.
  MCDwarfDwoLineTable SplitTypeUnitFileTable;

  // True iff there are multiple CUs in this module.
  bool SingleCU;
  bool IsDarwin;

  AddressPool AddrPool;

  DwarfAccelTable AccelNames;
  DwarfAccelTable AccelObjC;
  DwarfAccelTable AccelNamespace;
  DwarfAccelTable AccelTypes;

  DenseMap<const Function *, DISubprogram> FunctionDIs;
  ...
}
\end{lstlisting}


\subsection{DwarfException.h}

包括ARMException和DwarfCFEExcetion的类定义,这两个.cpp文件(./lib/CodeGen/AsmPrinter)没有对应的.h文件,就是因为定义是放在了DwarfException.h中。这两个类都是继承自EHStreamer的。

\fbox{
	\parbox{\textwidth}{
		class
		\begin{adjustwidth}{5em}{0em}
			DwarfCFIException [llvm]\\
			ARMException [llvm]\\
		\end{adjustwidth}
	}
}

\subsubsection{DwarfCFIException.class}
\begin{lstlisting}[language=C++]
class DwarfCFIException : public EHStreamer {
  /// shouldEmitPersonality - Per-function flag to indicate if .cfi_personality
  /// should be emitted.
  bool shouldEmitPersonality;

  /// shouldEmitLSDA - Per-function flag to indicate if .cfi_lsda
  /// should be emitted.
  bool shouldEmitLSDA;

  /// shouldEmitMoves - Per-function flag to indicate if frame moves info
  /// should be emitted.
  bool shouldEmitMoves;

  AsmPrinter::CFIMoveType moveTypeModule;
  ...
}
\end{lstlisting}
  
  
\subsubsection{ARMException.class}
\begin{lstlisting}[language=C++]
 class ARMException : public EHStreamer {
  void emitTypeInfos(unsigned TTypeEncoding) override;
  ARMTargetStreamer &getTargetStreamer();

  /// shouldEmitCFI - Per-function flag to indicate if frame CFI info
  /// should be emitted.
  bool shouldEmitCFI;
  ...
}
\end{lstlisting}


\subsection{DwarfExpression.h}

\subsubsection{DwarfExpression.class}
\begin{lstlisting}[language=C++]
/// Base class containing the logic for constructing DWARF expressions
/// independently of whether they are emitted into a DIE or into a .debug_loc
/// entry.
class DwarfExpression {
protected:
  const AsmPrinter &AP;
  // Various convenience accessors that extract things out of AsmPrinter.
  const TargetRegisterInfo *getTRI() const;
  unsigned getDwarfVersion() const;
  ...
}
\end{lstlisting}


\subsection{DwarfFile.h}


\subsubsection{DwarfFile.class}
\begin{lstlisting}[language=C++]
class DwarfFile {
  // Target of Dwarf emission, used for sizing of abbreviations.
  AsmPrinter *Asm;

  DwarfDebug &DD;

  // Used to uniquely define abbreviations.
  FoldingSet<DIEAbbrev> AbbreviationsSet;

  // A list of all the unique abbreviations in use.
  std::vector<DIEAbbrev *> Abbreviations;

  // A pointer to all units in the section.
  SmallVector<std::unique_ptr<DwarfUnit>, 1> CUs;

  DwarfStringPool StrPool;

  // Collection of dbg variables of a scope.
  DenseMap<LexicalScope *, SmallVector<DbgVariable *, 8>> ScopeVariables;

  // Collection of abstract subprogram DIEs.
  DenseMap<const MDNode *, DIE *> AbstractSPDies;

  /// Maps MDNodes for type system with the corresponding DIEs. These DIEs can
  /// be shared across CUs, that is why we keep the map here instead
  /// of in DwarfCompileUnit.
  DenseMap<const MDNode *, DIE *> MDTypeNodeToDieMap;


  ...
}
\end{lstlisting}



\subsection{DwarfStringPool.h}

\subsubsection{DwarfStringPool.class}
\begin{lstlisting}[language=C++]
// Collection of strings for this unit and assorted symbols.
// A String->Symbol mapping of strings used by indirect
// references.
class DwarfStringPool {
  StringMap<std::pair<MCSymbol *, unsigned>, BumpPtrAllocator &> Pool;
  StringRef Prefix;
  ...
}
\end{lstlisting}

\subsection{DwarfUnit.h}
Dwarf单元

\fbox{
	\parbox{\textwidth}{
		class
		\begin{adjustwidth}{5em}{0em}
  class
    RangeSpan [llvm]\\
    RangeSpanList [llvm]\\
    DwarfUnit [llvm]\\
    DwarfTypeUnit [llvm]\\
		\end{adjustwidth}
	}
}

\subsubsection{RangeSpan.class}
\begin{lstlisting}[language=C++]
// Data structure to hold a range for range lists.保持MCSymbol范围的数据结构
class RangeSpan {
public:
  RangeSpan(MCSymbol *S, MCSymbol *E) : Start(S), End(E) {}
  const MCSymbol *getStart() const { return Start; }
  const MCSymbol *getEnd() const { return End; }
  void setEnd(const MCSymbol *E) { End = E; }

private:
  const MCSymbol *Start, *End;
};
\end{lstlisting}

\subsubsection{RangSpanList.class}
\begin{lstlisting}[language=C++]
class RangeSpanList {
private:
  // Index for locating within the debug_range section this particular span.索引定位
  MCSymbol *RangeSym;
  // List of ranges.两个RangeSpan
  SmallVector<RangeSpan, 2> Ranges;
  ...
}
\end{lstlisting}

\subsubsection{DwarfUnit.class}
\begin{lstlisting}[language=C++]
//===----------------------------------------------------------------------===//
/// Unit - This dwarf writer support class manages information associated
/// with a source file.
class DwarfUnit {
protected:
  /// UniqueID - a numeric ID unique among all CUs in the module
  unsigned UniqueID;

  /// Node - MDNode for the compile unit.
  DICompileUnit CUNode;

  /// Unit debug information entry.单元DIE
  DIE UnitDie;

  /// Offset of the UnitDie from beginning of debug info section.
  unsigned DebugInfoOffset;

  /// Asm - Target of Dwarf emission.
  AsmPrinter *Asm;

  // Holders for some common dwarf information.
  DwarfDebug *DD;
  DwarfFile *DU;

  /// IndexTyDie - An anonymous type for index type.  Owned by UnitDie.
  DIE *IndexTyDie;

  /// MDNodeToDieMap - Tracks the mapping of unit level debug information
  /// variables to debug information entries.
  DenseMap<const MDNode *, DIE *> MDNodeToDieMap;

  /// MDNodeToDIEEntryMap - Tracks the mapping of unit level debug information
  /// descriptors to debug information entries using a DIEEntry proxy.
  DenseMap<const MDNode *, DIEEntry *> MDNodeToDIEEntryMap;

  /// DIEBlocks - A list of all the DIEBlocks in use.
  std::vector<DIEBlock *> DIEBlocks;
  
  /// DIELocs - A list of all the DIELocs in use.
  std::vector<DIELoc *> DIELocs;

  /// ContainingTypeMap - This map is used to keep track of subprogram DIEs that
  /// need DW_AT_containing_type attribute. This attribute points to a DIE that
  /// corresponds to the MDNode mapped with the subprogram DIE.
  DenseMap<DIE *, const MDNode *> ContainingTypeMap;

  // DIEValueAllocator - All DIEValues are allocated through this allocator.
  BumpPtrAllocator DIEValueAllocator;

  // DIEIntegerOne - A preallocated DIEValue because 1 is used frequently.
  DIEInteger *DIEIntegerOne;

  /// The section this unit will be emitted in.单元发出的section
  const MCSection *Section;
  ...
}
\end{lstlisting}

\subsubsection{DwarfTypeUnit.class}
\begin{lstlisting}[language=C++]
class DwarfTypeUnit : public DwarfUnit {
  uint64\_t TypeSignature;
  const DIE *Ty;
  DwarfCompileUnit &CU;
  MCDwarfDwoLineTable *SplitLineTable;
  ...
}
\end{lstlisting}



\subsection{EHStreamer.h}
\subsubsection{EHStreamer.class}
\begin{lstlisting}[language=C++]
/// Emits exception handling directives.发出异常处理指令
class EHStreamer : public AsmPrinterHandler {
protected:
  /// Target of directive emission.
  AsmPrinter *Asm;

  /// Collected machine module information.
  MachineModuleInfo *MMI;
  ...
}
\end{lstlisting}

\subsubsection{PadRange.struct}
\begin{lstlisting}[language=C++]
  /// Structure holding a try-range and the associated landing pad.
  struct PadRange {
    // The index of the landing pad.
    unsigned PadIndex;
    // The index of the begin and end labels in the landing pad's label lists.
    unsigned RangeIndex;
  };
\end{lstlisting}


\subsubsection{CallSiteEntry.struct}
\begin{lstlisting}[language=C++]
 /// Structure describing an entry in the call-site table.
  struct CallSiteEntry {
    // The 'try-range' is BeginLabel .. EndLabel.
    MCSymbol *BeginLabel; // zero indicates the start of the function.
    MCSymbol *EndLabel;   // zero indicates the end of the function.

    // The landing pad starts at PadLabel.
    MCSymbol *PadLabel;   // zero indicates that there is no landing pad.
    unsigned Action;
  };
\end{lstlisting}


\section{MC}
\par
\fbox{
	\parbox{\textwidth}{
		\begin{adjustwidth}{5em}{0em}
   		MCSection.h\\
    		MCSymbol.h\\
    		MCValue.h\\
    		MCLabel.h\\
    		MCExpr.h\\
		\end{adjustwidth}
	}
}
\subsection{MCSection.h}
\subsection{MCSymbol.h}
\subsection{MCValue.h}
\subsection{MCLabel.h}
\subsection{MCExpr.h}



MCSection.h(./include/llvm/MC)


\section{IR}
IR(Intermediate representation)中间表示(./include/llvm/IR)包含较多文件,挑选部分进行介绍。
\par
\fbox{
	\parbox{\textwidth}{
		\begin{adjustwidth}{5em}{0em}
    		LLVMContext.h\\
   		Module.h\\
   		Function.h\\
   		Type.h\\
   		DerivedTypes.h\\
   		AsmWriter.cpp\\
    		ModuleSlotTracker.h\\
    		IntrinsicInst.h\\
    		Verifier.h\\
		\end{adjustwidth}
	}
}
\par

\subsection{LLVMContext.h}
llvm使用的线程环境


\subsection{Module.h}
\fbox{
	\parbox{\textwidth}{
 	class
		\begin{adjustwidth}{5em}{0em}
    		Module [llvm]
		\end{adjustwidth}
	}
}
	
	
Module.h{./include/llvm/IR}	
Module接口存储LLVM信息,

typedef定义了几个ADT list的声明,globallist、fucntionlist、aliaslist、mdlist、SymTab,
还有每个list的迭代iterator。
\begin{lstlisting}[language=C++]
class Module {
  /// The type for the list of global variables.
  typedef iplist<GlobalVariable> GlobalListType;
  /// The type for the list of functions.
  typedef iplist<Function> FunctionListType;
  /// The type for the list of aliases.
  typedef iplist<GlobalAlias> AliasListType;
  /// The type for the list of named metadata.
  typedef ilist<NamedMDNode> NamedMDListType;
  /// The type of the comdat "symbol" table.
  typedef StringMap<Comdat> ComdatSymTabType;
  
    /// The Global Variable iterator.
  typedef GlobalListType::iterator                      global_iterator;
  /// The Global Variable constant iterator.
  typedef GlobalListType::const_iterator          const_global_iterator;

  /// The Function iterators.
  typedef FunctionListType::iterator                           iterator;
  /// The Function constant iterator
  typedef FunctionListType::const_iterator               const_iterator;

  /// The Function reverse iterator.
  typedef FunctionListType::reverse_iterator             reverse_iterator;
  /// The Function constant reverse iterator.
  typedef FunctionListType::const_reverse_iterator const_reverse_iterator;

  /// The Global Alias iterators.
  typedef AliasListType::iterator                        alias_iterator;
  /// The Global Alias constant iterator
  typedef AliasListType::const_iterator            const_alias_iterator;

  /// The named metadata iterators.
  typedef NamedMDListType::iterator             named_metadata_iterator;
  /// The named metadata constant iterators.
  typedef NamedMDListType::const_iterator const_named_metadata_iterator;
  ...
}
\end{lstlisting}


\begin{lstlisting}[language=C++]
class Module {
  ...
  LLVMContext &Context;           ///< The LLVMContext from which types and
                                  ///< constants are allocated.
  GlobalListType GlobalList;      ///< The Global Variables in the module
  FunctionListType FunctionList;  ///< The Functions in the module
  AliasListType AliasList;        ///< The Aliases in the module
  NamedMDListType NamedMDList;    ///< The named metadata in the module
  std::string GlobalScopeAsm;     ///< Inline Asm at global scope.
  ValueSymbolTable *ValSymTab;    ///< Symbol table for values
  ComdatSymTabType ComdatSymTab;  ///< Symbol table for COMDATs
  std::unique_ptr<GVMaterializer>
  Materializer;                   ///< Used to materialize GlobalValues
  std::string ModuleID;           ///< Human readable identifier for the module标识符
  std::string TargetTriple;       ///< Platform target triple Module compiled on
                                  ///< Format: (arch)(sub)-(vendor)-(sys0-(abi)
  void *NamedMDSymTab;            ///< NamedMDNode names.

  // We need to keep the string because the C API expects us to own the string
  // representation.
  // Since we have it, we also use an empty string to represent a module without
  // a DataLayout. If it has a DataLayout, these variables are in sync and the
  // string is just a cache of getDataLayout()->getStringRepresentation().
  std::string DataLayoutStr;
  DataLayout DL;

  friend class Constant;  
  ...
}
\end{lstlisting}

\subsection{Function.h}

Function.h(./include/llvm/IR)
typedef了Argument和BasicBlock的list及迭代器。
\begin{lstlisting}[language=C++]
class Function : public GlobalObject, public ilist_node<Function> {
public:
  typedef iplist<Argument> ArgumentListType;
  typedef iplist<BasicBlock> BasicBlockListType;

  // BasicBlock iterators...
  typedef BasicBlockListType::iterator iterator;
  typedef BasicBlockListType::const_iterator const_iterator;

  typedef ArgumentListType::iterator arg_iterator;
  typedef ArgumentListType::const_iterator const_arg_iterator;
  ...
}
\end{lstlisting}

类主要组成还是BasicBlock的list和Argument的list
ValueSymbolTable和AttributeSet分别是这两个类的.h文件中定义的类(./include/llvm/IR)。
\begin{lstlisting}[language=C++]
class Function : public GlobalObject, public ilist_node<Function> {
  ...
private:
  // Important things that make up a function!
  BasicBlockListType  BasicBlocks;        ///< The basic blocks
  mutable ArgumentListType ArgumentList;  ///< The formal arguments
  ValueSymbolTable *SymTab;               ///< Symbol table of args/instructions
  AttributeSet AttributeSets;             ///< Parameter attributes
  ...
}
\end{lstlisting}
  
\subsection{Type.h}
Type的基类,成员变量是线程环境和一个32位值低8位为TypeID,高24位为SubclassData
\begin{lstlisting}[language=C++]
class Type {
  ...
  private:
  /// Context - This refers to the LLVMContext in which this type was uniqued.
  LLVMContext &Context;

  // Due to Ubuntu GCC bug 910363:
  // https://bugs.launchpad.net/ubuntu/+source/gcc-4.5/+bug/910363
  // Bitpack ID and SubclassData manually.
  // Note: TypeID : low 8 bit; SubclassData : high 24 bit.
  uint32_t IDAndSubclassData;
  ...
}
\end{lstlisting}	
  

\subsection{DerivedTypes.h}
DerivedTypes衍生类型(./include/llvm/IR)中包含较多的Type,都是从Type.h中的Type基类继承过来的。
具体成员见下。
\par
\fbox{
	\parbox{\textwidth}{
		class
		\begin{adjustwidth}{5em}{0em}
    		IntegerType [llvm]\\
    		FunctionType [llvm]\\
    		CompositeType [llvm]\\
    		StructType [llvm]\\
    		SequentialType [llvm]\\
    		ArrayType [llvm]\\
    		VectorType [llvm]\\
    		PointerType [llvm]\\
		\end{adjustwidth}	
	}
}


\subsubsection{IntegerType.h}
\subsubsection{FunctionType.h}
\subsubsection{CompositeType.h}
\subsubsection{StructType.h}
\subsubsection{SequentialType.h}
\subsubsection{ArrayType.h}
\subsubsection{VectorType.h}
\subsubsection{PointerType.h}

\subsection{AsmWriter.cpp}
\par
\fbox{
	\parbox{\textwidth}{
	class
		\begin{adjustwidth}{5em}{0em}
    		TypePrinting [__anon2]
    		SlotTracker [llvm]
    		AssemblyWriter [__anon4]
		\end{adjustwidth}	
	}
}

\subsubsection{SlotTracker}
\begin{lstlisting}[language=C++]
//===----------------------------------------------------------------------===//
// SlotTracker Class: Enumerate slot numbers for unnamed values
//===----------------------------------------------------------------------===//
/// This class provides computation of slot numbers for LLVM Assembly writing.
/// 该类提供了汇编的slot number的相关操作。
class SlotTracker {
public:
  /// ValueMap - A mapping of Values to slot numbers.
  typedef DenseMap<const Value*, unsigned> ValueMap;

private:
  /// TheModule - The module for which we are holding slot numbers.
  const Module* TheModule;

  /// TheFunction - The function for which we are holding slot numbers.
  const Function* TheFunction;
  bool FunctionProcessed;
  bool ShouldInitializeAllMetadata;

  /// mMap - The slot map for the module level data.
  ValueMap mMap;
  unsigned mNext;

  /// fMap - The slot map for the function level data.
  ValueMap fMap;
  unsigned fNext;

  /// mdnMap - Map for MDNodes.
  DenseMap<const MDNode*, unsigned> mdnMap;
  unsigned mdnNext;

  /// asMap - The slot map for attribute sets.
  DenseMap<AttributeSet, unsigned> asMap;
  unsigned asNext;
  ...
}
\end{lstlisting}

\subsection{ModuleSlotTracker.h}
ModuleSlotTracker成员变量比较简单,unique_ptr的SlotTracker,和Modue、FUnction、SlotTracker三个指针。
\begin{lstlisting}[language=C++]
class ModuleSlotTracker {
  /// Storage for a slot tracker.
  std::unique_ptr<SlotTracker> MachineStorage;

  const Module *M = nullptr;
  const Function *F = nullptr;
  SlotTracker *Machine = nullptr;
  ...
}

\subsection{IntrinsicInst.h}



\subsection{Verifier.h}





\section{example}

\subsection{ModuleMaker.h}
\begin{lstlisting}[language=C++]
//===- examples/ModuleMaker/ModuleMaker.cpp - Example project ---*- C++ -*-===//
//
//                     The LLVM Compiler Infrastructure
//
// This file is distributed under the University of Illinois Open Source
// License. See LICENSE.TXT for details.
//
//===-------------------------------------===//
//
// This programs is a simple example that creates an LLVM module "from scratch",
// emitting it as a bitcode file to standard out.  This is just to show how
// LLVM projects work and to demonstrate some of the LLVM APIs.
//
//===-------------------------------------===//

#include "llvm/Bitcode/ReaderWriter.h"
#include "llvm/IR/Constants.h"
#include "llvm/IR/DerivedTypes.h"
#include "llvm/IR/Instructions.h"
#include "llvm/IR/LLVMContext.h"
#include "llvm/IR/Module.h"
#include "llvm/Support/raw_ostream.h"
using namespace llvm;

int main() {
  LLVMContext Context;	//声明线程环境

  // Create the "module" or "program" or "translation unit" to hold the
  // function 第一个参数"test"是ModuleID(string)
  Module *M = new Module("test", Context);

  // Create the main function: first create the type 'int ()'无参创建FunctionType,第一个参数为结果类型(Type× Result),第二个参数为是否变量参数(bool isVarArg)
  FunctionType *FT =
    FunctionType::get(Type::getInt32Ty(Context), /*not vararg*/false);

  // By passing a module as the last parameter to the Function constructor,
  // it automatically gets appended to the Module.函数的构造函数,通过最后一个参数M,将函数添加到模块里面。
  Function *F = Function::Create(FT, Function::ExternalLinkage, "main", M);

  // Add a basic block to the function... again, it automatically inserts
  // because of the last argument.创建基本块的构造函数,通过最后一个参数F,自动添加到Function中
  BasicBlock *BB = BasicBlock::Create(Context, "EntryBlock", F);

  // Get pointers to the constant integers...获取常数指针
  Value *Two = ConstantInt::get(Type::getInt32Ty(Context), 2);
  Value *Three = ConstantInt::get(Type::getInt32Ty(Context), 3);

  // Create the add instruction... does not insert...
  Instruction *Add = BinaryOperator::Create(Instruction::Add, Two, Three,
                                            "addresult");

  // explicitly insert it into the basic block...
  BB->getInstList().push_back(Add);

  // Create the return instruction and add it to the basic block
  BB->getInstList().push_back(ReturnInst::Create(Context, Add));

  // Output the bitcode file to stdout
  WriteBitcodeToFile(M, outs());

  // Delete the module and all of its contents.
  delete M;
  return 0;
}
\end{lstlisting}


\section{ourjob}
\subsection{llvm-irdwarf.h}
\begin{lstlisting}[language=C++]
// LLVM
#include "llvm/IR/IntrinsicInst.h"
#include "llvm/IR/LLVMContext.h"
#include "llvm/IR/Module.h"
#include "llvm/IR/ModuleSlotTracker.h"
#include "llvm/IR/Verifier.h"
#include "llvm/IRReader/IRReader.h"
#include "llvm/Support/CommandLine.h"
#include "llvm/Support/Debug.h"
#include "llvm/Support/Format.h"
#include "llvm/Support/FormattedStream.h"
#include "llvm/Support/ManagedStatic.h"
#include "llvm/Support/MemoryBuffer.h"
#include "llvm/Support/PrettyStackTrace.h"
#include "llvm/Support/Signals.h"
#include "llvm/Support/SourceMgr.h"
#include "llvm/Support/raw_ostream.h"
// STL
#include <string>

static llvm::cl::opt<std::string>
InputFilename(llvm::cl::Positional, llvm::cl::desc("Bitcode file"),
              llvm::cl::Required);

struct MDTracker {
    // map for MDNodes
    llvm::DenseMap<const llvm::MDNode *, unsigned> mdnMap;
    unsigned mdnNext;

    explicit MDTracker() : mdnMap(), mdnNext(0) {}

    void CreateMetadataSlot(const llvm::MDNode *N) {
        unsigned DestSlot = mdnNext;
        if (!mdnMap.insert(std::make_pair(N, DestSlot)).second)
            return;
        ++mdnNext;

        // Recursively add any MDNodes referenced by operands.
        for (unsigned i = 0, e = N->getNumOperands(); i != e; ++i)
            if (const llvm::MDNode *Op = llvm::dyn_cast_or_null<llvm::MDNode>(N->getOperand(i)))
                CreateMetadataSlot(Op);
    }

    void processInstructionMDNode(const llvm::Instruction &I) {
      // Process metadata used directly by intrinsics.
      if (const llvm::CallInst *CI = llvm::dyn_cast<llvm::CallInst>(&I))
        if (llvm::Function *F = CI->getCalledFunction())
          if (F->isIntrinsic())
            for (auto &Op : I.operands())
              if (auto *V = llvm::dyn_cast_or_null<llvm::MetadataAsValue>(Op))
                if (llvm::MDNode *N = llvm::dyn_cast<llvm::MDNode>(V->getMetadata()))
                  CreateMetadataSlot(N);

      // Process metadata attached to this instruction.
      llvm::SmallVector<std::pair<unsigned, llvm::MDNode *>, 4> MDs;
      I.getAllMetadata(MDs);
      for (auto &MD : MDs)
        CreateMetadataSlot(MD.second);
    }

    void processFunctionMDNode(const llvm::Function &F) {
        llvm::SmallVector<std::pair<unsigned, llvm::MDNode *>, 4> MDs;
        for (auto &BB : F) {
            F.getAllMetadata(MDs);
            for (auto &MD : MDs)
                CreateMetadataSlot(MD.second);

            for (auto &I : BB)
                processInstructionMDNode(I);
        }
    }

    void processModuleMDNode(const llvm::Module &M) {
      // Add metadata used by named metadata.
      for (const llvm::NamedMDNode &NMD : M.named_metadata()) {
        for (unsigned i = 0, e = NMD.getNumOperands(); i != e; ++i)
          CreateMetadataSlot(NMD.getOperand(i));
      }

      for (const llvm::Function &F : M)
        processFunctionMDNode(F);
    }

    void print(llvm::raw_ostream &ROS) {
        llvm::SmallVector<const llvm::MDNode *, 16> Nodes;
        Nodes.resize(mdnMap.size());
        for (auto &mdPair : mdnMap)
            Nodes[mdPair.second] = llvm::cast<llvm::MDNode>(mdPair.first);

        for (unsigned i = 0, e = Nodes.size(); i != e; ++i)
            print(ROS, i, Nodes[i]);
    }

    void print(llvm::raw_ostream &ROS, unsigned Slot, const llvm::MDNode *mdNode) {
        ROS << '!' << Slot << ": ";
        mdNode->print(ROS);
        ROS << "\n";
    }
};

static int DumpInput(llvm::StringRef ProgName, llvm::StringRef Filename, llvm::LLVMContext &Context) {
    llvm::SMDiagnostic Err;
    std::unique_ptr<llvm::Module> M = llvm::parseIRFile(Filename, Err, Context);
    if (!M) {
        Err.print(ProgName.data(), llvm::errs());
        return 1;
    }

    // Verify module immediately to catch problems before doInitialization() is
    // called on any passes.
    if (llvm::verifyModule(*M, &llvm::errs())) {
        llvm::errs() << ProgName << ": " << Filename
                     << ": error: input module is broken!\n";
        return 1;
    }

    M->dump();

    llvm::errs() << "-----------------------------------------------------\n";

    MDTracker mdTracker;
    mdTracker.processModuleMDNode(*M);
    mdTracker.print(llvm::errs());

    return 0;
}

int main(int argc, char **argv) {
    // Print a stack trace if we signal out.
    llvm::sys::PrintStackTraceOnErrorSignal();
    llvm::PrettyStackTraceProgram X(argc, argv);

    llvm::LLVMContext &Context = llvm::getGlobalContext();
    llvm::llvm_shutdown_obj Y;  // Call llvm_shutdown() on exit.

    llvm::cl::ParseCommandLineOptions(argc, argv, "llvm IR Dwarf Dumper\n");

    return DumpInput(argv[0], InputFilename, Context);
}
\end{lstlisting}






\end{document}
