\documentclass{ctexart}
\usepackage[colorlinks,
            linkcolor=black,
            anchorcolor=blue,
            citecolor=green]{hyperref}
  


\usepackage{multirow}

\usepackage{geometry}
\geometry{left=2.5cm,right=2.5cm,top=2.5cm,bottom=2.5cm}

\hfuzz=\maxdimen
\tolerance=10000
\hbadness=10000

\begin{document}
\tableofcontents

\section{第1章\ 计算机网络体系结构}

\subsection{计算机网络概述}
\subsubsection{计算机网络的概念}
计算机网络是一个将分散的、具有独立功能的计算机系统,通过通信设备与线路连接起来,由功能完善的软件实现资源共享和信息传递的系统。


\subsubsection{计算机网络的组成}
\subsubsection{计算机网络的功能}
1、数据通信(最基本和最重要的功能)

2、资源共享

3、分布式处理

4、提高可靠性

5、负载均衡
\subsubsection{计算机网络的分类}
1、按分布范围分类
1)广域网wan~交换技术

2)城域网man~以太网技术

3)局域网lan~广播技术

4)个人区域网pan

2、按传输技术分类

3、按拓扑结构分类

4、按使用者分类

5、按交换技术分类

6、按传输介质分类

\subsubsection{计算机网络的标准化工作及相关组织}
1)因特网草案

2)建议标准

3)草案标准

4)因特网标准

\subsubsection{计算机网络的性能指标}
带宽

时延

往返时延

吞吐量

速率

\subsection{计算机网络体系结构与参考模型}
\subsubsection{计算机网络分层结构}
\subsubsection{计算机网络协议、借口、服务的概念}
\subsubsection{ISO/OSI参考模型和TCP/IP模型}
1、OSI参考模型
(1)物理层~Physical~layer

(2)数据链路层~Data~Linker

(3)网络层~Network~Layer

(4)传输层~Transport~Layer

(5)会话层~Session~Layer

(6)表示层~Presentation~Layer

(7)应用层~Application~Layer

2、TCP/IP模型

(1)网络接口层

(2)网际层

(3)传输层

(4)应用层

\section{第2章\ 物理层}
\subsection{通信基础}
\subsubsection{基本概念}
1、数据、信号与码元

数据是指传送信息的实体。

信号是数据电器的或电磁的表现。

码元是指用一个固定时长的信号波形(数字脉冲),表示一位k进制数字,代表不同离散数值的基本波形,是数字通信中数字信号的计量单位,这个时长内的信号称为k进制码元,该时长称为码元宽度。
\\\\
2、信源、信道与信宿

信源:产生和发送数据的源头。

信道:信号的传输媒介

信宿:接受数据的终点。

通信信息交互方式:

1)单工通信

2)半双工通信

3)全双工通信

3、速率、波特与带宽

速率(数据率),数据的传输速度。

1)码元传输速率

\subsubsection{奈奎斯特定理与香农定理}
1、奈奎斯特定理

奈氏准则:理想低通信道下的极限数据传输率=2W$log_2 V$

2、香农定理

香农定理:信道的极限数据传输速率$Wlog_2 V(1+S/N)$
\subsubsection{编码与调制}
数据变换为模拟信号的过程为调制。

数据变换为数字信号的过程为编码。

1、数字数据编码为数字信号

(1)非归零码

(2)曼彻斯特编码

(3)差分曼彻斯特编码

(4)4B/5B编码

2、数字数据调制为模拟信号

(1)幅移键控ASK

(2)频移键控FSK

(3)相移键控PSK

(4)正交振幅调制QAM

\subsubsection{电路交换、报文交换与分组交换}
1、电路交换

2、报文交换

3、分组交换

\subsubsection{数据包与虚电路}

\subsection{传输介质}

\subsubsection{双绞线、同轴电缆、光纤与无线传输介质}
1、双绞线

2、同轴电缆

3、光纤

4、无线传输介质
无线点播、微博、红外线、激光

\subsubsection{物理层接口的特性}
1)机械特性

2)电气特性

3)功能特性

4)规程特性

\subsection{物理层设备}

\subsubsection{中继器}
又称转发器,将数字信号整形放大再转发出去,消除信号由于经过一长段电缆,因噪声或其他原因造成的失真和衰减,使信号的波形和强度达到所需要的要求,来扩大网络传输的距离。

\subsubsection{集线器}
多端口的中继器。半双工。

\section{第3章\ 数据链路层}

\subsection{数据链路层的功能}

\subsubsection{为网络层提供服务}
1)无确认的无连接服务

适用于实时通信或误码率较低的通信信道:以太网

2)有确认的无连接服务

适用于误码率较高的通信信道:无线通信

3)有确认的面向连接服务


\subsubsection{链路管理}

\subsubsection{帧界定、帧同步与透明传输}

?????帧界定:首部和尾部中含有很多控制信息,确定了帧的界限。

?????帧同步:接收方应当能从接受到的二进制比特流中区分出帧的起始与终止。

透明传输:不管传送数据是什么样的比特组合,都能在链路上传送。

\subsubsection{流量控制}
限制发送方的数据流量,使其发送速率不致超过接收方的接受能力。避免前面来不及接受的帧被后面不断发送的帧淹没,造成帧丢失而出错。

\subsubsection{差错控制}


位错:帧中的某些位出现了差错。

帧错:帧的丢失、重复或失序等错误。

\subsection{组帧}
\subsubsection{字符计数法}

帧头使用计数字段标明帧内字符数。

脆弱,计数字段出错,西区了帧边界话费的依据,接收方无法判断传输帧的结束位和下一帧的开始位,收发双方失去同步,造成灾难性的后果。

\subsubsection{字符填充的首位界定符法}

使用一些特定的字符来界定一帧的开始(DLE STX)与结束(DLE ETX)。

复杂、不兼容

\subsubsection{比特填充的首位标志法}

01111110使用一个特定的比特模式来标志一阵的开始和结束。信息位中每5个1后面跟一个0。

容易硬件实现,性能优于字符填充法。

\subsubsection{违规编码法}

借用违规编码序列来界定帧的起始和终止。

\subsection{差错控制}

\subsubsection{检错编码}

1、奇偶校验码

n-1位信息元和1位校验元组成。

奇校验码:附加校验元后码长为n码字中1的个数为奇数。

2、循环冗余码(Cyclic Redundancy Code,CRC)

G(x)为除数,G(x)(最高位和最低位为1)的阶数为r。

m~bit报文,除以G(x),生成r~bit的帧检验序列(FCS),帧(m+r)。


\subsubsection{纠错编码}

海明编码

海明码的1,2,4,8...位是校验位P,其余位填入m位数据D,S为海明编码后的数据。

校验规则:信息位只能是校验位之和,校验位取值为信息位加数位的异或。

1=1,2=2,3=1+2,4=4,5=1+4,6=2+4,7=1+2+4,8=8,9=1+8,10=2+8,11=1+2+8...

$S_1=P_1$,$S_2=P_2$,$S_3=D_1$,$S_4=P_3$,$S_5=D_2$,$S_6=D_3$,$S_7=D_4$,$S_8=P_4$,$S_9=D_5$,$S_10=D_6$,$S_11=D_7$...

$P_1 \oplus D_1 \oplus D_2 \oplus D_4 \oplus D_5 \oplus D_7 ... = 0$

$S_1 \oplus S_3 \oplus S_5 \oplus S_7  \oplus S_9 \oplus S_11...=0$
\\

$P_2 \oplus D_1 \oplus D_3 \oplus D_4 \oplus D_6 \oplus D_7...=0$

$S_2 \oplus S_3 \oplus S_6 \oplus S_7  \oplus S_10 \oplus S_11...=0$
\\

$P_3 \oplus D_2 \oplus D_3 \oplus D_4 \oplus D_6 \oplus D_7...=0$

$S_4 \oplus S_5 \oplus S_6 \oplus S_7 \oplus S_10 \oplus S_11...=0$
\\

$P_4 \oplus D_5 \oplus D_6 \oplus \oplus D_7...=0$

$S_8 \oplus D_9 \oplus D_10 \oplus \oplus D_11...=0$

\subsection{流量控制与可靠传输机制}
\subsubsection{流量控制、可靠传输与滑动窗口机制}
1、停止-等待流量控制基本原理
2、滑动窗口流量控制基本原理
3、可靠传输机制
\subsubsection{单帧滑动窗口与停止-等待协议}

\subsubsection{多帧滑动窗口与后退N帧协议(GBN)}

\subsubsection{多帧滑动窗口与选择重传协议(SR)}

\subsection{介质访问控制}

\subsection{局域网}

\subsection{广域网}

\subsection{数据链路层设备}


\section{第4章\ 网络层}

\subsection{网络层的功能}

\subsubsection{异构网络互联}
网络互联:指将两个以上的计算机网络,通过一定的方法,用一种或多种通信处理设备(即中间设备)相互连接起来,以构成更大的网络系统。
\\\\
中间设备:又称中间系统或中继系统。

根据中继系统所在层次,分为下列四中不同的中继系统。

物理层:中继器、集线器(Hub)

数据链路层:网桥、交换机

网络层:路由器

网络层以上:网关
\\\\

\subsubsection{路由与转发}
路由器功能:路由选择(确定那一条路径)、分组转发(当一个分组到达时所采取的动作)

路由选择:指按照复杂的分布式算法,根据从各相邻路由器所得到的关于整个网络拓扑的变化情况,动态的改变所选择的路由。

分组转发:指路由器根据转发表将用户的IP数据包从合适的端口转发出去。

\subsubsection{拥塞控制}


\subsection{路由算法}
\subsubsection{静态路由与动态路由}
静态路由算法:手工配置的路由信息。

简便可靠,在负荷稳定、拓扑变化不大的网络中运行效果很好。所以在高度安全的军事系统和较小的商业网络中广泛运用。
\\\\
动态路由算法:路由器上的路由表项是通过相互联接的路由器之间彼此交换信息,任何按照一定的算法优化处理的,这些信息在一定的时间里不断更新,以适应不断变化的网络,随时获得最优的寻路效果。

能改善网络性能并有助于流量控制,但算法复制,增加网络负担,有时候反映太快引起振荡,或反映太慢影响网络路由的一致性。
\subsubsection{距离-向量路由算法}
路由选择表:

每条路径的目的地(另一结点)

路径代价(距离)
\\\\
更新表的情况:

1)被通告一条路由,该路由在本结点的路由表中不存在,本地系统加入这条新的路由。

2)发送来的路由信息有一条到达某个目的地的路由,该路由比当前使用的路由有较短的距离。用经过发送路由信息结点的新路友替换路由表中达到那个目的地的现有路由。
\\\\
RIP算法,跳数作为距离的度量

\subsubsection{链路状态路由算法}
每个参与该算法的结点都有完全的网络拓扑信息,并执两个功能:

1)主动测试所有邻接结点的状态

2)定期将链路状态传播给所有其他的结点。
\\\\
链路状态路由算法的三点特征:
1)使用泛洪法,向系统中所有路由器信息发送信息。路由器通过所有端口想相邻路由器发送信息,但不向刚发来信息的路由器发送信息。

2)发送信息就是与路由器相邻的所有路由器的链路状态(度量)。
OSPF的度量为:费用、距离、时延、带宽。

3)链路状态发生变化时,路由器才向所有路由器发送此信息。
\\\\
适用情况:
用于大型的或路由信息变化聚敛的互联网环境。
\\\\
优点:
每个路由结点都使用原始状态数据独立计算路径,而不依赖中间结点的计算
链路状态报文不加改变地传播,易于查找故障。
链路状态报文仅运载来自单个结点关于直接链路的信息,其大小与网络中的路由结点数目无关,所有比距离-向量算法由更好的规模可伸展性。

\subsubsection{层次路由}
内部网关协议(IGP,域内路由选择):系统内部使用的路由选择协议。
RIP、OSPF
外部网关协议(EGP,域间路由选择):系统建使用的路由选择协议。
BGP

采用分层次划分区域的方法,OSPF将一个自治系统再划分为若干个区域,每个路由器都知道在本区域内如何把分组路由到目的地的细节,但不知道其他区域的内部结构。

采用分层次划分区域的方法,使交换信息的种类增多了,同时使OSPF协议更加复杂。但是使每一个区域内部交换路由信息的通信量减小,所有OSPF协议考研用于规模很大的自治系统中。
\subsection{IPv4}
\subsubsection{IPv4分组}
1、IPv4分组的格式
IP分组是\textbf{首部}和\textbf{数据}两部分组成。

首部前一部分是固定长度\textbf{20字节(160位)},后一部分是儿靠边的,用来提供错误检测及安全等机制。

版本(4位):

首部长度(4位):32位(4字节)为单位。最大60字节(15×4字节)。常用是20字节,可选字段为空。

总长度(16位):字节为单位。最大长度为$2^{16}$-1=65535字节.以太网帧的最大传送但与MTU为1500字节,一个IP数据包总长度不能超过下面数据链路层的MTU值。

标识(16位):产生一个数据报就加1,赋值给标识字段。一个数据报长度超过网络的MTU时,分片并且每片赋值一次标识号以便重新拼装原来的数据报。

标志(3位):

片便宜(13位):

首部检验和

生存时间

协议:

源地址字段:

目的地址字段:

2、IP数据报分片

3、网络层转发分组的流程
1)从数据报的收不提取目的主机的IP地址D,得到目的网络地址N。

2)若网络N与次路由器直接项链,则把数据报直接交付目的主机D。否则间接交付,执行3)。

3)若路由表中有目的地址为D的特定主机路由,,则把数据报传送给路由表中致命的下一路由器。否则执行4)。

4)若路由表中有到达网络N的路由,则把数据报传送给路由表指明的下一跳路由器,否则执行5)。

5)若路由表有一个末日路由,则把数据报传送给末日路由器。否则执行6)。

6)报告转发分组出错。

\subsubsection{IPv4地址与NAT}
1、IPv4地址

IP::={<网络号>,<主机号>}

主机号全为0表示本网络本身。
主机号全为1表示网络广播地址。

\begin{table}[!hbp]
\caption{三种类别的IP地址的使用范围}
\begin{tabular}{|c|c|c|c|c|}
\hline
网络类别 & 最大可用网络数& 第一个可用的网络号 & 最后一个可用的网络号 & 每个网络中的最大主机数 \\
\hline
A & $2^7-2$ & 1 & 126 & $2^{24}-2$ \\
\hline
B&$2^{14}-1$&128.1&191.255&$2^{16}-2$\\
\hline
C&$2^{21}-1$&192.0.1&223.255.255&$2^8-2$\\
\hline
\end{tabular}
\end{table} 

2、网络地址转换
网络地址转换(NAT),是通过专用网络地址转换为公用网络地址,从而对外因此了内部管理的IP地址。

私有地址网段
\begin{table}[!hbp]
\begin{tabular}{|c|c|c|c|c|}
\hline
A类 & 1个A类网段 & 10.0.0.0\~{}10.255.255.255 \\
\hline
B类 & 16个B类网段 & 172.16.0.0\~{}172.31.255.255 \\
\hline
C类 & 256个C类网段 & 192.168.0.0\~{}192.168.255.255 \\
\hline
\end{tabular}
\end{table} 
\subsubsection{子网划分与子网掩码、CIDR}
1、子网划分

主机号再分出子网号,分成三级IP地址

IP地址={<网络号>,<子网号>,<主机号>}



2、子网掩码

3、无分类编址CIDR
\subsubsection{ARP协议、DHCP协议与ICMP协议}
1、IP地址与硬件地址

2、地址解析协议ARP

3、动态主机配置协议DHCP

4、网际控制报文协议ICMP

\subsection{IPV6}

\subsection{路由协议}
\subsection{IP组播}
\subsection{移动IP}
\subsection{网络层设备}

\section{第5章\ 传输层}
\subsection{传输层提供的服务}
\subsection{UDP协议}
\subsection{TCP协议}

\section{第6章\ 应用层}
\subsection{网络应用模型}
\subsection{DNS系统}
\subsection{文件传输协议FTP}
\subsection{电子邮件}
\subsection{万维网WWW}


\end{document}
