\documentclass{ctexart}
\usepackage{verbatim}
\usepackage{amsmath}

%箭头等号等
\usepackage{extarrows}
%链接颜色,主要是目录链接
\usepackage[colorlinks,
            linkcolor=black,
            anchorcolor=blue,
            citecolor=green]{hyperref}


\hfuzz=\maxdimen
\tolerance=10000
\hbadness=10000

%多栏目录
\usepackage{multicol}
\AtBeginDocument{\addtocontents{toc}{\protect\begin{multicols}{2}}}
\AtEndDocument{\addtocontents{toc}{\protect\end{multicols}}}

\begin{document}
%\setlength{\parindent}{0pt}

%\begin{comment}
\tableofcontents
\newpage


\section{第一章 函数 极限 连续}
\subsection{函数}
\subsubsection{定义}
基本初等函数:
常数函数\ 幂函数\ 指数函数\ 对数函数\ 三角函数\ 反三角函数

初等函数:基本初等函数经过有限次的四则运算、复合运算得到的函数。
\\\\

\subsubsection{性质}

数列收敛的性质:唯一、有界、保序

\subsection{极限}
\subsubsection{定义}
数列的极限:设数列${a_n}$,a是常数,若对任意的$\varepsilon$>0,总存在自然数N,队人一的自然数n>N,由$|a_n>a|<\varepsilon$,
则称数列${a_n}$的极限是a,或数列${a_n}$收敛于a,表示为$\lim\limits_{n \rightarrow \infty}a_n = a$。

若数列不存在极限,则称数列${a_n}$发散。
\\\\

极限:设函数f(x)在点$x_0$的某个去心邻域$\mathring{U}(x_0$)内有定义,A是常数,若$\forall \varepsilon > 0,\exists \delta >0,\forall x: 0<|x-x_0|<\delta,有|f(x)-A|<\varepsilon$,则称函数f(x)当x$\rightarrow x_0$时存在极限,极限是A,记作
\begin{center}
$\lim\limits_{x \rightarrow x_0} f(x)=A$\\
\end{center}。
\\\\

导数:设函数y=f(x)在点$x_0$的某个邻域内由定义,党自变量x在$x_0$处获得增量$\bigtriangleup x$(点$x_0+\bigtriangleup x$仍在该邻域内)时,相应的函数的增量为$\bigtriangleup y=f(x_0+\bigtriangleup x)-f(x_0)$,如果

\begin{center}
$\lim\limits_{\bigtriangleup \rightarrow 0} \frac{\bigtriangleup y}{\bigtriangleup x}=\lim\limits_{\bigtriangleup x \rightarrow 0} \frac{f(x_0+\bigtriangleup x)-f(x_0)}{\bigtriangleup x}$
\end{center}

存在,则称函数y=f(x)在点$x_0$处可导,则成这个极限为函数y=f(x)在点$x_0$处的导数记为$f'(x_0)$。也记为

\begin{center}
$y'|_{x=x_0},\frac{dy}{dx}|_{x=x_0}$或$\frac{d}{dx}f(x)|_{x=x_0}$
\end{center}

即

\begin{center}
$f'(x_0)=\lim\limits_{\bigtriangleup x \rightarrow 0}\frac{\bigtriangleup y}{\bigtriangleup x}=\lim\limits_{\bigtriangleup x \rightarrow 0} \frac{f(x_0+\bigtriangleup x)-f(x_0)}{\bigtriangleup x}$
\end{center}


\subsubsection{性质}
极限的性质:唯一、局部有界、局部保序、局部保号
保号性:$\lim\limits_{x\rightarrow x_0}f(x)=A,A\neq 0,$则存在$x_0$的一个去心邻域,此邻域中f(x)与A同号。
\\\\

极限存在$\Leftrightarrow f(x_0^-)=f(x_0^+)=A$
\\\\

夹逼定理:在$x_0$的去心邻域中
$g(x) \leq f(x) \leq h(x),\lim\limits_{x \rightarrow x_0}g(x)=\lim\limits_{x \rightarrow x_0}h(x)=A$
,则
$\lim\limits_{x \rightarrow x_0}f(x)=A$
\\\\

柯西收敛准则:极限$\lim\limits_{x\rightarrow x_0}f(x)$存在$\leftrightarrow \forall \varepsilon >0,\exists \delta >0,\forall x',x'':0<|x'-x''|<\delta 与 0<|x''-x_0|<\delta ,\text{有}|f(x')-f(x'')|<\varepsilon$
\\\\

等价无穷小的替换:设$x\rightarrow x_0$时, $\alpha (x) \sim a(x)$ $\beta (x) \sim b(x)$,则
\begin{center}
$\lim\limits_{x\rightarrow x_0}\frac{\alpha (x)\gamma (x)}{\beta (x)\delta (x)}=\lim\limits_{x\rightarrow x_0}\frac{a (x)\gamma (x)}{b (x)\delta (x)}$ 
\end{center}
错误示例:
\begin{center}
$\lim\limits_{x\rightarrow 0}\frac{ln(1+x)-x}{x^2}=\lim\limits_{\frac{ln(1+x)}{x^2}-\frac{x}{x^2}}\xlongequal{*}\lim\limits_{\frac{x}{x^2}-\frac{x}{x^2}}=0$
\end{center}
*这一步不能用ln(1+x)~x替换,因为不是整个式子的乘、除用等价无穷小替换,而是局部的用等价无穷小替换。
\subsubsection{公式}
极限的加减乘除\\

(1)$\lim\limits_{x \rightarrow x_0 }[f(x)\pm g(x)]=\lim\limits_{x \rightarrow x_0} f(x)\pm \lim\limits_{x \rightarrow x_0}g(x)$
\\

(2)$\lim\limits_{x \rightarrow x_0}f(x)g(x)=\lim\limits_{x \rightarrow x_0}f(x) \lim\limits_{x \rightarrow x_0}g(x)$
\\

(3)$\lim\limits_{x \rightarrow x_0} \frac{f(x)}{g(x)}=\frac{\lim\limits_{x \rightarrow x_0}f(x)}{\lim\limits_{x \rightarrow x_0}g(x)}$
\\\\

洛必答法则:\\
$\lim\limits_{x\rightarrow x_0}f(x)=0 \text{或} \infty,\lim\limits_{x\rightarrow x_0}g(x)=0 \text{或} \infty$\\
f(x)与g(x)在$x_0$的去心邻域$\mathring{U}$内可导,且g'(x)$\neq$0\\
$\lim\limits_{x\rightarrow x_0}\frac{f(x)}{g(x)}=\lim\limits_{x\rightarrow x_0}\frac{f'(x)}{g'(x)}$
几个比较重要的极限
\\\\

$\lim\limits_{n \rightarrow \infty }(1+ \frac{1}{n})^n=e$

$\lim\limits_{n \rightarrow 0 }(1+ n)^\frac{1}{n}=e$

$\lim\limits_{x\rightarrow \infty} \sqrt[n]{n}=1$

如果$x\rightarrow 0,sinx \sim x,tanx \sim x, ln(1+x)\sim x, (1+x)^a-1\sim ax, arcsinx\sim x, arctanx\sim x, a^x-1\sim xlna(a>0,a \neq 1)$
\subsubsection{例题}

1、利用极限定义求导数\\

(1)$f(x)=ax+b$\\

$\lim\limits_{h \rightarrow 0}\frac{f(x+h)-f(x)}{h}=\lim\limits_{h \rightarrow 0}\frac{[a(x+h)+b]-(ax+b)}{h}=\lim\limits_{h \rightarrow 0}a=a$
\\

(2)$f(x)=x^n$\\

$\lim\limits_{h \rightarrow 0}\frac{f(x+h)f(x)}{h}=\lim\limits_{h \rightarrow 0}\frac{(x+h)^n-x^n}{h}\\=\lim\limits_{h \rightarrow 0}\frac{(x^n+C_n^1x^{n-1}h+C_n^2x^{n-2}h^2+...+C_n^{n-1}xh^{n-1}+C_n^nh^n)-x^n}{h}\\=\lim\limits_{h \rightarrow 0}=\lim\limits_{h \rightarrow 0}(C_n^1x^{n-1}+C_n^2x^{n-2}h+...+C_n^n-1xh^{n-2}+C_n^nh^{n-1})\\=C_n^1x^{n-1}=nx^{n-1}$
\\

(3)$f(x)=a^x(a>0,a\neq 1)$\\

$\lim\limits_{h \rightarrow 0}\frac{a^{x+h}-a^x-a^x}{h}=a^x\lim\limits_{h \rightarrow 0}\frac{a^h-1}{h}=a^x\lim\limits_{h \rightarrow 0}\frac{e^{hlna}-1}{h}\\
$
其中
\begin{center}
$\lim\limits_{h\rightarrow 0} (1+hlna)^{\frac{1}{hlna}}=e$\\
$\lim\limits_{h\rightarrow 0}(1+hlna)=e^{hlna}$\\
$\lim\limits_{h\rightarrow 0}e^{hlna}-1=hlna$\\
\end{center}
所以\\
原式$=a^x\lim\limits_{h\rightarrow 0}\frac{hlna}{h}=a^xlna$

2、定义判断
设f(x)=u(x)+v(x),g(x)=u(x)-v(x),并设$\lim\limits_{x\rightarrow x_0}u(x)$与$\lim\limits_{x\rightarrow x_0}v(x)$都不存在,则:\\
A若$\lim\limits_{x\rightarrow x_0}f(x)$不存在,则$\lim\limits_{x\rightarrow x_0}g(x)$必存在。\\
B若$\lim\limits_{x\rightarrow x_0}f(x)$不存在,则$\lim\limits_{x\rightarrow x_0}g(x)$必不存在。\\
C若$\lim\limits_{x\rightarrow x_0}f(x)$存在,则$\lim\limits_{x\rightarrow x_0}g(x)$必不存在。\\
D若$\lim\limits_{x\rightarrow x_0}f(x)$存在,则$\lim\limits_{x\rightarrow x_0}g(x)$必存在。\\


\subsection{连续}
\subsubsection{定义}
连续:设函数f(x)在U($x_0$)有定义。若函数f(x)在$x_0$存在极限,且极限就是f($x_0$),即
\begin{center}
$\lim\limits_{x \rightarrow x_0}f(x)=f(x_0)$
\end{center}
,则称函数f(x)在$x_0$连续,$x_0$是函数f(x)的连续点。
\\\\

间断点:若函数f(x)在点$x_0$不满足连续定义的条件,则称函数f(x)在点$x_0$是间断的,并称$x_0$是函数f(x)的间断点。

(1)函数f(x)在$x_0$没有定义。

(2)极限$\lim\limits_{x \rightarrow x_0}f(x)$存在,即$f(x_0-0)=f(x_0+0)$,但$\lim\limits_{x \rightarrow x_0}f(x) \neq f(x_0)$(可去间断点)。

(3)极限$\lim\limits_{x \rightarrow x_0}f(x)$不存在

\textcircled{1}(第一类间断点)$f(x_0-0)$与$f(x_0+0)$都存在,但不相等

\textcircled{2}(第二类间断点)$f(x_0-0)$与$f(x_0+0)$至少一个不存在
\\\\
\subsubsection{性质}
连续函数和差积商(分母不为零)连续。

初等函数在定义域内连续。
\subsubsection{例题}

\section{第二章 一元函数微分学}
\subsection{定义}
\subsection{性质}
\subsection{公式}
导数公式

$(a^x)'=a^xlna$(a为常数,a>0,a$\neq$1) 

$(log_ax)'=\frac{1}{xlna}(a>0,a\neq 1)$(a>1,a$\neq$1) 

$(tanx)'={sec}^2x$

$(cotx)'=-{csc^2}x$

$(secx)'=secxtanx$

$(cscx)'=-cscxcotx$

$(arcsinx)'=\frac{1}{\sqrt{1-x^2}}$

$(arccosx)'=-\frac{1}{\sqrt{1-x^2}}$

$ $

$ $
\\\\
微分公式

$da^x=a^xlnadx$(a为常数,a>0,a$\neq$1)
$dlog_ax=\frac{1}{xlna}dx(a>0,a\neq 1)$ (a>1,a$\neq$1)




\subsection{例题库}


\section{第三章 一元函数积分学}
\subsection{定义}
\subsection{性质}
\subsection{公式}
(1) $\int x^adx=\frac{1}{a+1}x^{a+1}+C(a\neq-1)$

(2) $\int \frac{1}{x}dx=ln|x|+C$

(3) $\int a^xdx=\frac{a^x}{lna}+C(a>0,a\neq1)$

(4) $\int e^xdx=e^x+C$

(5) $\int sinxdx=-cosx+C$

(6) $\int cosxdx=sinx+C$

(7) $\int tanxdx=-ln|cosx|+C$

(8) $\int cotdx=sinx+C$

(9) $\int secxdx=ln|secx+tanx|+C$

(10)  $\int cscxdx=ln|cscx-cotx|+C$

(11)  $\int sec^2dx=tanx+C$       

(12)  $\int csc^2xdx=-cotx+C$       

(13)  $\int \frac{1}{a^2+x^2}dx=\frac{1}{a}arctan\frac{x}{a}+C$

(14)  $\int \frac{1}{a^2-x^2}dx=\frac{1}{2a}ln|\frac{a+x}{a-x}|+C$

(15)  $\int \frac{1}{\sqrt{a^2-x^2}}dx=arcsin\frac{x}{a}+C$

(16)  $\int \frac{dx}{\sqrt{x^2 \pm a^2}}=ln|x+\sqrt{x^2 \pm a^2}|+C$

\subsection{例题库}


\section{第四章 向量代数与空间解析集合}
\subsection{定义}
\subsection{性质}
\subsection{公式}
\subsection{例题库}


\section{第五章 多元函数微分学}
\subsection{定义}
\subsection{性质}
\subsection{公式}
\subsection{例题库}
\newpage



%\end{comment}
\section{第六章 多元函数积分学}

\subsection{二重积分}

\subsection{三重积分}

\subsection{对弧长的线积分(第一类曲线积分)}
\subsubsection{定义} 
设L为XOY面上的分段曲线弧段,$f(x,y)$为定义在L上的有界函数,则f(x,y)在L上对弧长的线积分为\\
\begin{center}
$\int_Lf(x,y)ds \overset{\bigtriangleup}{=} \lim\limits_{\lambda \to 0} \sum\limits_{i=1}^n f(\xi _i,\eta _i)\bigtriangleup s_i$,
\end{center}
\par
其中$\bigtriangleup s_i$表示第i段小弧段长度,$\lambda=\max\limits_{1\leq i\leq n} \bigtriangleup s_i$,如果f(x,y)在L上连续,则$\int _L f(x,y)ds$存在。 

\subsubsection{性质}
性质1 与积分路径方向无关,即
\begin{center}
$\int _{L(\widehat{AB})} f(x,y)ds=\int _{L(\widehat{BA})} f(x,y)ds=\int_A^Bf(x,y)ds(A<B)$\\
\end{center}
\par
备注:第一类曲线积分中的ds恒为正数,$ds=\sqrt{x^2+y^2}$,定义中的$\bigtriangleup s$恒为正。在定积分中的dx则是可正可负的。\\\par
性质2 奇偶性与对称性

\subsubsection{公式}
公式1 参数方程的计算
曲线L为参数方程
$\left\{ 
	\begin{array}{l}
	y=y(t),\\
	x=x(t),
	\end{array}
\right. \alpha \leq t \leq \beta$,则\\
\begin{center}
$\int _Lf(x,y)ds=\int _\alpha ^beta f(x(t),y(t))\sqrt{x'^2(t)+y'^2(t)}dt$
\end{center}\par
公式2 直角坐标的计算
曲线L为直角坐标方程y=y(x),$a \leq x \leq b$,则
\begin{center}
$\int _L f(x,y)ds = \int _a ^b f(x,y(x))\sqrt{1+y'^2(x)}dx$
\end{center}\par
公式3 极坐标的计算
曲线L为极坐标方程$\rho=\rho (\theta)$,$\alpha \leq \theta \leq \beta$,则
\begin{center}
$\int _L f(x,y)ds=\int _\alpha ^\beta f(\rho (\theta) cos\theta , \rho(\theta) sin \theta)\sqrt{\rho^2+\rho'^2}d\rho$
\end{center}

\begin{comment}
\subsubsection{例题库}
例1、计算$\oint _Le^{\sqrt{x^2+y^2}}ds$,其中L为圆$ x^2+y^2=a^2$、直线$y=a$及x轴正方向所围扇形边界。
\\
\\
$\oint_Le^{\sqrt{x^2+y^2}}ds\\
=\oint_{L1}e^{\sqrt{x^2+y^2}}ds+\oint_{L2}e^{\sqrt{x^2+y^2}}+\oint_{L3}e^{\sqrt{x^2+y^2}}\\
=\int_0^{\frac{\sqrt{2}}{2}a} e^{\sqrt{2}x}d\sqrt2x + \int_0^\frac{\pi}{4} e^a\sqrt{(-acos\Theta)^2+(asin\Theta)^2}d\Theta+\int_0^a e^{\sqrt{x^2+0}} \sqrt{1+0}dx\\
=\int_0^{\frac{\sqrt{2}}{2} a}de^{\sqrt{2}x}+\int_0^\frac{\pi}{4}ae^ad\Theta+\int_0^ae^xdx\\
=e^a-1+\frac{\pi}{4}ae^a+e^a-1\\
=\frac{\pi}{4} ae^a+2e^a-2$

例2、计算$\_\tau(x^2+y^2+z^2)ds$,其中$\tau$:$x=acost$ $y=asint$ $z=kt$ $(0\leq t\leq2\pi)$
\\
\\
$ds=\sqrt{x'^2+y'^2+z'^2}dt\\
=\sqrt{(-asint)^2+(acost)^2+k^2}dt\\
=\sqrt{a^2+k^2}dt$\\\\
$\int_\tau(x^2+y^2+z^2)ds\\
=\int_0^{2\pi}(a^2cos^2t+a^2sin^2t+k^2t^2)\sqrt{a^2+k^2}dt\\
=\int_0^{2\pi} a^2\sqrt{(a^2+k^2)}dt+k^2\sqrt{a^2+k^2}\int_0^{2\pi}t^2dt\\
=a^2\sqrt{(a^2+k^2)} \int_0^{2\pi} dt *2\pi+k^2\sqrt{a^2+k^2}\frac{(2\pi)^3}{3}\\
=a^2\sqrt{(a^2+k^2)} *2\pi+\frac{8}{3} \pi^3k^2 \sqrt{a^2+k^2}$

例3、计算$I=\int _L |x|ds$,其中L为双扭线$(x^2+y^2)^2=a^2(x^2-y^2),a>0$
\\\\
$\int _L |x|ds
=\int _{L_1} |x|ds+\int _{L_2} |x|ds+\int _{L_3} |x|ds+\int _{L_4} |x|ds
=4\int \{L_1} 
$

例$\oint_L (x^2+y^2-2x+1)^nds$,其中L为圆周$x^2+y^2-2x=0$;
$\rho=2cos\theta$
$\oint_L (x^2+y^2-2x+1)^nds\\
=\oint ds
=\int _0^2cos\theta\sqrt{\rho^2+(\rho)'^2}d\theta\\
=\int $
\end{comment}



\subsection{对坐标的线积分(第二类曲线积分)}
\subsubsection{定义} 
\subsubsection{公式}

格林公式:
设闭区域D由光滑的曲线L围成,函数P(x,y)及Q(x,y)在D上具有一阶连续偏导数,则有
\begin{center}
$\iint_{D} (\frac{(\partial Q)}{\partial x}-\frac{(\partial P)}{\partial y})dxdy=\oint_L Pdx+Qdy$,
\end{center}
其中L是D的取正向的边界曲线。
\subsubsection{性质}

设P(x,y),Q(x,y)在单联通区域D上有连续一阶偏导数,则下列四条等价:

1、线积分$\int Pdx+Qdy$与路径无关

2、$\oint_C Pdx+Qdy=0$,其中C为D中任意分段光滑闭曲线

3、$\frac{\partial Q}{\partial y}=\frac{\partial Q}{\partial x}$,$\exists (x,y) \in D$

4、存在可微函数F(x,y),使P(x,y)dx+Q(x,y)dy=dF(x,y)

\subsubsection{例题库}

\subsection{对坐标的线积分(第一类曲面积分)}
\subsubsection{定义}
设$\sum$为分片光滑曲面片,f(x,y,z)为定义在$\sum$上的有界函数,f(x,y,z)在$\sum$上对面积的面积分为
\begin{center}
 $\iint_{\sum} f(x,y,z)dS\stackrel{\bigtriangleup}{} \lim\limits_{\lambda \rightarrow 0} \sum\limits_{i=1}^{n} f(\xi_i ,\eta_i ,\zeta _i)\bigtriangleup S_i$
\end{center}
其中$\bigtriangleup S_i$为第i个小曲面块的面积,如果f(x,y,z)在$\sum$上连续,则$\iint\limits_{\sum}f(x,y,z)dS$存在。
\subsubsection{公式}


\subsection{对坐标的线积分(第二类曲面积分)}

\end{document}