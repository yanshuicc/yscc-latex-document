%this document is not about Chinese politics, just for The national entrance-examination-for-postgradute 

\documentclass{ctexart}
\usepackage[colorlinks,
            linkcolor=black,
            anchorcolor=blue,
            citecolor=green]{hyperref}

\usepackage{geometry}
\geometry{left=2.5cm,right=2.5cm,top=2.5cm,bottom=2.5cm}

\hfuzz=\maxdimen
\tolerance=10000
\hbadness=10000

\begin{document}
\tableofcontents


\section{第一部分\ 马克思主义基本原理概论}

\subsection{第一章\ 马克思主义是关于无产阶级和人类解放的科学}
\subsubsection{一、马克思主义的产生和发展}

1.马克思主义的含义

马克思主义的内容:马克思主义哲学、马克思主义政治经济学和科学社会主义。作为中国共产党和社会主义事业指导思想的马克思主义,既包括由马克思、恩格斯创立和列宁等发展了的马克思主义,也包括中国共产党人将其与中国具体实际相结合,形成的马克思主义中国化理论成果。
\\

马克思主义科学思想体系的精髓: 立场、观点和方法。


\\\\

2、马克思主义产生的经济社会根源、实践基础和思想渊源

首先,资本主义经济的发展为马克思主义的产生提供了经济、社会历史条件。

其次,无产阶级反对资产阶级的斗争日趋激化。

再次,马克思恩格斯的革命实践和对人类文明成果的继承与创新。

马克思主义是在批判地继承、吸收德国古典哲学、英国古典政治经济学和法国、英国的空想社会主义合理成分的基础上,在深刻分析资本主义社会的发展趋势和科学总结工人阶级斗争实践的基础上,创立和发展起来的。
\\\\

3、马克思主义的创立

马克思1845年春天《关于费尔巴哈的提纲》和马克思、恩格斯1844-1846年合写的《德意志意识形态》,标志着马克思主义的基本形成。

1847年《哲学的贫困》1848年《共产党宣言》标志着马克思主义的公开问世。
\\\\

4、马克思主义基本原理

马克思主义基本原理,是马克思主义理论体系中最基本、最核心的内容,是对马克思主义的立场、观点和方法的集中概括。

马克思主义的基本立场,是马克思主义观察、分析和解决问题的根本立足点和出发点。

马克思主义的基本观点,是关于自然、社会和人类思维规律的科学认识,是对人类思想成果和社会实践经验的科学总结。

马克思主义的基本方法,是简历在辩证唯物主义和历史唯物主义世界观、方法论基础上的思想方法和工作方法,主要包括实事求是的方法、辩证分析的方法、历史分析的方法、群众路线的方法,等等。
\\\\

5、马克思主义在实践中的发展

马克思恩格斯根据实践的发展对自己创建的理论不断充实和完善

其后,列宁等马克思主义者在领导俄国革命中实现丰富和发展。

中国共产党将马克思列宁主义确立为自己的知道思想,并在长期奋斗中坚持把马克思主义基本原理同中国具体实际相结合,发展了马克思主义,先后产生了毛泽东思想和中国特色社会主义理论体系。

\subsubsection{二、马克思主义的鲜明特征}
1、马克思主义科学性和革命性的统一
强大生命力的根源、马克思主义基本和最鲜明的特征:
以实践为基础的科学性与革命性的统一

科学性:坚持世界的物质性和真理的客观性,力求按照世界本来面目如实的认识实践,力求全面的认识事物,并且透过现象深刻地揭示事物的本质和规律,自觉接受实践的检验,并在实践中不断丰富和发展。

革命性:坚持唯物辩证法,具有彻底的批评精神
\\\\

2、马克思主义的哲学基础、政治立场、理论品质和社会理想

(1)马克思主义的哲学基础

辩证唯物主义与历史唯物主义是马克思主义最根本的世界观和方法论、哲学基础。

马克思恩格斯运用唯物史观的基本原理,着重剖析资本主义社会,揭示了资本主义经济发展的规律,形成了科学的剩余价值学说,揭露了资本主义剥削的秘密,论证了社会化大生产与资本主义私有制之间的矛盾,得出了资本主义必然灭亡、社会主要必然胜利的结论。

马克思恩格斯运用辨证唯物主义与历史唯物主义的基本原理,总结各国工人运动的斗争经验,提出来无产阶级的历史使命,指明了实现这一历史使命的方向和道路,阐明了无产阶级革命和无产阶级专政理论以及无产阶级建党学说,从而创立了科学社会主义理论。

(2)马克思主义的政治立场

马克思主义政党的一切理论和奋斗都应致力于实现以劳动人民为主题的最广大人民的根本利益,这是马克思主义最鲜明的政治立场。


首先,这是马克思主义理论的本性决定的,其次,是无产阶级的历史革命决定的,最后,是否站在最广大人民的立场上,是唯物史观与唯心史观的分水岭,也是判断马克思主义政党的试金石。

(3)马克思主义的理论品质

坚持一切从实际出发,理论联系实际,实事求是,在实践中检验真理和发展真理,是马克思主义最重要的理论品质。

首先,这种品质是马克思主义理论本质的反映。马克思主义理论的本质属性,在于它的彻底的科学性、坚定的革命性和自觉的实践性,而彻底的科学性是最根本的。其次,这种品质是人类认识发展规律的具体表现。最后这种品质是理论创新的内在要求。

(4)马克思主义的社会理想

实现物质财富极大丰富、人品精神境界极大提高、每个人自由而全面发展的共产主义社会,是马克思主义最崇高的社会理想。
\\\\

3、学习和运用马克思主义的意义和方法

学习理论,武装头脑

坚持和弘扬理论联系实际的学风

用科学的态度对待马克思主义



\subsection{第二章\ 世界的物质性及其发展规律}
\subsubsection{一、物质世界和实践}
1、物质世界的客观存在

(1)世界观、方法论和哲学

世界观是人民对整个世界的总体看法和根本观点

方法论是人们认识和改造世界所遵循的根本方法的学说和理论体系

方法论同世界观是统一的,哲学是系统化、理论化的世界观,又是方法论。

(2)哲学基本问题及其内容
思维和存在是哲学的基本问题,这是由哲学作为世界观的学问这一性质和特点所决定的。

\textcircled{1} 思维和存在的关系问题是人类认识和改造世界不能回避的最基本问题,因而也是任何哲学派别都不能回避而必须回答的问题,是解决其他一切这些问题的前提和基础。

\textcircled{2} 思维和存在的问题是划分哲学中基本派别的依据

\textcircled{3} 思维和存在的关系问题也是人类实际生活中的基本问题,它普遍存在于人类的实际生活并决定着人们的思想和行动的出发点和方向。

(3)唯物主义和唯心主义,可知论和不可知论,辩证法和形而上学

(4)马克思主义哲学的创立在哲学史上的伟大变革

(5)马克思主义的物质观及其理论意义

(6)意识的起源和本质

(7)物质和运动,运动和静止,物质运动与时间、空间

物质和运动是不可分割的

运动是物质的存在方式和根本属性

物质是一切运动变化和发展过程的实在基础和承担者

(8)社会的物质性

(9)世界物质统一性原理及其意义
\\\\

2、社会生活本质上是实践的
\\\\
(1)实践的本质、基本特征和基本形式

实践是人类能动地改造世界的客观物质性活动。

实践具有 物质性、自觉能动性、社会历史性 等基本特征。

实践是物质性的活动,具有直接现实性。

实践是人类有意识的活动,体现了自觉的能动性。

实践是社会的历史的活动,具有社会历史性的特点。

实践的基本形式 包括 物质生产实践、社会政治实践、科学文化实践

物质生产实践是人类最基本的实践活动。

社会政治实践是人们社会生活中的一个重要方面

科学文化实践是改造自然和社会的准备性和探索性的实践活动。
\\\\
(2)实践与人的存在

实践是人所独有的活动

实践集中表现了人的本质和社会性。

实践对物质世界的改造是对象性的活动。
\\\\
(3)自然界和人类社会的分化和统一

实践是使物质世界分化为自然界和人类社会的历史前提,又是使自然界和人类社会统一起来的现实基础。
\\\\
(4)人和自然的关系

人与自然的关系是在实践中形成的、始终是处于一定社会关系中的、纳入了社会过程的物质交换过程,是具有社会性的物质交换关系。
\\\\
(5)社会生活的实践本质

社会生活的实践本质体现在三个方面

实践是社会关系形成的基础

实践形成了社会生活的基本领域

实践构成了社会发展的动力。
\\\\

3、客观规律性与主管能动性\\\\
(1)规律及其客观性

规律是本质的联系

规律是必然的联系

规律是稳定的联系
\\\\
规律是客观的。客观是规律的根本特点,它的存在不依赖与人的意识。相反,人的意识及其指导下的实践却要受规律的支配。
\\\\
(2)意识的能动作用及其表现

意识活动具有目的性和计划性
意识活动具有创造性

意识活动具有指导实践改造客观世界的作用

意识具有指导、控制人的行为和生理活动的作用
\\\\
(3)主观能动性与客观规律性的关系

尊重客观规律是发展主观能动性的前提

在尊重客观规律的基础上充分发挥主观能动性
\\\\
(4)正确发挥主观能动作用

从实际出发,努力认识和把握事物的发展规律。

实践是发挥人的主观能动性的基本途径。

主观能动性的发挥,还需要一定的物质基础和物质手段。
\\\\
(5)社会历史趋向与主体选择的关系

社会历史趋向与主体选择的关系是同主观能动性与客观规律性统一原理相关联的问题。

社会历史趋向是属于历史决定论的内容,是社会历史规律的作用。

主体选择是历史主体在社会发展中的能动性和选择性。

社会生活未来发展的多种可能性是主体选择的客观前提,主体的利益和需要是选择的内在根据。

\subsubsection{二、事物的普遍联系与发展}
1、联系和发展的普遍性
\\\\
(1)联系的内涵和特点

联系具有客观性

联系具有普遍性

联系具有多样性
\\\\
(2)事物普遍联系原理的方法论意义

马克思主义关于事物普遍联系的原理,要求人们要善于分析事物的具体联系,确立整体性、开放性的观念,从动态中考察事物的普遍联系。
\\\\
(3)联系与运动、变化、发展
\\\\
(4)发展的实质

发展的实质是新事物的产生和旧事物的灭亡。
\\\\
(5)发展与过程

事物的发展是一个过程。一切事物只有经过一定的过程才能实现自身的发展。

所谓过程是指一切事物都有其产生、发展和转化为其他事物的历史,都有它的过去、现在和未来。
\\\\
(6)唯物辩证法与科学发展观

辩证唯物主义和历史唯物主义是马克思主义最根本的世界论和方法论。

联系和发展是唯物辩证法的总特征。

党的十八大是从马克思主义世界观和方法论的高度,阐述了科学发展观的理论基础和理论意义,明确提出:科学发展观是马克思主义关于发展的世界观和方法论的集中体现。
\\\\

2、对立统一规律是事务发展的根本规律
\\\\
(1)唯物辩证法的实质和核心
\\\\
(2)矛盾的同一性和斗争性及其相互关系
\\\\
(3)矛盾的同一性和斗争性在事物发展中的作用
\\\\
(4)矛盾的同一性和斗争性原理的方法论意义
\\\\
(5)矛盾的普遍性和特殊性的含义及相关关系
\\\\
(6)矛盾的普遍性和特殊性辩证关系原理的意义
\\\\

3、事物发展过程中的两边和质变、可定和否定
\\\\
(1)事物存在的质、量、度
\\\\
(2)事务发展的量变和质变及其辩证关系
\\\\
(3)事物发展过程中的肯定和否定,辩证否定观及其方法论意义
\\\\
(4)否定之否定规律原理的意义
\\\\

4、唯物辩证法的基本范畴
\\\\
(1)原因和结果
\\\\
(2)必然性和偶然性
\\\\
(3)可能性和现实性
\\\\
(4)现象和本质
\\\\
(5)内容和形式
\\\\

5、唯物辩证法是认识世界和改造世界的根本方法
\\\\
(1)客观辩证法与主观辩证法
\\\\
(2)唯物辩证法与认识方法和工作方法
\\\\
(3)辩证思维的主要方法
\\\\
(4)辩证思维方法与现代科学思维方法
\subsection{第三章\ 认识的本质及其发展规律}
\subsubsection{一、认识与实践}
1、实践是认识的基础
\\\\
(1)实践和认识活动的主体、客体与中介
\\\\
(2)主体与客体的关系及相互作用的过程
\\\\
(3)实践对认识的决定作用
\\\\
(4)认识、理论对实践的指导作用
\\\\

2、认识的本质
\\\\
(1)唯物主义反映论与违心主义先验论的对立
\\\\
(2)辩证唯物主义能动反映论与就唯物主义只管反映论的区别
\\\\
(3)辩证唯物主义能动反映论的主要内容
\\\\

3、认识运动的基本规律
\\\\
(1)从实践到认识:感性认识到理性认识的飞跃
\\\\
(2)从认识到实践:理性认识到实践的飞跃
\\\\
(3)认识过程中的理性因素和非理性因素
\\\\
(4)认识过程的反复性和无限性
\\\\
(5)认识和实践的具体的历史的统一
\subsubsection{二、真理与价值}
1、真理的客观性、绝对性和相对性
\\\\
(1)真理的客观性
\\\\
(2)真理的绝对性和相对性及其辩证关系
\\\\
(3)真理与谬误、成功与失败
\\\\

2、真理的检验标准
\\\\
(1)实践是检验真理的唯一标准
\\\\
(2)实践表针的确定性与不确定性
\\\\

3、真理与价值的辩证统一
\\\\
(1)价值及其特性
\\\\
(2)价值评价及其特点和功能
\\\\
(3)树立正确的价值观
\\\\
(4)真理和价值的辩证统一关系
\subsubsection{三、认识与实践的统一}
1、一切从实际出发、实事求是和解放思想
\\\\

2、在实践中坚持和发展真理;理论创新和实践创新
\\\\

3、认识实践与改造实践、改造客观世界与改造主观世界
\\\\

4、自由与必然
\\\\

5、马克思主义认识论和党的思想路线

\subsection{第四章\ 人类社会及其发展规律}
\subsubsection{一、社会基本矛盾及其运动规律}
1、社会存在与社会意识
\\\\
(1)旧历史观的缺陷与唯物史观创立
\\\\
(2)社会存在和社会意识的含义、构成及作用
\\\\
(3)社会存在与社会意识的辩证关系原理的内容及其意义

2、生产力与生产关系矛盾运动的规律
\\\\
(1)生产力的含义与结构
\\\\
(2)生产关系的含义和内容
\\\\
(3)生产力与生产关系的相互关系
\\\\
(4)生产力与生产关系矛盾运动规律的原理的理论意义和现实意义

3、经济基础与上层建筑矛盾运动的规律
\\\\
(1)经济基础和上层建筑的内涵
\\\\
(2)国家的起源和实质
\\\\
(3)经济基础与上层建筑的相互关系
\\\\
(4)经济基础与上层建筑的矛盾运动及其规律

4、社会形态更替的一般规律及特殊形式
\\\\
(1)社会形态的内涵
\\\\
(2)社会形态更替的统一性和多样性
\\\\
(3)社会形态更替的必然性与人们的历史选择性
\\\\
(4)社会形态更替的前进行与曲折性
\subsubsection{二、社会历史发展的动力}
1、社会基本矛盾是社会发展的根本动力
\\\\
(1)社会基本矛盾的内容
\\\\
(2)社会基本矛盾在社会发展中的作用

2、阶级斗争在阶级社会发展中的作用
\\\\
(1)阶级的产生和本质
\\\\
(2)阶级斗争的根源和作用
\\\\
(3)阶级分析的方法

3、社会革命在阶级社会发展中的作用
\\\\
(1)社会革命的实质和根源
\\\\
(2)革命对社会发展的作用

4、改革的性质和作用

5、科学技术在社会发展中的作用
\\\\
(1)科学技术的含义
\\\\
(2)科学技术革命的作用
\\\\
(3)科学技术社会作用的两重性
\subsubsection{三、人民群众在历史发展中的作用}
1、人民群众是历史的创造者
\\\\
(1)两种历史观在历史创造者问题上的对立
\\\\
(2)现实的人及其活动与社会历史
\\\\
(3)人的本质
\\\\
(4)唯物史观考察历史创造者问题的原则
\\\\
(5)人民群众在创造历史过程中的决定作用
\\\\
(6)群众观点和群众路线

2、个人在社会历史中的作用
\\\\
(1)个人与社会历史
\\\\
(2)历史人物在历史发展中的作用
\\\\
(3)评价历史人物的科学方法
\\\\
(4)正确评价无产阶级领袖

\subsection{第五章\ 资本主义的形成及其本质}
\subsubsection{一、资本主义的形成及以私有制为基础的商品经济的矛盾}
1、资本主义生产关系的产生和资本主义生产方式的形成
\\\\

2、以私有制为基础的商品经济的基本矛盾
\\\\

3、马克思劳动价值论的意义

\subsubsection{二、资本主义经济制度的本质}
1、劳动力成为商品与货币转化为资本
\\\\

2、资本主义所有制
\\\\

3、生产剩余价值是资本主义生产方式的绝对规律
\\\\

4、资本积累
\\\\

5、直奔的循环周转与再生产
\\\\

6、工资与剩余价值的分配
\\\\

7、马克思剩余价值理论的意义
\\\\

8、资本主义的基本矛盾与经济危机

\subsubsection{三、资本主义的政治制度和意识形态}
1、资本主义的国家、政治制度及其本质
\\\\

2、资本主义意识形态的形成及其本质

\subsection{第六章\ 资本主义发展的历史进程}
\subsubsection{一、从自由竞争资本主义到垄断资本主义}
1、资本主义从自由竞争到垄断
\\\\

2、垄断资本主义的发展
\\\\

3、经济全球化及其后果

\subsubsection{二、当代资本主义的新变化}
1、当代资本主义经济政治新变化的表现和特点
\\\\

2、当代资本主义新变化的原因和实质

\subsubsection{三、资本主义的历史地位和发展趋势}
1、资本主义的历史地位
\\\\

2、资本主义为社会主义所代替的历史必然性
\\\\

3、从资本主义向社会主义过渡的复杂性和长期性

\subsection{第七章\ 社会主义社会及其发展}
\subsubsection{一、社会主义制度的建立}
1、空想社会主义的产生、发展和局限性
\\\\

2、科学社会主义的创立
\\\\

3、无产阶级革命的特点、形式
\\\\

4、马克思主义关于无产阶级革命的学说
\\\\

5、俄国十月革命的生理
\\\\

6、列宁领导下的苏维埃俄国对社会主义道路的探索过程
\\\\

7、列宁模式的形成、特征及作用
\\\\

8、社会主义从一国到多国的发展
\\\\

9、20世纪社会主义制度对人类历史发展的巨大贡献性及发展的曲折性
\\\\

10、无产阶级专政的性质、最终目标和国家形式
\\\\

11、社会主义民主
\subsubsection{二、社会主义在实践中发展和完善}
1、社会主义的基本特征及其在实践中的认识深化
\\\\

2、社会主义首先在经济文化相对落后的国家取得胜利的原因
\\\\

3、社会主义建设的艰巨性和长期性
\\\\

4、社会主义发展道路的多样性
\\\\

5、社会主义发展的前进性和曲折性
\\\\

6、社会主义在改革中的自我发展和自我完善
\\\\

\subsubsection{三、马克思主义政党在社会主义事业中的地位和作用}
1、马克思主义政党产生的条件和性质
\\\\

2、马克思主义政党的根本宗旨和组织原则
\\\\

3、马克思主义政党在社会主义革命和建设中的领导地位和作用
\\\\

\subsection{第八章\ 工厂主义是人类最崇高的社会思想}
\subsubsection{一、马克思主义经典作家对共产主义社会的展望}
1、马克思主义经典作家预见未来社会的科学立场和方法
\\\\
(1)在揭示人类社会发展一般规律的基础上致命社会发展的方向
\\\\
(2)在剖析资本主义社会旧世界中阐发未来新世界的特点
\\\\
(3)立足于揭示未来社会的一般特征,而不作详尽的细节描绘
\\\\

2、共产主义社会的基本特征
(1)物质财富极大丰富
\\\\
(2)社会关系高度和谐,人们精神境界极大提高
\\\\
(3)每个人自由而全面的发展,人类从必然王国想自由王国的飞跃

\subsubsection{二、共产主义社会是历史发展的必然趋势}
1、共产主义实现的历史必然性
\\\\

2、实现共产主义的伟大意义
\\\\

3、共产主义实现的长期性
\\\\

4、两个必然和两个绝不会
\\\\

\subsubsection{三、坚持和发展中国特色社会主义,为实现共产主义二奋斗}
1、共产主义的发展阶段
\\\\

2、共产主义远大理想与中国特色社会主义的关系





\section{第二部分\ 毛泽东思想和中国特色社会主义理论体系概论}
\subsection{第一章\ 马克思中国化两大理论成果}
\subsubsection{一、马克思主义中国化及其发展}
1、马克思主义中国化的提出

马克思主义中国化的必要性:

第一,实现马克思主义中国化,是解决中国实际问题的客观需求。

第二,实现马克思主义中国化,是马克思主义理论发展的内在要求
\\\\

2、马克思主义中国化的科学内涵

第一,马克思主义在指导中国革命、建设和改革的实践中实现具体化

第二,把中国革命、建设和改革的实践经验和历史经验,丰富和发展马克思主义理论

第三,把马克思主义植根于中国的优秀文化之中
\\\\

3、马克思主义中国化两大理论成果的关系

首先,毛泽东思想是中国特色社会主义理论体系的重要思想渊源

毛泽东思想所蕴含的马克思主义的立场、观点和方法,为中国特色社会主义理论体系提供了基本遵循。

毛泽东思想关于社会主义建设的理论,为开创和发展中国特色社会主义作了重要的理论准备

其次,中国特色社会主义理论体系在新的历史条件下进一步丰富和发展了毛泽东思想

最后,毛泽东思想和中国特色社会主义理论体系都是马克思列宁主义在中国的运用和发展。

\subsubsection{二、毛泽东思想}
1、毛泽东思想的形成和发展

(1)毛泽东思想形成和发展的时代背景和实践基础

20世纪上半叶,中国人民在苦难中挣扎。1840年鸦片战争以来,太平天国运动、中法战争、中日战争、戊戌变法以及孙中山领导的辛亥革命。1914年帝国列强主义为争夺实力范围发动了第一次世界大战。1917年俄国十月革命的胜利“改变了整个世界历史的方向,划分了整个世界历史的时代”,它使中国反帝反封建的民主革命从旧的世界资产阶级民主革命的一部分,转变成了世界无产阶级社会主义革命的一部分。它为中国送来了马克思列宁主义,帮助中国的先进分子用无产阶级的世界观作为观察国家命运的工具。以毛泽东为主要代表的中国共产党人领导中国革命取得胜利后,又精力了在“冷战”和两大阵营激烈对抗的国际环境,恢复国民经济、进行社会主义改造、探索社会主义建设道路的艰辛历程。毛泽东思想正是在这样的时代背景下形成和发展起来的。

中国共产党领导革命和建设的实践,是毛泽东思想形成的实践基础
\\\\

(2)毛泽东思想形成和发展的历史过程

土地革命战争时期,农村包围城市、武装夺取政权的道路,在荔路上阐述了中国革命的新道路,标志着毛泽东思想开始形成。

遵义会议以后,逐渐形成了比较系统的哲学思想、军事思想、统一战线思想和党的建设思想,科学的阐述了中国新民主主义革命的基本理论、基本路线和基本纲领,精辟地论证了党在民主革命时期的政策和策略,标志着毛泽东思想走向成熟。

1945年党的七大,毛泽东思想被确立为党的指导思想。
\\\\

2、毛泽东思想的主要内容

(1)新民主主义革命理论

反映了中国新民主主义革命客观规律的完备的理论形态,是毛泽东思想达到成熟的主要标志。
\\\\

(2)社会主义革命和社会主义建设理论
\\\\

(3)革命军队建设和军事战略的理论
\\\\

(4)政策和策略的理论
\\\\

(5)思想政治工作和文化工作的理论
\\\\

(6)党的建设理论

1981年十一届六中全会通过的《关于建国以来党的若干历史问题的决议》指出:

毛泽东思想的活的灵魂,是贯穿于上述各个理论组成部分的立场、观点和方法,它们有三个基本方面,即实事求是,群众路线,独立自主。

实事求是,就是一切从实际出发,理论联系实际,坚持在实践中检验真理和发展真理,不断神话对中国国情的认识,研究和把握社会发展的客观规律,找出适合中国情况的革命和建设道路,确定党领导人民改造中国、建设中国的战略策略,实现推动历史前进的目标。它是党的根本思想路线。

群众路线,就是一切为了群众,一切依靠群众,从群众中来,到群众中去,把马克思主义关于人民群众是历史创造者的原理,系统地运用在党的全部活动中。它是当的根本工作路线。

独立自主,就是坚持独立思考,走自己的路,把立足点放在依靠自己力量的基础上,同时积极争取外援,学习外国一切对我们有益的先进事物,它是党的根本政治原则。

\\\\

3、毛泽东思想的历史地位
(1)毛泽东思想是马克思主义中国化第一次历史性飞跃的理论成果
\\\\
(2)毛泽东思想是中国革命和建设的科学指南
\\\\
(3)毛泽东思想是党和人民的宝贵精神财富

\subsubsection{三、中国特色社会主义理论体系}
1、中国特色社会主义理论体系的形成和发展
1978年十一届三中全会,以邓小平为代表的共产党人确立了实事求是的思想路线,破除两个凡是,确立实践是检验真理的唯一标准,把党和国家的工作中心转移到经济建设上来,实行改革开放。

1982年,邓小平在党的十二大上提出了“建设有中国特色的社会主义”的重大命题。

1992年,邓小平发表南方谈话。

1997年,十五大正式使用“邓小平理论”的概念,并作为党的指导思想写入党章。

20世纪80年代末90年代初,东欧剧变、苏联解体,我国发生严重政治风波(六四运动)。以江泽民为主要代表的中国共产党人,十分注重在总结实践经验基础上推进理论创新,形成了三个代表重要思想。党的十六大把三个代表的重要思想写入党章,实现了党的指导思想又一次与时俱进。

(代表着中国先进生产力的发展要求

代表着中国先进文化的前进方向

代表着中国最广大人民的根本利益)

新世纪新阶段,以胡锦涛为总书记的党中央领导,提出来科学发展观,党的十八大将其确立为党必须长期坚持的指导思想写入党章。
\\\\

2、中国特色社会主义理论体系的主要内容

中国特色社会主义理论体系,紧紧围绕什么是社会主义、怎样建设社会主义,建设什么的党、怎样建设党,实现什么样的发展、怎样发展这三大基本问题。

\textcircled{1}中国特色社会主义的思想路线。党的思想路线是一切是从实际出发,理论联系实际,实事求是,在实践中检验真理和发展真理。实事求是是党的思想路线的核心,也是中国特色社会主义理论体系的精髓。

\textcircled{2}建设中国特色社会主义总依据理论。我国正处于并将长期处于社会主义初级阶段,这是当代中国的最大国情。坚持和发展中国特色社会主义,必须始终清醒地以社会主义初级阶段为依据,从中国最大的实际来思考和解决当代中国的一切问题。

\textcircled{3}社会主义本质和建设中国特色社会主义总任务理论。社会主义的本质是解放生产里,发展生产力,消灭剥削,消除两极分化,最终达到共同富裕。解放生产力和发展生产力是社会主义的根本任务,实现社会主义现代化和中华民族伟大复兴,是建设社会主义的总任务。

\textcircled{4}
社会主义改革开放理论

\textcircled{5}
建设中国特色社会主义总布局理论。坚持以经济建设为中心不动摇的前提下,坚持全面协调可持续发展的思路,全面落实经济建设、政治建设、文化建设、社会建设、生态文明建设“五位一体”的总布局。

\textcircled{6}
实现祖国完全统一的理论。按照“一个国家两种制度”的构想,实现祖国和平统一,符合中华民族的根本利益。

\textcircled{7}
中国特色社会主义外交和国际战略理论。和平与发展是当今时代的主题。要适应世界多极化、和经济全球化的发展趋势,始终坚持独立自主和平外交政策,维护国家主权、安全、发展利益,在和平共处五项原则的基础上同所有国家间发展友好合作。

\textcircled{8}
中国特色社会主义建设的根本目的和依靠理论。人民是国家的主人,全心全意为人民服务是我们党的宗旨。发展中国特色社会主义必须坚持以人为本,始终做到发展为了人民、发展依靠人民、发展成果由人民共享。

\textcircled{9}
国防和军队现代化建设理论。国防和军队建设在中国特色社会主义事业总体布局中占有重要地位,是国家安全的坚强后盾。

\textcircled{10}
中国特色社会主义建设的领导核心理论。中国共产党是中国特色社会主义的领导力量和根本保证。

\\\\

3、中国特色社会主义理论体系的最新成果

(1)中华民族伟大复兴的“中国梦”的提出

(2)以习近平为总书记的党中央在心的历史结点上提出并形成了“四个全面”的战略布局,“协调推进全面建成小康社会、全面深化改革、全面推进依法治国、全面从严治党,推动改革开放和社会主义现代化建设迈上新台阶”。

四个全面的战略布局是一个整体,既包括战略目标,又包括战略举措。战略目标是全面建成小康社会,战略举措是全面深化改革、全面推进依法治国、全面从严治党。

2020年全面建成小康社会,是实现中华民族伟大复兴中国梦的“关键一步”;全面深化改革是全面建成小康社会的动力源泉,是实现中国梦的“关键一招”;全面依法治国是全面深化改革的法制保障和全面建成小康社会的重要基石;全面从严治党则是全面建成小康社会、全面深化改革、全面依法治国的必然要求和根本保证。

4、中国特色社会主义理论体系的历史地位

(1)马克思主义中国化第二次历史性飞跃的理论成果

中国特色社会主义理论体系是马克思主义中国化的最新成果,总体上属于马克思主义基本原理同中国具体实际相结合的第二次历史性飞跃的理论成果。一系列新的理论概括,贯彻了马克思主义的群众观点,坚持马克思主义与时俱进的理论品质。

\\\\
(2)新时期全党全国各族人民团结奋斗的共同思想基础

\\\\
(3)实现中华民族伟大复兴中国梦的根本指针

\\\\

\subsubsection{四、实事求是思想路线与马克思主义中国化理论成果的精髓}
1、实事求是思想路线的形式和发展

延安整风和党的七大,实事求是的思想路线在全党得到了确立
\\\\

2、实事求是是思想路线的科学内涵

《中国共产党章程》中一直明确规定:“党的思想路线是一切从实际出发,理论联系实际,实事求是,在实践中检验真理和发展真理。”


\\\\

3、实事求是是马克思主义中国化理论成果的精髓


\subsection{第二章\ 新民主主义革命理论}
\subsubsection{一、新民主主义革命理论的形成}
1、近代中国国情和中国革命的时代特征
\\\\

2、新民主主义革命理论的形成和发展


\subsubsection{二、新民主主义革命的总路线和基本纲领}
1、新民主主义革命的总路线
(1)新民主主义革命的对象
\\\\
(2)新民主主义革命的动力
\\\\
(3)新民主主义革命的领导
\\\\
(4)新民主主义革命的性质和前途
\\\\

2、新民主主义的基本纲领
\\\\
(1)新民主主义的政治纲领
\\\\
(2)新民主主义的经济纲领
\\\\
(3)新民主主义的文化纲领

\subsubsection{三、新民主主义革命的道路和基本经验}
1、中国革命道路理论的主要内容及其依据
\\\\
(1)党对中国革命道路的艰难探索
\\\\
(2)农村保卫城市、武装夺取政权道路的依据及其内容
\\\\
(3)中国革命道路理论的意义
\\\\

2、新民主主义革命的三大法宝
\\\\
(1)统一战线
\\\\
(2)武装斗争
\\\\
(3)党的建设
\\\\
(4)中国共产党对统一战线、武装斗争和党的建设及其相互关系的认识历程
\\\\

3、新民主主义革命理论的意义

\subsection{第三章\ 社会主义改造理论}
\subsubsection{一、从新民主主义到社会主义的转变}
1、新民主主义社会的性质和特点
\\\\

2、党在过渡时期的总路线及其理论依据
\\\\
(1)党在过渡时期的总路线的提出
\\\\
(2)党在过渡时期的总路线的理论依据

\subsubsection{二、社会主义改造道路和历史经验}
1、适合中国特点的社会主义改造道路
\\\\
(1)农业、手工业的社会主义改造
\\\\
(2)资本主义工商业的社会主义改造
\\\\

2、社会主义改造的历史经验

\subsection{三、社会主义制度在中国的确立}
1、社会主义基本制度的确立及其理论依据
\\\\

2、确立社会主义基本制度的重大意义

\subsection{第四章\ 社会主义建设道路初步探索的理论成果}
\subsubsection{一、社会主义建设道路初步探索的重要思想成果}
1、调动一切积极因素为社会主义事业服务的思想

2、正确认识和处理社会主义社会矛盾的思想

3、走中国工业化道路的思想

4、初步探索的其他理论成果

\subsubsection{二、社会主义建设道路初步探索的意义和经验教训}
1、社会主义建设道路初步探索的意义
\\\\
(1)巩固和发展了我国的社会主义制度
\\\\
(2)为开创中国特色社会主义提供了宝贵的经验、理论准备、物质基础
\\\\
(3)丰富了科学社会主义的理论和实践

2、社会主义建设道路初步探索的经验教训

\subsection{第五章\ 建设中国特色社会主义总依据}
\subsubsection{一、社会主义初级阶段理论}
1、社会主义初级阶段理论的形成和发展

第一、马克思主义创始人以及列宁、斯大林对社会主义发展阶段的认识过程

第二、毛泽东对社会主义发展阶段的思考

第三、十一届三种全会以后,逐步作出了我国还处于并将长期处于社会主义初级阶段的科学论断,从而准确的把握了我国的基本国情。

81年十一届六中全会《关于建国以来当的若干历史问题的决议》,第一次提出我国还处于初级的阶段。

87年十三大第一次把社会主义初级阶段作为事关全局的基本国情加以把握,明确了这一问题是党制定路线、方针、政策的出发点和根本依据。

97年十五大进一步强调了社会主义初级阶段问题,制定了党在社会主义初级阶段的基本纲领。

十六大指出,我国正处于并将长期处于社会主义初级阶段。



\\\\

2、社会主义初级阶段的科学含义和主要特征

(1)社会主义初级阶段的科学含义

党十三大明确指出社会主义初级阶段包括两层含义:

第一、我国社会已经是社会主义社会。我们必须坚持而不能离开社会主义。

第二、我国的社会主义社会还处在初级阶段。我们必须从这个实际出发,而不能超越这个阶段。

前一层含义阐明的是初级阶段的社会性质,后一层含义则是阐明了我国现实中社会主义社会的发展程度。
\\
(2)社会主义初级阶段的基本特征

党的十五大指出了我国社会主义初级阶段的九条基本特征:

一是逐步摆脱不发达状态,基本实现社会主义现代化的历史阶段。

二是由农业人口占很大比重、主要依靠手工劳动的农业国,逐步转变为非农业人口占多数、包含现代农业和现代服务业的工业化国家的历史阶段。

三是由自然经济半自然经济占很大比重,逐步变为经济市场化程度较高的历史阶段。

四是由文盲半文盲人口占很大比重、科技教育文化落后,逐步转变为科技教育文化比较发达的历史阶段。

五是由贫困人口占很大比重、人民生活水平比较低,逐步转变为全体人民比较富裕的历史阶段。

六是由地区经济文化很不平衡,通过有先有后的发展,逐步缩小差距的历史阶段。

七是通过改革和探索,建立和完善比较成熟的充满活力的社会主义市场经济体制、社会主义民主政治体制和其他方面体制的历史阶段。

八是广大人民牢固树立建设中国特色社会主义共同理想,自强不息,锐意进取,艰苦奋斗,勤俭建国,在建设物质文明的同时努力建设精神文明的历史阶段。

九是逐步缩小同世界先进水平的差距,在社会主义基础上实现中华民族伟大复兴的历史阶段
\\
(3)社会主义初级阶段理论提出的重大意义
社会主义初级阶段理论是马克思主义关于社会主义发展阶段的新论断,是中国特色社会主义理论体系的立论基础。
社会主义初级阶段理论是建设中国特色社会主义的总依据,是党制定和执行正确路线、方针、政策的基本出发点。
\\\\

3、科学把握我国发展的阶段性特征

社会主义初级阶段是长期性与阶段性统一的动态发展过程。

第一,经济实力显著增强,同时发展中不平衡、不协调、不可持续的问题依然突出。

第二,经济社会发展取得全面进步,同时发展面临心的重大结构性问题,影响发展的体制机制障碍依然存在。

第三,对外开放日益扩大,同时面临的国际竞争日趋激烈。

\\\\

4、社会主义初级阶段的主要矛盾

主要矛盾是人民日益增长的物质文化需要同落后的社会生产之间的矛盾。



\subsubsection{二、社会主义初级阶段的基本路线和基本纲领}

揭示社会主义初级阶段的主要矛盾,反映基本矛盾运动规律的要求,是制定社会主义初级阶段基本路线的客观依据。

1、社会主义初级阶段的基本路线

(1)社会主义初级阶段基本路线确立的历史过程

78年十一届三中全会,初步提出改革开放的方针,针对当时出现的否定社会主义制度和党的领导、否定毛泽东思想的错误思潮,提出了四项基本原则,坚持社会主义道路、坚持无产阶级专政、坚持共产党的领导、坚持马列主义和毛泽东思想。

十二大,邓小平第一次提出建设中国特色的社会主义的概念,后来被概括为党的基本路线核心内容“一个中心、两个基本点”的思想逐步形成。

十二届六中全会根据邓小平的思路,提出来我国现代化建设的总体布局思想即以经济建设为中心,坚定不移第进行经济体制改革,坚定不移地进行政治体制改革,坚定不移地加强精神文明建设。

十三大党在社会主义初级阶段的基本路线:领导和团结全国各族人民,以经济建设为中心,坚持四项基本原则,坚持改革开放,自力更生,艰苦创业,为把我国建设成为富强、民主、文明的社会主义现代化国家而奋斗。

十七大通过党章把“和谐”、“富强”、“民主”、“文明”一起写入了基本路线。
\\\\

(2)党的基本路线的内涵

党的基本路线高度概括了党在社会主义初级阶段的奋斗目标、基本途径和根本保证、领导力量和依靠力量以及实现这一目标的基本方针,既紧紧抓住了中国现阶段的主要矛盾,又体现了运用社会主义基本矛盾运动的规律,全面推动历史进步,实现民富国强、民族振兴的要求。

第一,建设“富强民主文明和谐的社会主义现代化国家”。

第二,“一个中心,两个基本点”

以经济建设为中心,回答了社会主义的根本任务问题,体现了发展生产力的本质要求。

坚持四项基本原则,回答了解放和发展生产力的政治保证问题,体现了社会主义基本制度的要求。

坚持改革开放,会带来社会主义的发展动力和外部条件问题,体现了解放生产力的本质要求。

第三,领导和团结全国各族人民,这是实现社会主义现代化奋斗目标的领导力量和依靠理论。中国国共产党是中国特色社会主义事业的领导核心。

第四,自力更生,艰苦创业。这是我们党的优良传统,也是实现社会主义初级阶段奋斗目标的根本立足点。
\\\\

(3)毫不动摇地坚持党的基本路线

坚持党的基本路线,必须紧紧围绕经济建设这一中心。

坚持党的基本路线,必须把坚持思想基本原则同改革开放结合起来。

坚持四项基本原则和改革开放两个基本点的统一,必须旗帜鲜明第反对资产阶级自由化。

毫不动摇地坚持党的基本路线,把以经济建设为中心同四项基本原则、改革开放这两个基本点统一于建设中国特色社会主义的伟大实践。
\\\\

2、社会主义初级阶段的基本纲领

党的十五大根据社会主义初级阶段基本路线的要求,制定了党在社会主义初级阶段的基本纲领。

建设中国特色社会主义经济,就是在社会主义条件下发展市场经济,不断解放和发展生产力,坚持和完善公有制为主体、多种所有制经济共同发展的基本经济制度,坚持和完善按劳分配为主体、多种分配方式并存的分配制度,坚持和完善对外开放,推动经济持续健康发展,保证人民共享改革和发展成果。

建设中国特色社会主义政治,就是在中国共产党领导下,在人民当家作主的基础上,依法治国,发展社会主义民主政治,建设社会主义法治国家。实现社会安定、政府廉洁高效、全国各族人民团结和睦、生动活泼的政治局面。

建设中国特色社会主义文化,就是以马克思主义为指导,以培育有理想、有道德、有文化、有纪律的公民为目标,发展面向现代化、面向世界、面向未来的,民族的、科学的、大众的社会主义文化。建设社会主义核心价值体系,推动社会主义文化大发展大繁荣。

构建社会主义和谐社会,就是要按照民主法治、公平正义、诚信友爱、充满活力、安定有序、人与自然和谐相处的总要求和共同建设、共同享有的原则,以改善民主为重点,解决好人民最关心、最直接、最现实的利益问题,努力形成全体人民各尽其能、各得其所而又和谐相处的局面。

建设中国特色社会主义生态文明,就是坚持节约资源和保护环境的基本国策,着力推进绿色发展、循环发展、低碳发展,形成节约资源和保护环境的空间格局、产业结构、生产方式、生活方式,从源头上扭转生态环境恶化趋势,努力建设美丽中国,实现中华民族永续发展。

实现社会主义初级阶段基本纲领,必须正确认识和处理最高纲领和最低纲领的辩证统一关系。共产主义是共产党人的理想信念和精神支柱,实现共产主义是无产阶级政党的最高纲领。中国共产党制定的民主革命的纲领、走向社会主义过度的纲领、建设中国特色社会主义的纲领,都是党在特定历史阶段的最低纲领。

\subsection{第六章\ 社会主义本质和建设中国特色社会主义总任务}
\subsubsection{一、社会主义的本质}
1、社会主义本质理论的提出和科学内涵
\\\\
(1)社会主义本质理论的提出
\\\\
(2)社会主义本质的科学内涵
\\\\

2、社会主义本质理论的重要意义

\subsubsection{二、社会主义的根本任务}
1、解放和发展社会生产力
\\\\

2、大力发展科学技术
\\\\

3、坚持科学发展

\subsubsection{三、中国特色社会主义的发展战略}
1、三步走发展战略
\\\\

2、全面建成小康社会
\\\\

3、实现中华民族伟大复兴的中国梦
\\\\
(1)中华民族伟大复兴中国梦的提出
\\\\
(2)中国梦的思想内涵
\\\\
(3)中国梦的实现途径

\subsection{第七章\ 社会主义改革开放理论}
\subsubsection{一、改革开放是发展中国特色社会主义的必由之路}
1、改革开放是决定当代中国命运的关键抉择
\\\\

2、改革开放是社会主义制度的自我完善和发展

\subsubsection{二、鉴定不易地推进改革}
1、全面深化改革
\\\\

2、坚持改革的正确方向
\\\\

3、正确处理改革、发展、稳定的关系

\subsubsection{三、毫不动摇地坚持对外开放}
1、中国的发展离不开世界
\\\\

2、全方位、多层次、宽领域的对外开放
\\\\

3、全面提高开放经济水平



\subsection{第八章\ 建设中国特设社会主义总布局}
\subsubsection{一、建设中国特色社会主义经济}
1、社会主义市场经济理论和经济体制改革
\\\\
(1)党对计算与市场关系的认识经历了一个较长的过程
\\\\
(2)党的十四大之后,我国经济体制改革沿着社会主义市场经济体制的方向加速前进
\\\\
(3)处理好政府和市场的关系
\\\\

2、社会主义初级阶段的基本经济制度
\\\\
(1)社会主义初级阶段基本经济制度的确立及其理论依据
\\\\
(2)毫不动摇地巩固和发展公有制经济
\\\\
(3)毫不动摇鼓励、支持、引导非公有制经济发展
\\\\
(4)积极发展混合所有制经济
\\\\

3、社会主义初级阶段的分配制度
\\\\
(1)坚持按劳分配的主体地位
\\\\
(2)多种分配方式并存
\\\\

4、推动经济持续健康发展
\\\\
(1)推动经济持续健康发展,必须加快转变经济发展方式
\\\\
(2)推动经济持续健康发展,必须坚持走中国特色新型工业化、信息化、城镇化、农业现代化道路
\\\\
(3)推动经济持续健康发展,必须坚持走中国特色自主创新道路,实施创新驱动战略,全面依靠创新驱动发展,提高经济质量和效益
\\\\
(4)推动经济持续健康发展,必须健全城乡发展一体化体制机制

\subsubsection{二、建设中国特色社会主义政治}
1、坚持走中国特色社会主义政治发展道路
\\\\

2、发展社会主义民主
\\\\
(1)人民民主专政
\\\\
(2)人民代表大会制度
\\\\
(3)中国共产党领导的多党合作和政治协商制度是中国特色社会主义的政党制度,也是我国的一项基本政治制度
\\\\
(4)民族区域自治制度
\\\\
(5)基层群众自治制度
\\\\

3、建设社会主义法制国家
\\\\
(1)依法治国基本方略的确立的历程
\\\\
(2)依法治国的内涵及意义
\\\\
(3)加强社会主义法制建设
\\\\

4、推进政治体制改革

\subsubsection{三、建设中国特色社会主义文化}
1、坚持走中国特色社会主义文化发展道路
\\\\

2、建设社会主义核心价值体系
\\\\

3、培育和践行社会主义核心价值观
\\\\

4、加强思想道德建设
\\\\

5、发育教育和科学
\\\\

6、建设社会主义文化强国

\subsubsection{四、建设社会主义和谐社会}
1、建设社会主义和谐社会的总体思路
\\\\

2、保障和改善民生
\\\\

3、创新社会治理体制

\subsubsection{五、建设社会主义生态文明}
1、建设社会主义生态文明的总体要求
\\\\

2、树立生态文明理念
\\\\

3、坚持节约资源和保护环境的基本国策

\subsection{第九章\ 实现祖国完全统一的理论}
\subsubsection{一、实现祖国统一是中华民族的根本利益}
1、维护祖国统一是中华民族的优良传统
\\\\

2、实现祖国完全统一是中华民族伟大复兴的历史任务
\\\\

3、实现祖国完全统一是中国人民不可动摇的坚强意志

\subsubsection{二、和平统一、一国两制的科学构想及其实践}
1、和平统一、一国两制构想的形成和发展
\\\\
(1)从屋里解放台湾到和平解放台湾
\\\\
(2)和平统一、一国两制基本防止的形成和确立
\\\\

2、和平统一、一国两制构想的基本内容和重要意义
\\\\
(1)和平统一、一国两制构想的基本内容
\\\\
(2)和平统一、一国两制构想的重要意义
\\\\

3、一国两制构想在香港、澳门的成功实践
\\\\

4、新形势下对台湾工作方针

\subsection{第十章\ 中国特色社会主义外交和战略国际}
\subsubsection{一、外交和国际战略理论的形成依据}
1、和平与发展是当今时代的主题
\\\\

2、世界多计划和经济全球化趋势在曲折中发展
\\\\
(1)世界多极化在曲折中发展
\\\\
(2)经济全球化趋势深入发展
\\\\

3、抓住和用好战略机遇期
\\\\
(1)来自国际方面的机遇和挑战
\\\\
(2)来自国内方面的机遇和挑战
\subsubsection{二、坚持走和平发展的道路}
1、和平发展道路的根据和重要意义
\\\\

2、坚持独立自主和平外交政策
\\\\
(1)独立自主和平外交政策的形成与发展
\\\\
(2)独立自主和平外交政策的基本原则
\\\\

3、推动建设持久和平、共同繁荣的和谐世界


\subsection{第十一章\ 建设中国特色社会主义的根本目的和依靠力量的理论}
\subsubsection{一、建设中国特色社会主义的根本目的}
1、坚持一切为了人民
\\\\

2、坚持共同富裕的目标
\\\\

3、坚持经济社会发展与人的全面发展的统一

\subsubsection{二、中国特色社会主义建设的依靠理论}
1、工人、农民和知识分子是建设中国特色社会主义事业的根本力量
\\\\
(1)工人阶级是国际的领导阶级
\\\\
(2)农民阶级是人数最多的基本依靠力量
\\\\
(3)知识分子是中国工人阶级的一部分
\\\\

2、新的社会阶层是中国特色社会主义事业的建设者
\\\\

3、巩固和发展全国各组人民的大团结

\subsubsection{三、巩固和发展爱国统一战线}
1、新时期爱国统一战线的内容和基本任务
\\\\

2、加强党对统一战线的领导
\\\\

3、全面贯彻党的民族宗教政策
\\\\
(1)全面贯彻党的民族政策,正确处理民族问题
\\\\
(2)全面贯彻党的宗教政策,正确处理宗教问题

\subsubsection{四、国防和军队现代化建设}
1、人民解放军的性质和作用
\\\\

2、建立巩固国防和强大军队的意义和内涵
\\\\

3、加强军队革命化现代化正规化建设

\subsection{第十二章\ 中国特色社会主义领导核心理论}
\subsubsection{一、党的领导是社会主义现代化建设的根本保证}
1、中国共产党的性质和宗旨
\\\\
(1)中国共产党的性质
\\\\
(2)中国共产党的宗旨
\\\\

2、中国共产党的执政地位是历史和人民的选择
\\\\

3、坚持党的领导必须改善党的领导
\subsubsection{二、全面提高党的建设科学化水平}
1、以改革创新精神推进党的建设新的伟大工程
\\\\
(1)现阶段,党面临的四大考验
\\\\
(2)新形势下,党需要应对的四大危险
\\\\

2、将强党的执政能力建设
\\\\

3、加强党的先进性和纯洁性建设
\\\\

4、建设学习型、服务型、创新型马克思主义执政党

\section{第三部分\ 中国近代史纲要}

\subsection{第一章\ 反对外国侵略的斗争}

\subsubsection{一、资本-帝国主义对中国的侵略及近代中国社会性质的演变}
1、鸦片战争前的中国与世界
\\\\

2、资本-帝国主义对中国的侵略
\\\\


3、近代实惠的半殖民地半封建性质
\\\\


\subsubsection{二、低于外来武装侵略,争取民族独立的斗争}
1、反抗外来侵略的斗争

第一、人民群众的反侵略斗争

第二、爱国官兵的反侵略斗争
\\\\

2、粉碎瓜分中国的图谋

(1)边疆危机和瓜分危机

(2)义和团运动与猎枪灌粉中国图谋的破产

\subsubsection{三、反侵略战争的失败与民族意识的觉醒}

1、反侵略战争失败的及其原因

第一、社会制度的腐败是根本原因

第二、经济技术的落后是近代中国反侵略战争失败的另一个重要原因
\\\\

2、民族意识的觉醒



\subsection{第二章\ 对国家出路的早期探索}

\subsubsection{一、农民群众斗争风暴的起落}

1、太平天国农民战争

(1)太平天国农民战争爆发的原因

(2)金田起义和太平天国的建立

(3)《天朝田亩制度》和《资政新篇》

(4)从天京事变到太平天国的败亡
\\\\


2、太平天国农民斗争的意义和局限

(1)太平天国起义的历史意义

(2)太平天国农民斗争的局限性和教训

\subsubsection{二、洋务运动的兴衰}

1、洋务事业的兴办 
\\\\


2、洋务运动的历史作用及其失败

(1)洋务运动的历史作用

(2)洋务运动失败的原因

封建性。

对外国具有依赖性。

洋务企业的管理腐朽性。
\\\\


\subsubsection{三、维新运动的兴起和夭折}

1、戊戌维新运动的兴起

(1)维新派倡导就往和变法的活动

(2)维新派与守旧派的论战

(3)百日维新
\\\\

2、戊戌维新运动的意义和教训

(1)戊戌维新运动的意义

(2)戊戌维新运动失败的原因和教训

\subsection{第三章\ 辛亥革命与君主专制制度的终结}

\subsubsection{一、举起近代民族民主革命的旗帜}

1、辛亥革命爆发的历史条件
\\\\

2、资产阶级革命派的活动
\\\\

3、三民主义学说和资产阶级共和国方案
\\\\

4、关于革命与改良的辩论

\subsubsection{二、辛亥革命与建立民国}

1、武昌起义与封建帝制的覆灭

(1)武装奇异与保路风潮

(2)武昌首义与各地响应
\\\\

2、中华民国的建立

(1)中华民国的建立

(2)中华民国临时约法
\\\\

3、辛亥革命的历史意义

\subsubsection{三、辛亥革命的失败}

1、封建军阀专制统治的形成
\\\\

2、辛亥革命失败的原因和教训

(1)挽救共和的努力及其受挫

(2)辛亥革命失败的原因和教训


\subsection{第四章\ 开天辟地的大事变}

\subsubsection{一、新文化运动和五四运动}
1、新文化运动与思想解放的潮流

(1)新文化运动的兴起

(2)新文化运动的基本内容

(3)五四以前新文化运动的历史意义

(4)五四以前新文化运动的局限
\\\\

2、十月革命对中国的影响
\\\\

3、五四运动的爆发与中国新民主主义革命的开端

(1)五四运动的发生和发展

(2)中国新民主主义革命的开端

\subsubsection{二、马克思主义广泛传播与中国共产党诞生}

1、中国早起马克思主义思想运动

(1)早起马克思主义者的队伍

(2)早期马克思主义思想运动

(3)新文化运动的发展
\\\\

2、马克思主义与中国工人运动的结合

(1)中国共产党的早期组织

(2)中国共产党早期住在的活动
\\\\

3、中国共产党诞生的历史必然性和伟大意义

(1)中国共产党第一次全国代表大会

(2)中国共产党创建的历史特点

(3)中国共产党成立的伟大意义

\subsubsection{三、中国革命的新局面}

1、制定革命纲领,发动工农运动

(1)制定反帝反封建的民主革命纲领

(2)发动工农群众开展革命运动
\\\\

2、国共合作形成与大革命的兴起

(1)国共合作的形成

(2)大革命的兴起

(3)北伐战争的胜利进展
\\\\

3、大革命的意义、失败原因和教训

\subsection{第五章\ 中国革命的新道路}

\subsubsection{一、对革命新道路的艰苦探索}

1、国民党在全国统治的建立

(1)南京国民政府的成立

(2)国民党政权的性质
\\\\

2、土地革命战争的兴起

(1)大革命失败后的艰难环境

(2)开展武装反抗国民党反动统治的斗争
\\\\

3、农村包围城市、武装夺取政权的道路

(1)对中国革命新道路的探索

(2)反围剿战争与土地革命

\subsubsection{二、中国革命在探索中曲折前进}

1、土地革命战争的发展及其挫折

(1)农村革命根据地的建设

(2)土地革命战争的严重挫折
\\\\

2、遵义会议与中国革命的历史性转折
\\\\

3、红军长征的胜利

\subsection{第六章\ 中华民族的抗日斗争}

\subsubsection{一、日本发动灭亡中国的侵略战争}

1、日本灭亡中国的计划及其实施

(1)从九一八事变到华北事变

(2)卢沟桥事变与日本全面侵华战争
\\\\

2、参保的殖民统治和中华民族的深重灾难

(1)日本在其占领区的残暴统治

(2)侵华日军的严重罪行
\\\\

\subsubsection{二、从局部抗战到全国性抗战}

1、中国共产党举起武装抗日的旗帜
\\\\

2、局部抗战与救亡运动
\\\\

3、一二·九运动
\\\\

4、西安事变
\\\\

5、抗日民族统一战线的形成
\\\\

6、全国性抗战的开始

(1)国共合作,共赴日难

(2)全民族同仇敌忾,奋起抗战

\subsubsection{三、国民党与抗日的正面战场}
1、战略防御阶段的正面战场
\\\\

2、战略相持阶段的正面战场

\subsubsection{四、中国共产党成为抗日战争的中流砥柱}

1、全面抗战的路线和持久站的方针

(1)实行全面的全民族抗战的路线

(2)采取持久战的战略方针
\\\\

2、敌后战场的开辟与游击战争的发展

(1)敌后战场的开辟和发展

(2)游击战争的战略地位和作用
\\\\

3、坚持抗战、团结、进步的方针
\\\\

4、抗日民主根据地的建设

\\\\
5、大后方的抗日民主运动和进步文化工作
\\\\

6、延安整风运动
\\\\

7、中共七大

\subsubsection{五、抗日战争的胜利及其意义}

1、抗日战争的胜利
\\\\

2、中国人民抗日战争在世界反法西斯战争中的地位
\\\\

3、抗日战争胜利的意义、原因和基本经验

(1)抗日战争胜利的意义

(2)抗日战争胜利的原因



\subsection{第七章\ 为新中国而奋斗}

\subsubsection{一、从争取和平民主到进行滋味战争}

1、抗战胜利后的国际国内局势
\\\\

2、中国共产党争取和平民主的斗争
\\\\

3、国民党发动内战和解放区军民的自卫战争

(1)全面内战爆发

(2)以革命战争反对反革命战争

(3)以自卫战争粉碎国民党的军事进攻

\subsubsection{二、国民党政府处于全民的包围中}

1、全国解放战争的胜利发展 

(1)人民解放转入战略进攻

(2)提出“打倒蒋介石,解放全中国”的口号
\\\\

2、土地改革和农民的广泛发动

(1)从《五四指示》到《中国土地法大纲》

(2)土地改革运动的热潮
\\\\

3、第二条战线的形成

(1)国民党通知去的政治经济危机

(2)学生运动的高涨

(3)人民民主运动的发展

\subsubsection{三、中国共产党与民主党派的合作}
1、中国共产党与民主党派的团结合作
\\\\

2、第三条道路的幻灭
\\\\

3、中国共产党领导的多党合作、政治协商格局的形成

\subsubsection{四、创建人民民主专政的新中国}

1、南京国民党政党的覆灭

(1)辽沈、淮海、平津三大战役

(2)人民解放军向全国进军
\\\\

2、人民政协与《共同纲领》

(1)筹建新中国

(2)人民政协会议的召开与《共同纲领》的指定
\\\\

3、中国革命胜利的原因和基本经验

(1)中国革命胜利的原因

(2)中国革命胜利的基本经验



\subsection{第八章\ 社会主义基本制度在中国的确立}

\subsubsection{一、从新民朱主义想社会主义过度的开始}

1、中华人民共和国的成立及其伟大意义
\\\\

2、新民主主义社会的建立
\\\\

3、完成民主革命的遗留任务,恢复国民经济

(1)完成民主革命的遗留任务

(2)恢复和发展国民经济

(3)维护国家主权和安全

(4)加强党的自身建设
\\\\

4、开始向社会主义过度

\subsubsection{二、选择社会主义道路}
1、工业化的任务和发展道路

第一、提出国家工业化的任务

第二、选择社会主义工业化的道路
\\\\

2、过度使其总路线的提出
\\\\

3、实行社会主义改造的必要性和条件

\subsubsection{三、有中国特点的向社会主义过渡的道路}

1、社会主义工业化与社会主义改造同时并举
\\\\

2、农业、手工业合作化运动的发展

(1)农业合作化任务的提出

(2)农业合作化的基本方针

(3)农业合作化的发展和基本完成
\\\\

3、对资本主义工商业赎买政策的实施
\\\\

4、社会主义基本制度在中国的全面确立及其意义



\subsection{第九章\ 社会主义建设在探索中曲折发展}

\subsubsection{一、社会主义建设的初步探索}

1、全面建设和社会主义的开端
\\\\

2、中共八大路线的制定
\\\\

3、探索社会主义建设道路的初步结果

(1)《论十大关系》的发展

(2)《关于正确处理人民内部矛盾的问题》的发表

第一、关于社会主义社会两类不同性质的社会矛盾

第二、关于社会主义社会的基本矛盾

(3)整风运动和反右派斗争

\subsubsection{二、探索中的严重曲折}

1、大跃进及其纠正

(1)大跃进和人民公社化运动的发动

(2)国民经济的调整
\\\\

2、文化大革命及其结束

(1)文化大革命的全面发动

(2)全面内乱的形成

(3)粉碎林彪反革命集团

(4)挫败“四人帮” “组阁”图谋

(5)1975年整顿和文化大革命的结束
\\\\

3、严重的曲折和深刻的教训

(1)文化大革命的性质

(2)文化大革命发生的原因

(3)文化大革命的教训

\subsubsection{三、建设的成就,探索的结果}

1、工业体系和国民经济体系的基本建立
\\\\

2、人民生活水平的提高与文化、医疗、科技事业的发展
\\\\

3、国际地位的提高与国际环境的改善
\\\\

4、探索中形成的建设社会主义的若干重要原则

\subsection{第十章\ 改革开放与现代化建设新时期}
\subsubsection{一、历史性的伟大转折和改革开放的起步}

1、关于真理标准问题的讨论
\\\\

2、中共十一届三中全会的伟大转折
\\\\

3、农村改革的突破性进展
\\\\

4、坚持四项基本原则的提出
\\\\

5、科学评价毛泽东和毛泽东思想



\subsubsection{二、改革开放和现代化建设新局面的展开}

1、中共十二大制定社会主义现代化建设纲领
\\\\

2、改革终点从农村转向城市
\\\\

3、多层次对外开放格局的形成
\\\\

4、中共十三大提出社会主义初级阶段理论和党的基本路线

第一,社会主义初级阶段理论

第二,社会主义初级阶段基本路线

第三,大会制定了经济体质改革和政治体质改革的基本任务和奋斗目标
\\\\



5、“三步走”发展战略的制定和实施

\subsubsection{三、中国特色社会主义事业的跨世纪发展}

1、邓小平发表南方谈话
\\\\

2、中共十四大确立社会主义市场经济体质的改革目标
\\\\

3、中共十五大高举邓小平理论伟大旗帜,提出跨世纪发展战略
\\\\

4、推动解决“三农”问题和推进国有企业的改革
\\\\

5、中国加入世界贸易组织
\\\\

6、“三个代表”重要思想的提出

\subsubsection{四、新的历史起点上推进中国特色社会主义}

1、中共十六大制定全面建设小康社会的行动纲领
\\\\

2、树立和落实科学发展观
\\\\

3、构建社会主义和谐社会
\\\\

4、加强党的执政能力建设和先进性建设
\\\\

5、中共十七大总结改革开放的历史进程和基本经验,提出全面建设小康社会奋斗目标的新要求

\subsubsection{五、坚定不移沿着中国特色社会主义道路前进}

1、中共十八大制定全面建成小康社会的战略部署
\\\\

2、改革开放以来取得的巨大成就及其根本原因和主要经验

(1)改革开放以来取得的巨大成就

(2)取得巨大成绩的根本原因和主要经验
\\\\

3、改革开放前和改革开放后两个历史使其的相互联系和重大区别
\\\\

4、党和人民九十多年奋斗、创造、积累的根本成就
\\\\

5、坚持和发展中国特色社会主义是实现中华民族伟大复兴的必由之路

6、开展党的群众路线教育实践活动

\section{第四部分\ 思想道德修养与法律基础}


\subsection{第一章\ 追求远大理想 坚定崇高信念}
\subsubsection{一、梳理科学的理想信念}
1、理想信念的含义、特征与作用

(1)理想的含义与特征

(2)信念的含义与特征

(3)理想信念的作用

2、确立马克思主义的科学信仰

(1)马克思主义是科学的又是崇高的

(2)马克思主义具有持久的生命力

(3)马克思主义以改造世界为己任

3、树立中国特色社会主义的共同理想

(1)坚定对中国共产党的信任

(2)坚定中国特色社会主义信念

(3)鉴定实现中华民族伟大复兴的信心

\subsubsection{二、理想信念的实现}

1、立志高远与始于足下

(1)立志当高远

(2)立志做大事

(3)立志须躬行
\\\\

2、实现理想的长期性、艰巨性和曲折性

(1)理想的实现是一个过程

(2)正确对待实现理想过程中的顺境与逆境
\\\\

3、在实践中化理想为现实

(1)正确认识理想与现实的关系是实现理想的思想基础

(2)坚定的信念是实现理想的重要条件

(3)用于实践、艰苦奋斗是实现理想的根本途径

\subsection{第二章\ 继承爱国传统 弘扬中国精神}

\subsubsection{一、中华民族的爱国主义传统}

1、爱国主义的科学内涵
\\\\

2、爱国主义的优良传统
\\\\

3、爱国主义的时代价值
\\\\	

(1)爱国主义是中华民族既往开来的精神支柱

(2)爱国主义是维护祖国统一和民族团结的纽带

(3)爱国主义是实现中华民族伟大复兴的动力

(4)爱国主义是个人实现人生价值的力量源泉

\subsubsection{二、新时期的爱国主义}
1、爱国主义与经济全球化

(1)经济全球化形势要弘扬爱国主义

(2)经济全球化与当代大学生的爱国主义
\\\\

2、爱国主义与爱社会主义和拥护祖国统一

(1)爱国主义与爱社会主义的一致性

(2)爱国主义与拥护祖国统一的一致性
\\\\

3、爱国主义与弘扬民族精神

(1)中华民族精神的内涵

(2)大力弘扬和培育民族精神
\\\\

4、爱国主义与弘扬时代精神

(1)改革创新是时代精神的核心

(2)弘扬以改革为核心的时代精神

\subsubsection{三、做忠诚的爱国者}

1、自觉维护国家利益

(1)自觉维护国家利益,就要承担对国家应尽的义务

(2)自觉维护国家利益,就要维护改革发展稳定的大局

(3)自觉维护国家利益,就要树立民族自尊心和自豪感
\\\\

2、促进民族团结

3、维护祖国统一是中华民族的优良传统

4、增强国防观念

(1)增强国防观念是新时期爱国主义的重要内容

(2)增强国防观念的重要意义

5、增强国家安全意识

(1)确立新的国家安全观

(2)自觉履行维护国家安全的义务

\subsection{第三章\ 领悟人生妙谛 创造人生价值}

\subsubsection{一、树立正确的人生观}

1、人生观的科学内涵

2、最求崇高的人生目的

(1)人生目的决定人生道路

(2)人生目的决定人生态度

(3)人生目的决定人生价值标准

3、确立积极进去的人生态度

(1)人生态度与人生观

(2)端正人生态度

4、正确认识人生价值

\subsubsection{二、创造有价值的人生}

1、人生价值的标准与评价

(1)人生的自我价值与社会价值

(2)人生价值的标准

(3)人生价值的评价
\\\\

2、人生价值实现的条件

(1)人生价值实现的社会条件

(2)走与社会实践相结合的道路

\subsubsection{三、科学对待人生}
1、促进自我身心的和谐
\\\\

2、促进个人与他人的和谐

(1)促进个人和他人和谐应坚持的原则

(2)正确认识和处理竞争与合作的关系
\\\\

3、促进个人与社会的和谐

(1)正确认识个体性和社会性的统一关系

(2)正确认识个人需要与社会需要的统一关系

(3)正确认识个人利益与社会利益的统一关系

(4)正确认识享受个人权利与承担社会责任的统一关系
\\\\

4、促进人与自然的和谐

(1)正确认识人对自然的依存关系

(2)科学把我人对自然的改造活动

(3)自觉真爱自然,保护生态


\subsection{第四章\ 学习道德理论 注重道德实践}

\subsubsection{一、道德以及历史发展}
1、道德的本质
\\\\

2、道德的功能与作用

(1)道德的主要功能

(2)道德的社会作用
\\\\

3、道德的历史发展

\subsubsection{二、集成和弘扬中华民族优良道德传统}
1、中华民族优良道德传统的主要内容

(1)注重整体利益、国家利益和民族利益,强调对社会、民族、国家的责任意识和奉献精神

(2)推崇“仁爱”原则,追求人际和谐

(3)讲求谦敬礼让,强调克骄防矜

(4)倡导言行一致,强调恪守诚信

(5)追求精神境界,把道德理想的实现看作是一种高层次的需求

(6)重视道德践履,强调修养的重要性,倡导道德主体要在完善自身中发挥自己的能动作用
\\\\

2、正确对待中华民族道德传统

\subsubsection{三、践行和弘扬社会主要道德}
1、社会主义道德建设与社会主义市场经济
\\\\

2、社会主义道德建设的核心和原则

(1)为人民服务是社会主义道德建设的核心

(2)集体主义是社会主义道德建设的原则

\subsubsection{四、恪守公民基本道德规范}
1、我国公民基本道德规范

(1)公民基本道德规范的主要内容

(2)诚实授信是公民道德建设的重点

(3)大学生与诚信道德
\\\\

2、树立和践行社会主义荣辱观

(1)社会主义荣辱观的科学内涵

(2)社会主义荣辱观体现了社会主义道德建设的客观要求

(3)践行社会主义荣辱观



\subsection{第五章\ 领会法律精神 理解法律体系}
\subsubsection{一、法律的概念及其历史发展}

1、法律的一般含义
\\\\

2、法律的历史发展

\subsubsection{二、社会主义法律精神}
1、社会主义法律的本质
\\\\

2、社会主义法律的作用
(1)法律的规范作用

(2)社会主义法律的社会作用
\\\\

3、社会主义法律的运行
(1)法律制定

(2)法律执行

(3)法律适用

(4)法律遵守

\subsubsection{三、我国宪法确立的基本原则和制度}
1、我国宪法的特征和基本原则

(1)宪法的特征

(2)宪法的基本原则
\\\\

2、我国的国家制度
(1)人民民主专政制度

(2)人民代表大会制度

(3)中国共产党领导的多党合作政治协商制度

(4)民族区域自治制度

(5)基层群众自治制度

(6)基本经济制度
\\\\

3、我国公民的基本权利与基本义务

(1)我国公民的基本权利

(2)我国公民的基本义务

\subsubsection{四、中国特色社会主义法律体系}
1、中国特色社会主义法律体系的形成
\\\\

2、中国特设社会主义法律体系的特征
\\\\

3、中国特色社会主义法律特细的构成

(1)中国特色社会主义法律体系的层次

(2)中国特色社会主义法律体系的部门


\subsection{第六章\ 树立法制理念 维护法律权威}
\subsubsection{一、树立社会主义法治理念}

1、树立社会主义法治理念的重要意义
\\\\

2、社会主义法治理念的基本内容

(1)依法治国

(2)执法为民

(3)公平正义

(4)服务大局

(5)党的领导
\\\\

3、自觉树立社会主义法治理念


\subsubsection{二、培养社会主义法治思维方式}
1、法治思维方式的基本含义和特征
\\\\

2、正确理解法治思维方式
\\\\

3、培养法治思维方式的途径

\subsubsection{三、维护社会主义法律权威}

1、维护法律权威的意义
\\\\

2、保障法律的至上地位
\\\\

3、努力成为法律权威的坚定维护者

\subsection{第七章\ 遵守行为规范 锤炼高尚品格} 

\subsubsection{一、公共生活中的道德与法律}
1、公共生活与公共秩序
\\\\

2、公共生活中的道德规范

(1)文明礼貌

(2)助人为乐

(3)爱护公物

(4)保护环境

(5)遵纪守法
\\\\

3、公共生活中的有关法律

(1)治安管理处罚法

(2)集会游行示威法

(3)道路交通安全法

(4)环境保护法

(5)维护互联网安全的决定


\subsubsection{二、职业生活中的道德与法律}
1、职业生活中的道德规范
(1)爱岗敬业

(2)诚实守信

(3)办事公道

(4)服务群众

(5)奉献社会
\\\\

2、职业生活中的有关法律
(1)劳动法

(2)就业促进法

(3)劳动合同发

(4)劳动争议调解仲裁法
\\\\

3、大学生的择业与创业

(1)正确认识当前我国的就业形势

(2)树立正确的择业观与创业观

(3)在艰苦中锻炼,在实践中成才

\subsubsection{三、婚姻生活中的道德与法律}

1、恋爱、婚姻家庭中的道德规范

(1)恋爱中的道德规范

(2)婚姻家庭生活中的道德规范
\\\\

2、婚姻家庭生活中的有关法律

(1)婚姻法

(2)继承法


\subsubsection{四、个人品德养成的道德与法律}
1、个人品德及其作用
\\\\

2、个人品德与道德修养

(1)加强个人道德修养的自觉性

(2)采取积极有效的道德修养方法

(3)自觉向道德模范学习

(4)积极参与社会实践

3、个人品德与法律修养
\section{第五部分\ 形势与政策以及当代世界经济与政治}

\section{政治常用语}

马克思恩格斯运用唯物史观的基本原理,着重剖析资本主义社会,揭示了资本主义经济发展的规律,形成了科学的剩余价值学说,揭露了资本主义剥削的秘密,论证了社会化大生产与资本主义私有制之间的矛盾,得出了资本主义必然灭亡、社会主要必然胜利的结论。

马克思恩格斯运用辨证唯物主义与历史唯物主义的基本原理,总结各国工人运动的斗争经验,提出来无产阶级的历史使命,指明了实现这一历史使命的方向和道路,阐明了无产阶级革命和无产阶级专政理论以及无产阶级建党学说,从而创立了科学社会主义理论。


透过现象深刻地揭示

全面地

丰富和发展




历史前提

现实基础

包括。。。几个基本组成部分

思想体系

精髓

根本立足点

出发点

以。。。为代表的共产党人
\end{document}
