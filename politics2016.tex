\documentclass{ctexart}
\usepackage[colorlinks,
            linkcolor=black,
            anchorcolor=blue,
            citecolor=green]{hyperref}

\usepackage{geometry}
\geometry{left=2.5cm,right=2.5cm,top=2.5cm,bottom=2.5cm}

\hfuzz=\maxdimen
\tolerance=10000
\hbadness=10000

\begin{document}
\tableofcontents


\section{第一部分\ 马克思主义基本原理概论}

\subsection{第一章\ 马克思主义是关于无产阶级和人类解放的科学}
\subsubsection{一、马克思主义的产生和发展}

1.马克思主义的含义

马克思主义的内容:马克思主义哲学、马克思主义政治经济学和科学社会主义。作为中国共产党和社会主义事业指导思想的马克思主义,既包括由马克思、恩格斯创立和列宁等发展了的马克思主义,也包括中国共产党人将其与中国具体实际相结合,形成的马克思主义中国化理论成果。
\\

马克思主义科学思想体系的精髓: 立场、观点和方法。

马克思主义基本原理,是马克思主义理论体系中最基本、最核心的内容,是对马克思主义的立场、观点和方法的集中概括。

马克思主义的基本立场,是马克思主义观察、分析和解决问题的根本立足点和出发点。

马克思主义的基本观点,是关于自然、社会和人类思维规律的科学认识,是对人类思想成果和社会实践经验的科学总结。

马克思主义的基本方法,是简历在辩证唯物主义和历史唯物主义世界观、方法论基础上的思想方法和工作方法,主要包括实事求是的方法、辩证分析的方法、历史分析的方法、群众路线的方法,等等。
\\\\

2、马克思主义产生的经济社会根源、实践基础和思想渊源

首先,资本主义经济的发展为马克思主义的产生提供了经济、社会历史条件。

其次,无产阶级反对资产阶级的斗争日趋激化。

再次,马克思恩格斯的革命实践和对人类文明成果的继承与创新。

马克思主义是在批判地继承、吸收德国古典哲学、英国古典政治经济学和法国、英国的空想社会主义合理成分的基础上,在深刻分析资本主义社会的发展趋势和科学总结工人阶级斗争实践的基础上,创立和发展起来的。
\\\\

3、马克思主义的创立

马克思1845年春天《关于费尔巴哈的提纲》和马克思、恩格斯1844-1846年合写的《德意志意识形态》,标志着马克思主义的基本形成。

1847年《哲学的贫困》1848年《共产党宣言》标志着马克思主义的公开问世。
\\\\

4、马克思主义在实践中的发展

马克思恩格斯根据实践的发展对自己创建的理论不断充实和完善

其后,列宁等马克思主义者在领导俄国革命中实现丰富和发展。

中国共产党将马克思列宁主义确立为自己的知道思想,并在长期奋斗中坚持把马克思主义基本原理同中国具体实际相结合,发展了马克思主义,先后产生了毛泽东思想和中国特色社会主义理论体系。

\subsubsection{二、马克思主义的鲜明特征}
1、马克思主义科学性和革命行的统一
\\\\

2、马克思主义的哲学基础、政治立场、理论品质和社会理想

(1)马克思主义的哲学基础

辩证唯物主义与历史唯物主义是马克思主义最根本的世界管和方法论。

(2)马克思主义的政治立场

马克思主义政党的一切理论和奋斗都应致力于实现以劳动人民为主题的最广大人民的根本利益,这是马克思主义最鲜明的政治立场。

(3)马克思主义的理论品质

坚持一切从实际出发,理论联系实际,实事求是,在实践中检验真理和发展真理,是马克思主义最重要的理论品质。

(4)马克思主义的社会理想

实现物质财富极大丰富、人品精神境界极大提高、每个人自由而全面发展的共产主义社会,是马克思主义最崇高的社会理想。
\\\\
3、学习和运用马克思主义的意义和方法

学习理论,武装头脑

坚持和弘扬理论联系实际的学风

用科学的态度对待马克思主义



\subsection{第二章\ 世界的物质性及其发展规律}
\subsubsection{一、物质世界和实践}
1、物质世界的客观存在

(1)世界观、方法论和哲学

世界观是人民对整个世界的总体看法和根本观点

方法论是人们认识和改造世界所遵循的根本方法的学说和理论体系

方法论同世界观是同意的,哲学是系统化、理论化的世界观,又是方法论。

(2)哲学基本问题及其内容
思维和存在是哲学的基本问题,这是由哲学作为世界观的学问这一性质和特点所决定的。

\textcircled{1} 思维和存在的关系问题是人类认识和改造世界不能回避的最基本问题,因而也是任何哲学派别都不能回避而必须回答的问题,是解决其他一切这些问题的前提和基础。

\textcircled{2} 思维和存在的问题是划分哲学中基本派别的依据

\textcircled{3} 思维和存在的关系问题也是人类实际生活中的基本问题,它普遍存在于人类的实际生活并决定着人们的思想和行动的出发点和方向。

(3)唯物主义和唯心主义,可知论和不可知论,辩证法和形而上学

(4)马克思主义哲学的创立在哲学史上的伟大变革

(5)马克思主义的物质观及其理论意义

(6)意识的起源和本质

(7)物质和运动,运动和静止,物质运动与时间、空间

物质和运动是不可分割的

运动是物质的存在方式和根本属性

物质是一切运动变化和发展过程的实在基础和承担者

(8)社会的物质性

(9)世界物质统一性原理及其意义
\\\\

2、社会生活本质上是实践的
\\\\
(1)实践的本质、基本特征和基本形式

实践是人类能动地改造世界的客观物质性活动。

实践具有 物质性、自觉能动性、社会历史性 等基本特征。

实践是物质性的活动,具有直接现实性。

实践是人类有意识的活动,体现了自觉的能动性。

实践是社会的历史的活动,具有社会历史性的特点。

实践的基本形式 包括 物质生产实践、社会政治实践、科学文化实践

物质生产实践是人类最基本的实践活动。

社会政治实践是人们社会生活中的一个重要方面

科学文化实践是改造自然和社会的准备性和探索性的实践活动。
\\\\
(2)实践与人的存在

实践是人所独有的活动

实践集中表现了人的本质和社会性。

实践对物质世界的改造是对象性的活动。
\\\\
(3)自然界和人类社会的分化和统一

实践是使物质世界分化为自然界和人类社会的历史前提,又是使自然界和人类社会统一起来的现实基础。
\\\\
(4)人和自然的关系

人与自然的关系是在实践中形成的、始终是处于一定社会关系中的、纳入了社会过程的物质交换过程,是具有社会性的物质交换关系。
\\\\
(5)社会生活的实践本质

社会生活的实践本质体现在三个方面

实践是社会关系形成的基础

实践形成了社会生活的基本领域

实践构成了社会发展的动力。
\\\\

3、客观规律性与主管能动性\\\\
(1)规律及其客观性

规律是本质的联系

规律是必然的联系

规律是稳定的联系
\\\\
规律是客观的。客观是规律的根本特点,它的存在不依赖与人的意识。相反,人的意识及其指导下的实践却要受规律的支配。
\\\\
(2)意识的能动作用及其表现

意识活动具有目的性和计划性
意识活动具有创造性

意识活动具有指导实践改造客观世界的作用

意识具有指导、控制人的行为和生理活动的作用
\\\\
(3)主观能动性与客观规律性的关系

尊重客观规律是发展主观能动性的前提

在尊重客观规律的基础上充分发挥主观能动性
\\\\
(4)正确发挥主观能动作用

从实际出发,努力认识和把握事物的发展规律。

实践是发挥人的主观能动性的基本途径。

主观能动性的发挥,还需要一定的物质基础和物质手段。
\\\\
(5)社会历史趋向与主体选择的关系

社会历史趋向与主体选择的关系是同主观能动性与客观规律性统一原理相关联的问题。

社会历史趋向是属于历史决定论的内容,是社会历史规律的作用。

主体选择是历史主体在社会发展中的能动性和选择性。

社会生活未来发展的多种可能性是主体选择的客观前提,主体的利益和需要是选择的内在根据。

\subsubsection{二、事物的普遍联系与发展}
1、联系和发展的普遍性
\\\\
(1)联系的内涵和特点

联系具有客观性

联系具有普遍性

联系具有多样性
\\\\
(2)事物普遍联系原理的方法论意义

马克思主义关于事物普遍联系的原理,要求人们要善于分析事物的具体联系,确立整体性、开放性的观念,从动态中考察事物的普遍联系。
\\\\
(3)联系与运动、变化、发展
\\\\
(4)发展的实质

发展的实质是新事物的产生和旧事物的灭亡。
\\\\
(5)发展与过程

事物的发展是一个过程。一切事物只有经过一定的过程才能实现自身的发展。

所谓过程是指一切事物都有其产生、发展和转化为其他事物的历史,都有它的过去、现在和未来。
\\\\
(6)唯物辩证法与科学发展观

辩证唯物主义和历史唯物主义是马克思主义最根本的世界论和方法论。

联系和发展是唯物辩证法的总特征。

党的十八大是从马克思主义世界观和方法论的高度,阐述了科学发展观的理论基础和理论意义,明确提出:科学发展观是马克思主义关于发展的世界观和方法论的集中体现。
\\\\

2、对立统一规律是事务发展的根本规律
\\\\
(1)唯物辩证法的实质和核心
\\\\
(2)矛盾的同一性和斗争性及其相互关系
\\\\
(3)矛盾的同一性和斗争性在事物发展中的作用
\\\\
(4)矛盾的同一性和斗争性原理的方法论意义
\\\\
(5)矛盾的普遍性和特殊性的含义及相关关系
\\\\
(6)矛盾的普遍性和特殊性辩证关系原理的意义
\\\\

3、事物发展过程中的两边和质变、可定和否定
\\\\
(1)事物存在的质、量、度
\\\\
(2)事务发展的量变和质变及其辩证关系
\\\\
(3)事物发展过程中的肯定和否定,辩证否定观及其方法论意义
\\\\
(4)否定之否定规律原理的意义
\\\\

4、唯物辩证法的基本范畴
\\\\
(1)原因和结果
\\\\
(2)必然性和偶然性
\\\\
(3)可能性和现实性
\\\\
(4)现象和本质
\\\\
(5)内容和形式
\\\\

5、唯物辩证法是认识世界和改造世界的根本方法
\\\\
(1)客观辩证法与主观辩证法
\\\\
(2)唯物辩证法与认识方法和工作方法
\\\\
(3)辩证思维的主要方法
\\\\
(4)辩证思维方法与现代科学思维方法
\subsection{第三章\ 认识的本质及其发展规律}
\subsubsection{一、认识与实践}
1、实践是认识的基础
\\\\
(1)实践和认识活动的主体、客体与中介
\\\\
(2)主体与客体的关系及相互作用的过程
\\\\
(3)实践对认识的决定作用
\\\\
(4)认识、理论对实践的指导作用
\\\\

2、认识的本质
\\\\
(1)唯物主义反映论与违心主义先验论的对立
\\\\
(2)辩证唯物主义能动反映论与就唯物主义只管反映论的区别
\\\\
(3)辩证唯物主义能动反映论的主要内容
\\\\

3、认识运动的基本规律
\\\\
(1)从实践到认识:感性认识到理性认识的飞跃
\\\\
(2)从认识到实践:理性认识到实践的飞跃
\\\\
(3)认识过程中的理性因素和非理性因素
\\\\
(4)认识过程的反复性和无限性
\\\\
(5)认识和实践的具体的历史的统一
\subsubsection{二、真理与价值}
1、真理的客观性、绝对性和相对性
\\\\
(1)真理的客观性
\\\\
(2)真理的绝对性和相对性及其辩证关系
\\\\
(3)真理与谬误、成功与失败
\\\\

2、真理的检验标准
\\\\
(1)实践是检验真理的唯一标准
\\\\
(2)实践表针的确定性与不确定性
\\\\

3、真理与价值的辩证统一
\\\\
(1)价值及其特性
\\\\
(2)价值评价及其特点和功能
\\\\
(3)树立正确的价值观
\\\\
(4)真理和价值的辩证统一关系
\subsubsection{三、认识与实践的统一}
1、一切从实际出发、实事求是和解放思想
\\\\

2、在实践中坚持和发展真理;理论创新和实践创新
\\\\

3、认识实践与改造实践、改造客观世界与改造主观世界
\\\\

4、自由与必然
\\\\

5、马克思主义认识论和党的思想路线

\subsection{第四章\ 人类社会及其发展规律}
\subsubsection{一、社会基本矛盾及其运动规律}
1、社会存在与社会意识
\\\\
(1)旧历史观的缺陷与唯物史观创立
\\\\
(2)社会存在和社会意识的含义、构成及作用
\\\\
(3)社会存在与社会意识的辩证关系原理的内容及其意义

2、生产力与生产关系矛盾运动的规律
\\\\
(1)生产力的含义与结构
\\\\
(2)生产关系的含义和内容
\\\\
(3)生产力与生产关系的相互关系
\\\\
(4)生产力与生产关系矛盾运动规律的原理的理论意义和现实意义

3、经济基础与上层建筑矛盾运动的规律
\\\\
(1)经济基础和上层建筑的内涵
\\\\
(2)国家的起源和实质
\\\\
(3)经济基础与上层建筑的相互关系
\\\\
(4)经济基础与上层建筑的矛盾运动及其规律

4、社会形态更替的一般规律及特殊形式
\\\\
(1)社会形态的内涵
\\\\
(2)社会形态更替的统一性和多样性
\\\\
(3)社会形态更替的必然性与人们的历史选择性
\\\\
(4)社会形态更替的前进行与曲折性
\subsubsection{二、社会历史发展的动力}
1、社会基本矛盾是社会发展的根本动力
\\\\
(1)社会基本矛盾的内容
\\\\
(2)社会基本矛盾在社会发展中的作用

2、阶级斗争在阶级社会发展中的作用
\\\\
(1)阶级的产生和本质
\\\\
(2)阶级斗争的根源和作用
\\\\
(3)阶级分析的方法

3、社会革命在阶级社会发展中的作用
\\\\
(1)社会革命的实质和根源
\\\\
(2)革命对社会发展的作用

4、改革的性质和作用

5、科学技术在社会发展中的作用
\\\\
(1)科学技术的含义
\\\\
(2)科学技术革命的作用
\\\\
(3)科学技术社会作用的两重性
\subsubsection{三、人民群众在历史发展中的作用}
1、人民群众是历史的创造者
\\\\
(1)两种历史观在历史创造者问题上的对立
\\\\
(2)现实的人及其活动与社会历史
\\\\
(3)人的本质
\\\\
(4)唯物史观考察历史创造者问题的原则
\\\\
(5)人民群众在创造历史过程中的决定作用
\\\\
(6)群众观点和群众路线

2、个人在社会历史中的作用
\\\\
(1)个人与社会历史
\\\\
(2)历史人物在历史发展中的作用
\\\\
(3)评价历史人物的科学方法
\\\\
(4)正确评价无产阶级领袖

\subsection{第五章\ 资本主义的形成及其本质}
\subsubsection{一、资本主义的形成及以私有制为基础的商品经济的矛盾}
1、资本主义生产关系的产生和资本主义生产方式的形成
\\\\

2、以私有制为基础的商品经济的基本矛盾
\\\\

3、马克思劳动价值论的意义

\subsubsection{二、资本主义经济制度的本质}
1、劳动力成为商品与货币转化为资本
\\\\

2、资本主义所有制
\\\\

3、生产剩余价值是资本主义生产方式的绝对规律
\\\\

4、资本积累
\\\\

5、直奔的循环周转与再生产
\\\\

6、工资与剩余价值的分配
\\\\

7、马克思剩余价值理论的意义
\\\\

8、资本主义的基本矛盾与经济危机

\subsubsection{三、资本主义的政治制度和意识形态}
1、资本主义的国家、政治制度及其本质
\\\\

2、资本主义意识形态的形成及其本质

\subsection{第六章\ 资本主义发展的历史进程}
\subsubsection{一、从自由竞争资本主义到垄断资本主义}
1、资本主义从自由竞争到垄断
\\\\

2、垄断资本主义的发展
\\\\

3、经济全球化及其后果

\subsubsection{二、当代资本主义的新变化}
1、当代资本主义经济政治新变化的表现和特点
\\\\

2、当代资本主义新变化的原因和实质

\subsubsection{三、资本主义的历史地位和发展趋势}
1、资本主义的历史地位
\\\\

2、资本主义为社会主义所代替的历史必然性
\\\\

3、从资本主义向社会主义过渡的复杂性和长期性

\subsection{第七章\ 社会主义社会及其发展}
\subsubsection{一、社会主义制度的建立}
1、空想社会主义的产生、发展和局限性
\\\\

2、科学社会主义的创立
\\\\

3、无产阶级革命的特点、形式
\\\\

4、马克思主义关于无产阶级革命的学说
\\\\

5、俄国十月革命的生理
\\\\

6、列宁领导下的苏维埃俄国对社会主义道路的探索过程
\\\\

7、列宁模式的形成、特征及作用
\\\\

8、社会主义从一国到多国的发展
\\\\

9、20世纪社会主义制度对人类历史发展的巨大贡献性及发展的曲折性
\\\\

10、无产阶级专政的性质、最终目标和国家形式
\\\\

11、社会主义民主
\subsubsection{二、社会主义在实践中发展和完善}
1、社会主义的基本特征及其在实践中的认识深化
\\\\

2、社会主义首先在经济文化相对落后的国家取得胜利的原因
\\\\

3、社会主义建设的艰巨性和长期性
\\\\

4、社会主义发展道路的多样性
\\\\

5、社会主义发展的前进性和曲折性
\\\\

6、社会主义在改革中的自我发展和自我完善
\\\\

\subsubsection{三、马克思主义政党在社会主义事业中的地位和作用}
1、马克思主义政党产生的条件和性质
\\\\

2、马克思主义政党的根本宗旨和组织原则
\\\\

3、马克思主义政党在社会主义革命和建设中的领导地位和作用
\\\\

\subsection{第八章\ 工厂主义是人类最崇高的社会思想}
\subsubsection{一、马克思主义经典作家对共产主义社会的展望}
1、马克思主义经典作家预见未来社会的科学立场和方法
\\\\
(1)在揭示人类社会发展一般规律的基础上致命社会发展的方向
\\\\
(2)在剖析资本主义社会旧世界中阐发未来新世界的特点
\\\\
(3)立足于揭示未来社会的一般特征,而不作详尽的细节描绘
\\\\

2、共产主义社会的基本特征
(1)物质财富极大丰富
\\\\
(2)社会关系高度和谐,人们精神境界极大提高
\\\\
(3)每个人自由而全面的发展,人类从必然王国想自由王国的飞跃

\subsubsection{二、共产主义社会是历史发展的必然趋势}
1、共产主义实现的历史必然性
\\\\

2、实现共产主义的伟大意义
\\\\

3、共产主义实现的长期性
\\\\

4、两个必然和两个绝不会
\\\\

\subsubsection{三、坚持和发展中国特色社会主义,为实现共产主义二奋斗}
1、共产主义的发展阶段
\\\\

2、共产主义远大理想与中国特色社会主义的关系


\section{第二部分\ 毛泽东思想和中国特色社会主义理论体系概论}
\subsection{第一章\ 马克思中国化两大理论成果}
\subsubsection{一、马克思主义中国化及其发展}
1、马克思主义中国化的提出
\\\\

2、马克思主义中国化的科学内涵
\\\\

3、马克思主义中国化两大理论成果的关系

\subsubsection{二、毛泽东思想}
1、毛泽东思想的形成和发展
\\\\
(1)毛泽东思想形成和发展的时代背景和实践基础
\\\\
(2)毛泽东思想形成和发展的历史过程
\\\\

2、毛泽东思想的主要内容
(1)新民主主义革命理论
\\\\
(2)社会主义革命和社会主义建设理论
\\\\
(3)革命军队建设和军事战略的理论
\\\\
(4)政策和策略的理论
\\\\
(5)思想政治工作和文化工作的理论
\\\\
(6)党的建设理论
\\\\

3、毛泽东思想的历史地位
(1)毛泽东思想是马克思主义中国化第一次历史性飞跃的理论成果
\\\\
(2)毛泽东思想是中国革命和建设的科学指南
\\\\
(3)毛泽东思想是党和人民的宝贵精神财富

\subsubsection{三、中国特色社会主义理论体系}
1、中国特色社会主义理论体系的形成和发展
1978年十一届三中全会,以邓小平为代表的共产党人确立了实事求是的思想路线,破除两个凡是,确立实践是检验真理的唯一标准,把党和国家的工作中心转移到经济建设上来,实行改革开放。

1992,邓小平发表南方谈话。

1997年,十五大正式使用“邓小平理论”的概念,并作为党的指导思想写入党章。

20世纪80年代末90年代初,东欧剧变、苏联解体,我国发生严重政治风波(六四运动)。以江泽民为主要代表的中国共产党人,十分注重在总结实践经验基础上推进理论创新,形成了三个代表重要思想。党的十六大把三个代表的重要思想写入党章,实现了党的指导思想又一次与时俱进。

(代表着中国先进生产力的发展要求

代表着中国先进文化的前进方向

代表着中国最广大人民的根本利益)

新世纪新阶段,以胡锦涛为总书记的党中央领导,提出来科学发展观,党的十八大将其确立为党必须长期坚持的指导思想写入党章。
\\\\

2、中国特色社会主义理论体系的主要内容

\textcircled{1}中国特色社会主义的思想路线。党的思想路线是一切是从世纪出发,理论联系实际,实事求是,在实践中检验真理和发展真理。实事求是是党的思想路线的核心,也是中国特色社会主义理论体系的精髓。

\textcircled{2}建设中国特色社会主义总依据理论。我国正处于并将长期处于社会主义初级阶段,这是当代中国的最大国情。坚持和发展中国特色社会主义,必须始终清醒地以社会主义初级阶段为依据,从中国最大的实际来思考和解决当代中国的一切问题。

\textcircled{3}社会主义本质和建设中国特色社会主义总任务理论。社会主义的本质是解放生产里,发展生产力,消灭剥削,消除两极分化,最终达到共同富裕。解放生产力和发展生产力是社会主义的根本任务,实现社会主义现代化和中华民族伟大复兴,是建设社会主义的总任务。

\textcircled{4}
\textcircled{5}
\textcircled{6}
\textcircled{7}
\textcircled{8}
\textcircled{9}
\textcircled{10}
\\\\

3、中国特色社会主义理论体系的历史地位
(1)马克思主义中国化第二次历史性飞跃的理论成果
\\\\
(2)新时期全党全国各族人民团结奋斗的共同思想基础
\\\\
(3)实现中华民族伟大复兴中国梦的根本指针


\subsubsection{四、实事求是思想路线与马克思主义中国化理论成果的精髓}
1、实事求是思想路线的形式和发展
\\\\

2、实事求是是思想路线的科学内涵
\\\\

3、实事求是是马克思主义中国化理论成果的精髓

\subsection{第二章\ 新民主主义革命理论}
\subsubsection{一、新民主主义革命理论的形成}
1、近代中国国情和中国革命的时代特征
\\\\

2、新民主主义革命理论的形成和发展


\subsubsection{二、新民主主义革命的总路线和基本纲领}
1、新民主主义革命的总路线
(1)新民主主义革命的对象
\\\\
(2)新民主主义革命的动力
\\\\
(3)新民主主义革命的领导
\\\\
(4)新民主主义革命的性质和前途
\\\\

2、新民主主义的基本纲领
\\\\
(1)新民主主义的政治纲领
\\\\
(2)新民主主义的经济纲领
\\\\
(3)新民主主义的文化纲领

\subsubsection{三、新民主主义革命的道路和基本经验}
1、中国革命道路理论的主要内容及其依据
\\\\
(1)党对中国革命道路的艰难探索
\\\\
(2)农村保卫城市、武装夺取政权道路的依据及其内容
\\\\
(3)中国革命道路理论的意义
\\\\

2、新民主主义革命的三大法宝
\\\\
(1)统一战线
\\\\
(2)武装斗争
\\\\
(3)党的建设
\\\\
(4)中国共产党对统一战线、武装斗争和党的建设及其相互关系的认识历程
\\\\

3、新民主主义革命理论的意义

\subsection{第三章\ 社会主义改造理论}
\subsubsection{一、从新民主主义到社会主义的转变}
1、新民主主义社会的性质和特点
\\\\

2、党在过渡时期的总路线及其理论依据
\\\\
(1)党在过渡时期的总路线的提出
\\\\
(2)党在过渡时期的总路线的理论依据

\subsubsection{二、社会主义改造道路和历史经验}
1、适合中国特点的社会主义改造道路
\\\\
(1)农业、手工业的社会主义改造
\\\\
(2)资本主义工商业的社会主义改造
\\\\

2、社会主义改造的历史经验

\subsection{三、社会主义制度在中国的确立}
1、社会主义基本制度的确立及其理论依据
\\\\

2、确立社会主义基本制度的重大意义

\subsection{第四章\ 社会主义建设道路初步探索的理论成果}
\subsubsection{一、社会主义建设道路初步探索的重要思想成果}
1、调动一切积极因素为社会主义事业服务的思想

2、正确认识和处理社会主义社会矛盾的思想

3、走中国工业化道路的思想

4、初步探索的其他理论成果

\subsubsection{二、社会主义建设道路初步探索的意义和经验教训}
1、社会主义建设道路初步探索的意义
\\\\
(1)巩固和发展了我国的社会主义制度
\\\\
(2)为开创中国特色社会主义提供了宝贵的经验、理论准备、物质基础
\\\\
(3)丰富了科学社会主义的理论和实践

2、社会主义建设道路初步探索的经验教训

\subsection{第五章\ 建设中国特色社会主义总依据}
\subsubsection{一、社会主义初级阶段理论}
1、社会主义初级阶段理论的形成和发展
\\\\

2、社会主义初级阶段的科学含义和主要特征
\\\\
(1)社会主义初级阶段的科学含义
\\\\
(2)社会主义初级阶段的基本特征
\\\\
(3)社会主义初级阶段理论提出的重大意义
\\\\

3、科学把握我国发展的阶段性特征

\subsubsection{二、社会主义初级阶段的基本路线和基本纲领}
1、社会主义初级阶段的主要矛盾
\\\\

2、社会主义初级阶段的基本路线
\\\\
(1)社会主义初级阶段基本路线确立的历史过程
\\\\
(2)党的基本路线的内涵
\\\\
(3)毫不动摇地坚持党的基本路线
\\\\

3、社会主义初级阶段的基本纲领

\subsection{第六章\ 社会主义本质和建设中国特色社会主义总任务}
\subsubsection{一、社会主义的本质}
1、社会主义本质理论的提出和科学内涵
\\\\
(1)社会主义本质理论的提出
\\\\
(2)社会主义本质的科学内涵
\\\\

2、社会主义本质理论的重要意义

\subsubsection{二、社会主义的根本任务}
1、解放和发展社会生产力
\\\\

2、大力发展科学技术
\\\\

3、坚持科学发展

\subsubsection{三、中国特色社会主义的发展战略}
1、三步走发展战略
\\\\

2、全面建成小康社会
\\\\

3、实现中华民族伟大复兴的中国梦
\\\\
(1)中华民族伟大复兴中国梦的提出
\\\\
(2)中国梦的思想内涵
\\\\
(3)中国梦的实现途径

\subsection{第七章\ 社会主义改革开放理论}
\subsubsection{一、改革开放是发展中国特色社会主义的必由之路}
1、改革开放是决定当代中国命运的关键抉择
\\\\

2、改革开放是社会主义制度的自我完善和发展

\subsubsection{二、鉴定不易地推进改革}
1、全面深化改革
\\\\

2、坚持改革的正确方向
\\\\

3、正确处理改革、发展、稳定的关系

\subsubsection{三、毫不动摇地坚持对外开放}
1、中国的发展离不开世界
\\\\

2、全方位、多层次、宽领域的对外开放
\\\\

3、全面提高开放经济水平



\subsection{第八章\ 建设中国特设社会主义总布局}
\subsubsection{一、建设中国特色社会主义经济}
1、社会主义市场经济理论和经济体制改革
\\\\
(1)党对计算与市场关系的认识经历了一个较长的过程
\\\\
(2)党的十四大之后,我国经济体制改革沿着社会主义市场经济体制的方向加速前进
\\\\
(3)处理好政府和市场的关系
\\\\

2、社会主义初级阶段的基本经济制度
\\\\
(1)社会主义初级阶段基本经济制度的确立及其理论依据
\\\\
(2)毫不动摇地巩固和发展公有制经济
\\\\
(3)毫不动摇鼓励、支持、引导非公有制经济发展
\\\\
(4)积极发展混合所有制经济
\\\\

3、社会主义初级阶段的分配制度
\\\\
(1)坚持按劳分配的主体地位
\\\\
(2)多种分配方式并存
\\\\

4、推动经济持续健康发展
\\\\
(1)推动经济持续健康发展,必须加快转变经济发展方式
\\\\
(2)推动经济持续健康发展,必须坚持走中国特色新型工业化、信息化、城镇化、农业现代化道路
\\\\
(3)推动经济持续健康发展,必须坚持走中国特色自主创新道路,实施创新驱动战略,全面依靠创新驱动发展,提高经济质量和效益
\\\\
(4)推动经济持续健康发展,必须健全城乡发展一体化体制机制

\subsubsection{二、建设中国特色社会主义政治}
1、坚持走中国特色社会主义政治发展道路
\\\\

2、发展社会主义民主
\\\\
(1)人民民主专政
\\\\
(2)人民代表大会制度
\\\\
(3)中国共产党领导的多党合作和政治协商制度是中国特色社会主义的政党制度,也是我国的一项基本政治制度
\\\\
(4)民族区域自治制度
\\\\
(5)基层群众自治制度
\\\\

3、建设社会主义法制国家
\\\\
(1)依法治国基本方略的确立的历程
\\\\
(2)依法治国的内涵及意义
\\\\
(3)加强社会主义法制建设
\\\\

4、推进政治体制改革

\subsubsection{三、建设中国特色社会主义文化}
1、坚持走中国特色社会主义文化发展道路
\\\\

2、建设社会主义核心价值体系
\\\\

3、培育和践行社会主义核心价值观
\\\\

4、加强思想道德建设
\\\\

5、发育教育和科学
\\\\

6、建设社会主义文化强国

\subsubsection{四、建设社会主义和谐社会}
1、建设社会主义和谐社会的总体思路
\\\\

2、保障和改善民生
\\\\

3、创新社会治理体制

\subsubsection{五、建设社会主义生态文明}
1、建设社会主义生态文明的总体要求
\\\\

2、树立生态文明理念
\\\\

3、坚持节约资源和保护环境的基本国策

\subsection{第九章\ 实现祖国完全统一的理论}
\subsubsection{一、实现祖国统一是中华民族的根本利益}
1、维护祖国统一是中华民族的优良传统
\\\\

2、实现祖国完全统一是中华民族伟大复兴的历史任务
\\\\

3、实现祖国完全统一是中国人民不可动摇的坚强意志

\subsubsection{二、和平统一、一国两制的科学构想及其实践}
1、和平统一、一国两制构想的形成和发展
\\\\
(1)从屋里解放台湾到和平解放台湾
\\\\
(2)和平统一、一国两制基本防止的形成和确立
\\\\

2、和平统一、一国两制构想的基本内容和重要意义
\\\\
(1)和平统一、一国两制构想的基本内容
\\\\
(2)和平统一、一国两制构想的重要意义
\\\\

3、一国两制构想在香港、澳门的成功实践
\\\\

4、新形势下对台湾工作方针

\subsection{第十章\ 中国特色社会主义外交和战略国际}
\subsubsection{一、外交和国际战略理论的形成依据}
1、和平与发展是当今时代的主题
\\\\

2、世界多计划和经济全球化趋势在曲折中发展
\\\\
(1)世界多极化在曲折中发展
\\\\
(2)经济全球化趋势深入发展
\\\\

3、抓住和用好战略机遇期
\\\\
(1)来自国际方面的机遇和挑战
\\\\
(2)来自国内方面的机遇和挑战
\subsubsection{二、坚持走和平发展的道路}
1、和平发展道路的根据和重要意义
\\\\

2、坚持独立自主和平外交政策
\\\\
(1)独立自主和平外交政策的形成与发展
\\\\
(2)独立自主和平外交政策的基本原则
\\\\

3、推动建设持久和平、共同繁荣的和谐世界


\subsection{第十一章\ 建设中国特色社会主义的根本目的和依靠力量的理论}
\subsubsection{一、建设中国特色社会主义的根本目的}
1、坚持一切为了人民
\\\\

2、坚持共同富裕的目标
\\\\

3、坚持经济社会发展与人的全面发展的统一

\subsubsection{二、中国特色社会主义建设的依靠理论}
1、工人、农民和知识分子是建设中国特色社会主义事业的根本力量
\\\\
(1)工人阶级是国际的领导阶级
\\\\
(2)农民阶级是人数最多的基本依靠力量
\\\\
(3)知识分子是中国工人阶级的一部分
\\\\

2、新的社会阶层是中国特色社会主义事业的建设者
\\\\

3、巩固和发展全国各组人民的大团结

\subsubsection{三、巩固和发展爱国统一战线}
1、新时期爱国统一战线的内容和基本任务
\\\\

2、加强党对统一战线的领导
\\\\

3、全面贯彻党的民族宗教政策
\\\\
(1)全面贯彻党的民族政策,正确处理民族问题
\\\\
(2)全面贯彻党的宗教政策,正确处理宗教问题

\subsubsection{四、国防和军队现代化建设}
1、人民解放军的性质和作用
\\\\

2、建立巩固国防和强大军队的意义和内涵
\\\\

3、加强军队革命化现代化正规化建设

\subsection{第十二章\ 中国特色社会主义领导核心理论}
\subsubsection{一、党的领导是社会主义现代化建设的根本保证}
1、中国共产党的性质和宗旨
\\\\
(1)中国共产党的性质
\\\\
(2)中国共产党的宗旨
\\\\

2、中国共产党的执政地位是历史和人民的选择
\\\\

3、坚持党的领导必须改善党的领导
\subsubsection{二、全面提高党的建设科学化水平}
1、以改革创新精神推进党的建设新的伟大工程
\\\\
(1)现阶段,党面临的四大考验
\\\\
(2)新形势下,党需要应对的四大危险
\\\\

2、将强党的执政能力建设
\\\\

3、加强党的先进性和纯洁性建设
\\\\

4、建设学习型、服务型、创新型马克思主义执政党

\section{第三部分\ 中国近代史纲要}
\subsection{第一章\ 反对外国侵略的斗争}
\subsection{第二章\ 对国家出路的早期探索}
\subsection{第三章\ 辛亥革命与君主专制制度的终结}
\subsection{第四章\ 开天辟地的大事变}
\subsection{第五章\ 中国革命的新道路}
\subsection{第六章\ 中华民族的抗日斗争}
\subsection{第七章\ 为新中国而奋斗}
\subsection{第八章\ 社会主义基本制度在中国的确立}
\subsection{第九章\ 社会主义建设在探索中曲折发展}
\subsection{第十章\ 改革开放与现代化建设新时期}

\section{第四部分\ 思想道德修养与法律基础}
\subsection{第一章\ 追求远大理想 坚定崇高信念}
\subsection{第二章\ 继承爱国传统 弘扬中国精神}
\subsection{第三章\ 领悟人生妙谛 创造人生价值}
\subsection{第四章\ 学习道德理论 注重道德实践}
\subsection{第五章\ 领会法律精神 理解法律体系}
\subsection{第六章\ 树立法制理念 维护法律权威}
\subsection{第七章\ 遵守行为规范 锤炼高尚品格} 

\section{第五部分\ 形势与政策以及当代世界经济与政治}

\section{政治常用语}
历史前提

现实基础

包括。。。几个基本组成部分

思想体系

精髓

根本立足点

出发点

以。。。为代表的共产党人
\end{document}
